\documentstyle[11pt]{article}
\parindent 0em 
\begin{document}
\title{ Combining a Specification with an User Interface in the
IFAD VDM-SL Toolbox}
\author{Brigitte Froehlich  TU Graz}
\date{1996--02--21}
\maketitle

\section{Advanced Database Example}
This example combines a specification of a map database with an
user interface implemented in tcl-tk and C++ and can be executed by
the interpreter of the IFAD VDM-SL Toolbox. The most important feature 
of this application is such that the user determines the sequence of 
modifications on the database by an user interface. 

The database is implemented with the map type and provides functions
to insert an item pair( {\sl Insert}), to look for a certain key
({\sl Lookup}), to check either the key is defined in the database 
({\sl Defined}) and to remove an item pair ({\sl Remove}). 
In order to support prototyping with an user interface,
a specification has been developed which build around this functions,
I/0 functions implemented in code and defined also in a VDM-SL
implementation module. This specification  contains a function 
({\tt GUI\_Menu()}) which simulates a kind of a menu structure in order 
to direct the control flow. The menu is implemented as part of the the user 
interface and  enables the user to select the next function to execute. 

The interpreter of the Toolbox uses the functions provided by the user 
interface also in order to get the argument value for key and data 
if these are required by the executed function of the map database. 
The result values of the function other than the database are displayed  
on the screen as well as error messages and status information.

\section{Integration into the IFAD VDM-SL Toolbox}
\begin{enumerate}
\item Creating a shared library\\
Update the paths in the Makefile and create the shared library 
{\tt ./my\_gui.so}. The gnu c++ compiler versions 2.7.2 and installations of
tcl7.4 and tk4.0 are required in order to compile c++ the sources.

\item Interpreting the specification\\
In order to execute the specification a VDM-SL script {\tt demo} has been
developed. This script reads in every VDM-SL module of the system, initializes
the system, pushes the module {\tt GuiDB} on top of the module stack and
displays all functions and operations defined in this module.
The function which contains the menu loop is started and from now on the
user direct the flow of control by the user interface.
You will enjoy this new way of executing a specification.

\begin{verbatim}
vdm> script demo

vdm> r MapDB.vdm
Parsing and installing "/temp/bfroeh/phd/paper2/tcl-tk/demo/MapDB.vdm" ...
ERRORS: 0
vdm> r GuiDB.vdm
Parsing and installing "/temp/bfroeh/phd/paper2/tcl-tk/demo/GuiDB.vdm" ...
ERRORS: 0
vdm> unset tex
tex unset
vdm> r gui.vdm
Parsing and installing "/temp/bfroeh/phd/paper2/tcl-tk/demo/gui.vdm" ...
ERRORS: 0
vdm> init
Initializing specification ...
vdm> push GuiDB
Current module is now: GuiDB
vdm> functions
Explicit functions and operations defined in current module:
GUI_Defined  GUI_Insert  GUI_LookUp  GUI_Menu  GUI_Remove  init_database  
vdm> p GUI_Menu()
\end{verbatim}

\item Executing the selected function\\
In order to get the arguments for a certain function of the
 map database specification, the interpreter will always execute a 
function defined in the shared library.
(e.g The function {\tt GUI\_Insert} executes the {\tt Insert} function of the
map database specification. This function takes as arguments a key value, and
data value, which are taken from the user inputs in the user interface.)
The return values of functions which are different from the database, are
displayed in the user interface, too.
\end{enumerate}

\begin{appendix}
\section{An Overview of the Files}
\begin{tabular}{|l|l|}
\hline
\multicolumn{2}{|c|}{\bf VDM-SL Specifications}\\
\hline
MapDB.vdm & Specification of a map database\\
DB.vdm & Extends the map database with the functions of an user interface\\
& implemented in code and imported from the implementation module\\
gui.vdm & Implementation module which contains VDM-SL definitions \\
&  related to the intergrated C++ code\\
\hline
\multicolumn{2}{|c|}{ }\\
\hline
\multicolumn{2}{|c|}{\bf C++ code}\\
\hline
tcl\_init.cc & Functions to generate a tcl interpreter and open a main window \\
my\_gui.cc & Type conversion functions related to the implementation module\\
gui\_func.cc & Functions which generates the user interfaces by evaluating \\
& a tcl script\\
\hline
\multicolumn{2}{|c|}{ }\\
\hline
\multicolumn{2}{|c|}{\bf Header files}\\
\hline
tcl\_init.h &\\
gui\_func.h&\\
\hline
\multicolumn{2}{|c|}{ }\\
\hline
\multicolumn{2}{|c|}{\bf Tcl Scripts}\\
\hline
getdata.tcl & Input of a key vale and data value\\
getkey.tcl & Input of a key value\\
showdata.tcl & Display key value and related data value on the screen\\
showdef.tcl & Display information either the key is defined or is not defined in\\
& the database.\\
menu.tcl & Provides the menu in order to select the next function to execute\\
showmsg.tcl & Display an message or status information related to a function on the \\
&screen.\\
\hline
\multicolumn{2}{|c|}{ }\\
\hline
\multicolumn{2}{|c|}{\bf Auxiliary Files}\\
\hline
Makefile & Makefile for creating the shared library\\
demo & VDM-SL script which parse and initialize the specifications in order \\
& to execute the system\\
info.tex & Describes the example\\
\hline
\end{tabular}
\end{appendix}
\end{document}
