%#######################################################################
%
% Introduction and motivation for open tool support and VDM + UML
%
\section{Connecting UML and VDM++}
%-----------------------------------------------------------------------
%
% Outline
%
\begin{frame}
  \frametitle{Outline}
  \tableofcontents[current]
\end{frame}





%-----------------------------------------------------------------------
\subsection{Transforming VDM++ to UML}
%-----------------------------------------------------------------------

%
% Transformation
%
\frame
{
  \frametitle{The Vienna Development Method}

\begin{center}
	\begin{block}<+->{Not created yet}
	Some thing about VDM
	\end{block}

\end{center}
}


%-----------------------------------------------------------------------
\subsection{Transforming UML to VDM++}
%-----------------------------------------------------------------------




%-----------------------------------------------------------------------
\subsection{Merging}
%-----------------------------------------------------------------------




%-----------------------------------------------------------------------
\subsection{Supported features}
%-----------------------------------------------------------------------

%
% Overview of supported features
%
\frame
{
  \frametitle{Supported features Overview}
\begin{center}
\begin{tabular}{lcc}
  Description        & VDM to UML & UML to VDM \\\hline
  Classes            & X & X  \pause\\
  Abstract Classes   & X & X  \pause\\
  Inheritance        & X & X  \pause\\
  Fields             & X & X  \pause\\
  Static access      & X & X  \pause\\      
  Default Fields     & X & X  \pause\\
  Visibility         & X & X  \pause\\
  Threads            & X & X  \pause\\
  
  Types             & & \\\hline
  set, seq, seq1     & X & X  \pause\\
  Product            & X & X  \pause\\
  Union              & X & X  \pause\\
  Record             & X & X  \pause\\ 
  Optional           & X & X  \pause\\
  Object reference   & X & X  \pause\\
  
  Combinatorial Testing & & \\\hline
  Repeat             & X & X  \pause\\
  Choice             & X & X  \pause\\
  
  
\end{tabular}

\end{center}
}



