%#######################################################################
%
% Introduction and motivation for open tool support and VDM + UML
%
\section{Connecting UML and VDM++}
%-----------------------------------------------------------------------
%
% Outline
%
\begin{frame}
  \frametitle{Outline}
  \tableofcontents[current]
\end{frame}


%-----------------------------------------------------------------------
\subsection{History}
%-----------------------------------------------------------------------
%
% VDM Tools
%
\frame
{
  \frametitle{Existing tool}

VDM Tools
  \begin{itemize}
	%\itemsep=1cm
  		\item<1-> Commercial tool from CSK
  		\item<2-> No editor
  		\item<3-> Includes a UML transformation
  		\begin{itemize}
  			\item Rose VDM Link first release (1997)
  			\item UML 1.4
  		\end{itemize}
	  	
  \end{itemize}


}


\note
{

  \begin{itemize}
	%\itemsep=1cm
  		\item Only tool
  		\item Commercial
  		\item Older UML Transformation Rose VDM Link from (1997)
  		\item UML 1.4
	  	
  \end{itemize}




}


%-----------------------------------------------------------------------
\subsection{New UML contribution}
%-----------------------------------------------------------------------

%
% extending uml trans
%
\frame
{
  \frametitle{Extending UML support}

  \begin{itemize}
	\itemsep=1cm
  		\item<1-> Change to UML version 2.x
  		\item<2-> Extend UML subset for Class Diagrams
  		\item<3-> UML Sequence Diagrams
	  	
  \end{itemize}


}

\note
{

  \begin{itemize}
	%\itemsep=1cm
  		\item Since UML 2 has since been released look into new features and uses
  		\item Extending the subset of UML used to model VDM, active, abstract,x-or association
  		\item Look into new diagrams which could be use full, new VDM feature for testing \textbf{Traces}
	  	
  \end{itemize}

}


%
% Traces
%
\frame
{
  \frametitle{Combinatorial Testing in VDM}

  \begin{itemize}
	%\itemsep=1cm
  		\item<1-> New feature recently introduced to VDM - 2008
  		\item<2-> Automatic execution of a large number of test cases generated from templates in form of trace definitions
  		\item<3-> Add graphical representation of trace definitions in UML as Sequence Diagrams
	  	
  \end{itemize}


}

%
% Traces example
%
\frame
{
  \frametitle{VDM Combinatorial Test example}
\begin{center}
\vdmSpecLineNum{UseStack.vpp}{Traces}{VDM:Collections}

\end{center}
}

%
% Traces example expanded
%
\frame
{
  \frametitle{VDM Combinatorial Test example evaluation}
\begin{columns}
\begin{column}[l]{5cm}

	\begin{center}
	\vdmSpecLineNum{UseStackTrace.vpp}{Traces}{VDM:Collections}
	\end{center}

\end{column}
\begin{column}[r]{5cm}
	\begin{itemize}
		\item<2-> TC1:
		\begin{itemize}
			\item stack.Reset()
			\item stack.Push(2)
			\item stack.Push(5)
		\end{itemize}
		\item<3-> TC2:
		\begin{itemize}
			\item stack.Reset()
			\item stack.Push(8)
			\item stack.Push(8)
			\item stack.Pop()
		\end{itemize}
		\item<4-> ... 14 more
	\end{itemize}

\end{column}
\end{columns}
}


%
% Traces example UML
%
\frame
{
  \frametitle{VDM Combinatorial Test example Sequence Diagram}
  
\begin{columns}
\begin{column}[l]{5cm}

	\begin{center}
	\vdmSpecLineNum{UseStackTrace.vpp}{Traces}{VDM:Collections}
	\end{center}

\end{column}
\begin{column}[r]{5cm}
	
	\begin{center}
	\includegraphics[width=\textwidth]{images/TracesSequenceDiagramEx2.png}%
	\end{center}

\end{column}
\end{columns}
  

}

