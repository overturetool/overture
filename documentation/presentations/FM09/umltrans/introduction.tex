%#######################################################################
%
% Introduction and motivation for open tool support and VDM + UML
%
\section{Introduction}
%-----------------------------------------------------------------------
%
% Motivation
%
\frame
{
  \frametitle{Motivation}

%\begin{block}<+->{Abstract}
%A short introduction to the Overture project showing how to bootstrap the tools for VDM. Short demo of the Editor and Debug execution.
%\end{block}
\begin{center}
  \begin{itemize}
	%\itemsep=1cm
  		\item<1-> Improve tools support for formal methods (VDM++)
  		\item<2-> Only one commercial tool for VDM (VDM Tools) 
	  	\item<3-> Rose VDM Link UML1 (last updated 1997) 
		\item<4-> Best of both worlds VDM - UML.  		
  \end{itemize}
\end{center}
}



%
% Goal
%
\frame
{
  \frametitle{Goal}

\begin{center}
  \begin{itemize}
	%\itemsep=1cm
  		\item<1-> Introduce a connection to UML Class Diagrams and Sequence Diagrams
  		\item<2-> Integrate the tool into Eclipse for easy eccess
	  	
  \end{itemize}
\end{center}
}

%-----------------------------------------------------------------------
\subsection{Tool support}
%-----------------------------------------------------------------------
%
% Improce Tools support
%
\frame
{
  \frametitle{Improve tool support}
\begin{center}
	\begin{block}<+->{Not created yet}
	Some thing about tool support
	\end{block}

\end{center}
}

%-----------------------------------------------------------------------
\subsection{Vienna Development Method}
%-----------------------------------------------------------------------
%
% VDM
%
\frame
{
  \frametitle{The Vienna Development Method}

\begin{center}
	\begin{block}<+->{Not created yet}
	Some thing about VDM
	\end{block}

\end{center}
}

%-----------------------------------------------------------------------
\subsection{Unified Modelling Language}
%-----------------------------------------------------------------------
%
% UML
%
\frame
{
  \frametitle{Unified Modeling Language - UML}

\begin{center}

	\begin{block}<+->{Not created yet}
	Some thing about UML
	\end{block}

\end{center}
}
