%
% Table of conent frame
%
\begin{frame}[c]
	\titlepage
\end{frame}

\begin{frame}[c]
  \frametitle{Outline}
  \tableofcontents %[pausesections] %[current,currentsubsection]
%  \tableofcontents[current]
\end{frame}


%#######################################################################
\section{Overture}
%-----------------------------------------------------------------------
\frame
{
  \frametitle{The Overture project}

\begin{block}<+->{Mission}
	Overture's mission is twofold: 
  \begin{itemize}
  		\item to provide an industrial-strength tool that supports the use of precise abstract models in software development, and 
  		\item to foster an environment that allows researchers and other interested parties to experiment with modifications and extensions to the tool.      

  \end{itemize}
\end{block}

The Overture tools are being developed by volunteers, researchers and students.
}



%-----------------------------------------------------------------------
\subsection{Sites}
%-----------------------------------------------------------------------
\frame
{
  \frametitle{Sites}


\begin{center}
  \begin{itemize}
	%\itemsep=1cm
  		\item Main page (Twiki): www.overturetool.org
  		\item Download page: https://sf.net/projects/overture/
  		\item Documentation: http://overture.sf.net/maven/doc/
  \end{itemize}
\end{center}
}
%-----------------------------------------------------------------------
\subsection{Development environment}
%-----------------------------------------------------------------------
\frame
{
  \frametitle{Development environment}


\begin{center}
  \begin{itemize}
	%\itemsep=1cm
  		\item Maven 2
  		\item Eclipse 3.5 Classic
  		\item DLTK 1.0
  \end{itemize}
\end{center}
}

%#######################################################################
\section{Contributing}
%-----------------------------------------------------------------------
%
% Outline
%
\begin{frame}
  \frametitle{Outline}
  \tableofcontents[current]
\end{frame}


\frame
{
  \frametitle{How to contribute}


\begin{center}
  ...
\end{center}
}

%-----------------------------------------------------------------------
\subsection{Structure}
%-----------------------------------------------------------------------
\frame
{
  \frametitle{Development environment}


\begin{center}
  \begin{itemize}
	%\itemsep=1cm
  		\item Core
  		\item IDE
  \end{itemize}
\end{center}
}

\frame
{
  \frametitle{Check out}


\begin{center}
\small \texttt{svn checkout https://overture.svn.sourceforge.net/svnroot/overture/trunk overture}
\end{center}
}

%#######################################################################
\section{Example}
%-----------------------------------------------------------------------
%
% Outline
%
\begin{frame}
  \frametitle{Outline}
  \tableofcontents[current]
\end{frame}


%-----------------------------------------------------------------------
\subsection{Core}
%-----------------------------------------------------------------------
\frame
{
  \frametitle{Adding a core component}


\begin{center}
 
\end{center}
}

%-----------------------------------------------------------------------
\subsection{Eclipse}
%-----------------------------------------------------------------------


\frame
{
  \frametitle{How to create a GUI}


\begin{center}
 
\end{center}
}

\frame
{
  \frametitle{Wrap core component into a plug-in}


\begin{center}
 
\end{center}
}

\frame
{
  \frametitle{Creating Eclipse plug-in}


\begin{center}
 
\end{center}
}



\begin{frame}[plain,c]
  \begin{center}
	\LARGE \structure{Thank you!}

	\vspace{2cm}
	\href{www.overturetool.org}{www.overturetool.org}
\end{center}
\end{frame}
