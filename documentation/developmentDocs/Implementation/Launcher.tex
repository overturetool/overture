\section{Launcher}
The launcher is the part of the eclipse framework responsible to "launch" the code to run, in our case a model to be interpreted. Associated with a launch is a launch configuration that contains information used to launch the code. The inspiration for the Overture implementation can be found \href{http://www.eclipse.org/articles/Article-Launch-Framework/launch.html}{here}\footnote{\url{http://www.eclipse.org/articles/Article-Launch-Framework/launch.html}} \cite{Szurszewski03}. 
Also connected with the launcher is its UI part which is the window that pops up when a defining a new launch configuration.

\subsection{The Players}

\begin{description}
\item[launchConfigurationType:] it is an extension point where new launch configurations can be declared. A launch configuration describes a way to launch a model; 

\item[launchConfigurationDelegate:] it is a delegate associated with a launchConfigurationType. The delegate is in charge of, using the configuration set for launch, starting up the interpreter process. The configuration contains which is the launch mode (i.e. "run", "debug") among other settings;

\item[launchConfigurationTypeImages:] it is an extension point to select an image associated with a launch configuration type;

\item[launchConfigurationTabGroups:] it is an extension point that defines a tab group associated with a certain configurationType. The tab group is a group of tabs presented when creating a new launch configuration which contain the enables the user to graphically set the settings to be used in the launch (configuring it);

\item[launchShortcuts:] it is an extension point that enables the definition of shortcuts to launch models without configuring the launch, i.e. without bringing up the launch configuration tabs.

\item[launchGroups:] didnt get there yet... not sure if it will be needed.

\end{description}
