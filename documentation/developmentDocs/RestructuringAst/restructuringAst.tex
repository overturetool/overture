\documentclass{overturerep}

\usepackage{graphics}
\usepackage{times}
\usepackage{listings}

\usepackage[usenames,dvipsnames]{color}
\usepackage{graphicx}
\usepackage{latexsym}
\usepackage{longtable} % ,multirow}
\usepackage{hyperref}
\usepackage{float}

\usepackage{geometry}
\usepackage{tikz}
\usetikzlibrary{positioning,shapes,shadows,arrows}

\floatstyle{ruled}
\newfloat{program}{thp}{lop}
\floatname{program}{Extension point}

\newcommand{\vdmtools}{VDMTools}
\newcommand{\vdmstyle}[1]{\texttt{#1}}
\newcounter{exerciseno}

%\newcommand{\Exercise}[1]{%
%    \textbf{Exercise \thechapter.\theexerciseno}
%   \refstepcounter{exerciseno} #1 $\Box$\\ }%}
%\newcommand{\initexercise}{\setcounter{exerciseno}{0}}
%\newcounter{exerciseno}

\newcommand\thebookexercise{\thechapter.\arabic{exerciseno}}
\newenvironment{myexercise}{\par
  \refstepcounter{exerciseno}%
  \indent\textbf{Exercise\ \thebookexercise}\enskip}{$\Box$\\
}
\newenvironment{myhardexercise}{\par
  \refstepcounter{exerciseno}%
  \indent\textbf{Exercise\ \thebookexercise $\star$}\enskip}{$\Box$\\
}
\newcommand{\initexercise}{\setcounter{exerciseno}{0}}
\newenvironment{mysolution}{
} %% This will be replaced by a perl script extracting the solutions
  %% and inserting them automatically into the solutions chapter.
 

%\newcommand{\insertcommentedvdm*}[2]{}
  %% This macro is identified by perl script which will move the
  %% parameter to the solutions chapter.
% definition of VDM++, JavaCC, JJTree, JTB, ANTLR and SableCC for listings
\lstdefinelanguage{VDM++}
  {morekeywords={\#act, \#active, \#fin, \#req, \#waiting, abs, all, allsuper, always, and, answer, 
     assumption, async, atomic, be, bool, by, card, cases, char, class, comp, compose, conc, cycles,
     dcl, def, del, dinter, div, do, dom, dunion, duration, effect, elems, else, elseif, end,
     error, errs, exists, exists1, exit, ext, floor, for, forall, from, functions, 
     general, hd, if, in, inds, init, inmap, input, instance, int, inter, inv, inverse, iota, is, 
     isofbaseclass, isofclass, inv, inverse, lambda, let, map, mu, mutex, mod, nat, nat1, new, merge, 
     munion, not, of, operations, or, others, per, periodic, post, power, pre, pref, 
     private, protected, public, qsync, rd, responsibility, return, reverse, samebaseclass, 
     sameclass, psubset, rem, rng, sel, self, seq, seq1, set, skip, specified, st, 
     start, startlist, static, subclass, subset, subtrace, sync, synonym, then, thread, 
     threadid, time, tixe, tl, to, token, traces, trap, types, undefined, union, using, values, 
     variables, while, with, wr, yet, RESULT, false, true, nil, periodic pref, rat, real},
   %keywordsprefix=mk\_,
   %keywordsprefix=a\_,
   %keywordsprefix=t\_,
   %keywordsprefix=w\_,
   sensitive,
   morecomment=[l]--,
   morestring=[b]",
   morestring=[b]',
  }[keywords,comments,strings]
\lstdefinelanguage{JavaCC}
  {morekeywords={options, PARSER\_BEGIN, PARSER\_END, SKIP, TOKEN},
   sensitive=false,
  }[keywords]

% define the layout for listings
\lstdefinestyle{tool}{basicstyle=\ttfamily,
                         frame=trBL, 
			 showstringspaces=false, 
			 frameround=ffff, 
			 framexleftmargin=0mm, 
			 framexrightmargin=0mm}
\lstdefinestyle{mystyle}{basicstyle=\ttfamily,
                         frame=trBL, 
%                         numbers=left, 
%			 gobble=0, 
			 showstringspaces=false, 
%			 linewidth=\textwidth, 
			 frameround=fttt, 
			 aboveskip=5mm,
			 belowskip=5mm,
			 framexleftmargin=0mm, 
			 framexrightmargin=0mm}


% Slightly nicer XML printing than the default.
\lstdefinelanguage{DCXML}
{
	morecomment=[n][keywordstyle]{<}{\ },
	morecomment=[n][keywordstyle]{<}{\\},
	morecomment=[n][keywordstyle]{<}{>},
	morecomment=[n][keywordstyle]{<}{\	},	
	morestring=[b]',
	morestring=[b]"
}

% Some nice defaults for listings
\lstset{ 
	basicstyle=\tiny, 
%	identifierstyle=\textit, 
	breaklines=true, 
	frame=shadowbox, 
	float=htbp, 
	xleftmargin=5pt, 
	xrightmargin=5pt,
	keywordstyle=\color[rgb]{0,0,1},
   commentstyle=\color[rgb]{0.133,0.545,0.133},
   stringstyle=\color[rgb]{0.627,0.126,0.941},
}




\newcommand{\kw}[1]{{\tt #1}}

%%%%%%%%%%%%%%%%%% Commands for bibtex %%%%%%%%%%%%%%%%%%%%%%%
%************************************************************************
%                                                                       *
%       Bibliography and Terminology supporting commands                *
%                                                                       *
%************************************************************************

\newcommand{\bthisbibliography}[1]{\chapter*{References}%
   \begin {list} {}%
     {\settowidth {\labelwidth} {[#1]XX}%
      \setlength {\leftmargin} {\labelwidth}%
      \addtolength{\leftmargin} {\labelsep}%
      \setlength {\parsep} {1ex}%
      \setlength {\itemsep} {2ex}%
     }
  }
\newcommand{\ethisbibliography}{\end{list}}
\newcommand{\refitem}[2]
  {\bibitem[#1]{#2}}

% Requirements environment
\newenvironment{reqs}{%
\begin{enumerate}
%\renewcommand{\labelenumi}{\textbf{R\theenumi}}
\renewcommand{\theenumi}{\textbf{R\arabic{enumi}}}
}{%
\end{enumerate}}

%\newcommand{\Exercise}[1]{%
%    \textbf{Exercise \thechapter.\theexerciseno}
%   \refstepcounter{exerciseno} #1 $\Box$\\ }%}
%\newcommand{\initexercise}{\setcounter{exerciseno}{0}}

\newcommand{\experience}[1]{%
\begin{center}
\fbox{
\begin{minipage}[t]{.8\textwidth}
#1
\end{minipage}}
\end{center}}

\newcommand{\markDone}[1]{{\color{Gray}#1}}
\newcommand{\developer}[1]{{\scriptsize \fbox{#1}}}


\usepackage{fancyhdr}

\pagestyle{fancy}
\fancyhead{}
\fancyhead[LO]{\leftmark}
\fancyhead[RE]{Restructuring of AST}
\fancyhead[RO,LE]{\resizebox{0.05\textwidth}{!}{\includegraphics{logo.jpg}}}
\fancyfoot[C]{\thepage}


%%%%%%%%%%%%%%%%%%%%%% Commands associated with development docs
\newcommand{\epoint}[1]{{\tt #1}}
\newcommand{\class}[1]{\textit{#1}}
\newcommand{\java}[1]{{\tt #1}}
\newcommand{\lstjava}[1]{
	\lstset
	{
		language=Java,
		basicstyle=\ttfamily\footnotesize,
		captionpos=b,
		caption=#1
	}
}


\begin{document}
 
\title{Restructuring of AST in Overture components: Overture-II}
\author{Kenneth Lausdahl\\
        Augusto Ribeiro}

\date{\today}

%\frontmatter
\reportno{TR-2010-04????}     

\pagenumbering{roman}
\maketitle
%\addtocounter{page}{2}
\tableofcontents
\newpage


\section*{Document History}
\begin{center}
\begin{tabular}{|c|c|c|}
\hline
\textbf{Revision} & \textbf{Notes} & \textbf{Date} \\ \hline
1 & Initial proposed version& 5.2011 \\ \hline
\end{tabular}
\end{center}


\pagenumbering{arabic} 
\setcounter{page}{1}

\chapter{Introduction}
Blah

\section{Goals}

Our goals are to have:
\begin{enumerate}
\item A single VDM AST shared by all Overture development. e.g. Parser, Type Checker, interpreter, IDE and other analysis.
\item An AST which only holds the semantic VDM abstract syntax thus no code for analysis.
\item An AST which is well structured and supports analysis through pre defined tree visitors which should visit the complete tree in a pre defined way.
\begin{enumerate}
\item Depth first ..
\item ...
\end{enumerate}
\item An AST which can be used both in VDM models for interpretation, pure Java code and code generated VDM models.

\item \textbf{The AST must support reuse of VDMJ both: Parser, Type Checker and interpreter. This means that the general way VDMJ works must be taken into account but not dictate the structure but a way to reshape the current code to fit the new AST must exist. ``This excludes complete redevelopment''.}
\item The ability to develop new plug-ins (including ones not developed by Overture) which can do complete analysis of the AST without requiring any changes to the AST.
\item The ability to either share an instance of the Overture AST among plugins or provide a clone working copy which a plug-in can manipulate.
\end{enumerate}

\section{Plan}

\begin{enumerate}

\item Decide which features the AST should have. Which methods do we require by \texttt{Node} to enable the different kinds of analysis we want. And what is needed to hold the kind of information we require.
\item Create a tool which generates such an AST for Java.
\item Define the structure of the Overture AST in the format required by the new tool.
\item Split up the tasks of replacing the old AST with the new one:
\begin{enumerate}
\item Parser
\item Type Checker
\item Interpreter
\item All IDE plug-ins
\end{enumerate}
\item During the development of the new projects where the new AST is used test cases must be created. This means for e.g. the parser that one positive and one negative test must be written for each expression implemented.

\end{enumerate}



\section{Analysis support}
A number of questions has been raised about how the AST should be structured: In particular the need for a visitor of the AST and the dislike of visitors in general.
A few of the needed features are:
\begin{itemize}
\item Link between: Parent $\leftrightarrow$ Child
\item Search method; allowing any parent of a type X to be found to be found for a node
\item Visitors
\end{itemize}

The next sections will try to address the problem with integration of the current type checker and interpreter with the AST which is implemented inside the current AST.


\subsection{Visitor}
A strait forward visitor would then have a visit method for all types of nodes like shown in listing~\ref{} and there is no return value.
\lstset{tabsize=2,frame=single}

\begin{lstlisting}[language=java]
public abstract class Node
{
	/**
	 * Applies this node to the {@link Analysis} visitor {@code analysis}.
	 * @param analysis the {@link Analysis} to which this node is applied
	 */
	public abstract void apply(Analysis analysis);
}
\end{lstlisting}

\begin{lstlisting}[language=java]
public class AnalysisAdaptor implements Analysis
{
	/**
	* Called by the {@link ABinopExp} node from {@link ABinopExp#apply(Switch)}.
	* @param node the calling {@link ABinopExp} node
	*/
	public void caseABinopExp(ABinopExp node)
	{

	}
}
\end{lstlisting}

Invocation
\begin{lstlisting}[language=java]
node.accept(new AnalysisAdaptor());
\end{lstlisting}

\subsection{Supporting the VDMJ Type Checker through a \texttt{typeCheck} method}
This idea have previously been discussed over mail. It was discussed how the code from VDMJ which builds on the fact that each node knows how to type check it self and that the general feeling was that visitors is difficult to use.
\begin{enumerate}
\item Use a visitor to visit all nodes through a generated visit structure
\item Implement a method in each node which delegates the method to an instance outside the node.
\end{enumerate}

\subsection{Delegation through explicit method invocation}
To allow a node to behaviour like in the current AST a method could be added to the node itself to enable type check and eval. Listing~\ref{} shows this for the base class \texttt{Node}. It is important to understand that this approach requires the AST to be updated if a new analysis should be done also the type argument can not be made type safe do to the lack of knowledge of the type between \texttt{Node} and the return type of the concrete analysis.

\textit{This was generally the preferred idea. However see section~\ref{} for a more general approach.}

\begin{lstlisting}[language=java]
public abstract class Node
{
	public <Typ extends Node> Typ typeCheck(TypeChecker tc, Environment env, NameScope scope, TypeList qualifiers)
	{
		return null;
	}

	public IValue eval(Eval evaluator, Context ctxt)
	{
		return null;
	}
}
\end{lstlisting}

And \texttt{ABinopExp} would then look like in listing~\ref{} where \texttt{TypeChecker} and \texttt{Evaluator} is two container classes for all type checkers: expression, pattern etc. And the same goes for eval.
\begin{lstlisting}[language=java]
public class ABinopExp extends PExp
{
	@Override
	public <Typ extends Node> Typ typeCheck(TypeChecker tc, Environment env, NameScope scope, TypeList  qualifiers)
	{
		return tc.getPExp().caseABinopExp(this, env, scope, qualifiers);
	}


	@Override
	public IValue eval(Eval evaluator, Context ctxt)
	{
		return evaluator.getPExp().caseABinopExp(this, ctxt);
	}
}
\end{lstlisting}

The idea presented in listing~\ref{} would then for the type checker be implemented as shown below in listing~\ref{} for expressions. It looks very similar to whats inside VDMJ and it is almost a direct copy/paste job. The same applies to eval.

\begin{lstlisting}[language=java]
public static class ExpressionTc extends PExpTypeChecker
	{
	@Override
	public <Typ extends Node> Typ caseABinopExp(ABinopExp source,
			Environment env, NameScope scope, TypeList qualifiers)
	{
		PType expected = null;
		
		if(source.getBinop() instanceof APlusBinop || source.getBinop() instanceof AMinusBinop)
		{
			expected = new AIntType();
		}else if( source.getBinop() instanceof ALazyAndBinop ||source.getBinop() instanceof ALazyOrBinop)
		{
			expected = new ABoolType();
		}
		
		Node ltype = source.getLeft().typeCheck(this.parent(), env, scope, null);
		Node rtype = source.getRight().typeCheck(this.parent(), env, scope, null);

		if (!expected.getClass().isInstance(ltype))
		{
			report(3065, "Left hand of " + source.getBinop() + " is not "
					+ expected);
		}

		if (!expected.getClass().isInstance(rtype))
		{
			report(3066, "Right hand of " + source.getBinop() + " is not "
					+ expected);
		}
		source.setType((PType) expected);
		return (Typ) expected;
	}
}
\end{lstlisting}

\begin{lstlisting}[language=java]
PExp exp = new ABinopExp(new AIntConstExp(new TNumbersLiteral("2")), 
new APlusBinop(new TPlus()), 
new AIntConstExp(new TNumbersLiteral("5")));
//Type Check we get the type back
PType n = exp.typeCheck(typeChecker, new Environment(), new NameScope(), null);
//Eval we get a value back
IValue n = exp.eval(new Eval(new PBinopEval(), new PUnopEval(), new CustomPExpEval(), new PBooleanEval(), new PTypeEval()), null);
\end{lstlisting}

\subsubsection{Generalization}
If we take a closer look at the idea behind typecheck and eval from the node class we can see a pattern. This is just a visitor pattern wrapped as an QuestionAnswer visitor, so we could do this generally like:
\begin{lstlisting}[language=java]
public abstract class Node
{
	/**
	 * Returns the answer for {@code answer} by applying this node with the
	 * {@code question} to the {@link QuestionAnswer} visitor.
	 * @param caller the {@link QuestionAnswer} to which this node is applied
	 * @param question the question provided to {@code answer}
	 * @return the answer as returned from {@code answer}
	 */
	public abstract <Q,A> A apply(QuestionAnswer<Q,A> caller, Q question);
}
\end{lstlisting}

\begin{lstlisting}[language=java]
public class ABinopExp extends PExp
{
	/**
	* Calls the {@link QuestionAnswer<Q, A>#caseABinopExp(ABinopExp)} of the {@link QuestionAnswer<Q, A>} {@code caller}.
	* @param caller the {@link QuestionAnswer<Q, A>} to which this {@link ABinopExp} node is applied
	* @param question the question provided to {@code caller}
	*/
	@Override
	public <Q, A> A apply(QuestionAnswer<Q, A> caller, Q question)
	{
		return caller.caseABinopExp(this, question);
	}
}
\end{lstlisting}
This approach allows us to do any analysis in a type safe way like:

And we define our type checker and evaluator as:
\begin{lstlisting}[language=java]
public class TypeCheckInfo
{
	public Environment env;
	public NameScope scope;
	public TypeList qualifiers;
}

public class TypeCheckVisitor extends QuestionAnswerAdaptor<TypeCheckInfo, PType>
{
	@Override
	public PType caseABinopExp(ABinopExp node, TypeCheckInfo question)
	{
		PType expected = null;

		if (node.getBinop() instanceof APlusBinop
				|| node.getBinop() instanceof AMinusBinop)
		{
			expected = new AIntType();
		} else if (node.getBinop() instanceof ALazyAndBinop
				|| node.getBinop() instanceof ALazyOrBinop)
		{
			expected = new ABoolType();
		}

		Node ltype = node.getLeft().apply(this,question);
		Node rtype = node.getRight().apply(this,question);

		if (!expected.getClass().isInstance(ltype))
		{
			report(3065, "Left hand of " + node.getBinop() + " is not "
					+ expected);
		}

		if (!expected.getClass().isInstance(rtype))
		{
			report(3066, "Right hand of " + node.getBinop() + " is not "
					+ expected);
		}
		node.setType((PType) expected);
		return expected;
	}
}
\end{lstlisting}

and invoke this like

\begin{lstlisting}[language=java]
PType t = exp.apply(new TypeCheckVisitor(), new TypeCheckInfo());
\end{lstlisting}

This approach provides enables both a type checker and interpreter to be implemented and other things to be implemented through the same interface on the AST nodes.


\section{AST structure}
To be written:

\lstset{morekeywords={Tokens,Abstract,Syntax,Tree,Aspect,Declaration},morecomment=[l]{//}}
\begin{lstlisting}
Tokens

  plus = '+';
  int = 'int';
  real = 'real';
  bool = 'bool';
  true = 'true';
  false = 'false';
  and_and = '&&';
  or_or = '||';
  numbers_literal = 'some regex for numbers';

Abstract Syntax Tree

binop
    =  {plus} [token]:plus
    |   {minus} [token]:minus
    |   {lazy_and} [token]:and_and
    |   {lazy_or} [token]:or_or
    ;
    
unop
    =   {negate} [token]:minus
    ;
    
exp
    =  {binop} [left]:exp binop [right]:exp
    |   {unop} unop exp
    |   {int_const} numbers_literal
    |   {boolean_const} boolean
    |   {apply} [root]:exp [args]:exp*
    ;
    
boolean
    =  {true}
    |   {false}
    ;
    
type
    =  {real} [token]:real
    |   {int} [token]:int
    |   {bool} [token]:bool
    ;

Aspect Declaration

exp 
    = [type]:type
    ;
\end{lstlisting}

\bibliographystyle{acm}

\bibliography{bib/bibliography}

\end{document}
