\documentclass{overturerep}
\usepackage{url}
\usepackage{graphics}
\usepackage{times}
\usepackage{listings}
\usepackage{color}
\usepackage{graphicx}
\usepackage{latexsym}
\usepackage{longtable} % ,multirow}

\newcommand{\vdmtools}{VDMTools}
\newcommand{\vdmstyle}[1]{\texttt{#1}}
\newcounter{exerciseno}

%\newcommand{\Exercise}[1]{%
%    \textbf{Exercise \thechapter.\theexerciseno}
%   \refstepcounter{exerciseno} #1 $\Box$\\ }%}
%\newcommand{\initexercise}{\setcounter{exerciseno}{0}}
%\newcounter{exerciseno}

\newcommand\thebookexercise{\thechapter.\arabic{exerciseno}}
\newenvironment{myexercise}{\par
  \refstepcounter{exerciseno}%
  \indent\textbf{Exercise\ \thebookexercise}\enskip}{$\Box$\\
}
\newenvironment{myhardexercise}{\par
  \refstepcounter{exerciseno}%
  \indent\textbf{Exercise\ \thebookexercise $\star$}\enskip}{$\Box$\\
}
\newcommand{\initexercise}{\setcounter{exerciseno}{0}}
\newenvironment{mysolution}{
} %% This will be replaced by a perl script extracting the solutions
  %% and inserting them automatically into the solutions chapter.
 
%\newcommand{\insertcommentedvdm*}[2]{}
  %% This macro is identified by perl script which will move the
  %% parameter to the solutions chapter.
% definition of VDM++, JavaCC, JJTree, JTB, ANTLR and SableCC for listings
\lstdefinelanguage{VDM++}
  {morekeywords={\#act, \#active, \#fin, \#req, \#waiting, abs, all, allsuper, always, and, answer, 
     assumption, async, atomic, be, bool, by, card, cases, char, class, comp, compose, conc, cycles,
     dcl, def, del, dinter, div, do, dom, dunion, duration, effect, elems, else, elseif, end,
     error, errs, exists, exists1, exit, ext, floor, for, forall, from, functions, 
     general, hd, if, in, inds, init, inmap, input, instance, int, inter, inv, inverse, iota, is, 
     isofbaseclass, isofclass, inv, inverse, lambda, let, map, mu, mutex, mod, nat, nat1, new, merge, 
     munion, not, of, operations, or, others, per, periodic, post, power, pre, pref, 
     private, protected, public, qsync, rd, responsibility, return, reverse, samebaseclass, 
     sameclass, psubset, rem, rng, sel, self, seq, seq1, set, skip, specified, st, 
     start, startlist, static, subclass, subset, subtrace, sync, synonym, then, thread, 
     threadid, time, tixe, tl, to, token, traces, trap, types, undefined, union, using, values, 
     variables, while, with, wr, yet, RESULT, false, true, nil, periodic pref, rat, real},
   %keywordsprefix=mk\_,
   %keywordsprefix=a\_,
   %keywordsprefix=t\_,
   %keywordsprefix=w\_,
   sensitive,
   morecomment=[l]--,
   morestring=[b]",
   morestring=[b]',
  }[keywords,comments,strings]
\lstdefinelanguage{JavaCC}
  {morekeywords={options, PARSER\_BEGIN, PARSER\_END, SKIP, TOKEN},
   sensitive=false,
  }[keywords]

% define the layout for listings
\lstdefinestyle{tool}{basicstyle=\ttfamily,
                         frame=trBL, 
			 showstringspaces=false, 
			 frameround=ffff, 
			 framexleftmargin=0mm, 
			 framexrightmargin=0mm}
\lstdefinestyle{mystyle}{basicstyle=\ttfamily,
                         frame=trBL, 
%                         numbers=left, 
%			 gobble=0, 
			 showstringspaces=false, 
%			 linewidth=\textwidth, 
			 frameround=fttt, 
			 aboveskip=5mm,
			 belowskip=5mm,
			 framexleftmargin=0mm, 
			 framexrightmargin=0mm}
%\lstdefinestyle{mystyle}{basicstyle=\sffamily\small,
%			 frame=tb,
%                         numbers=left,
%			 gobble=0,
%			 showstringspaces=false,
%			 linewidth=345pt,
%			 frameround=ffff,
%			 framexleftmargin=8mm,
%			 framexrightmargin=8mm,
%			 framextopmargin=1mm,
%			 framexbottommargin=1mm,
%			 aboveskip=7mm,
%			 belowskip=5mm,
%			 xleftmargin=10mm,}

\lstset{style=mystyle}
\lstset{language=VDM++}
\lstset{alsolanguage=Java}
% The command below enables you to escape into normal LaTeX mode inside your 
% VDM chunks by starting with a `?�� character and ending with a `����
%\lstset{escapeinside=?��}

\newcommand{\kw}[1]{{\tt #1}}

%%%%%%%%%%%%%%%%%% Commands for bibtex %%%%%%%%%%%%%%%%%%%%%%%
%************************************************************************
%                                                                       *
%       Bibliography and Terminology supporting commands                *
%                                                                       *
%************************************************************************

\newcommand{\bthisbibliography}[1]{\chapter*{References}%
   \begin {list} {}%
     {\settowidth {\labelwidth} {[#1]XX}%
      \setlength {\leftmargin} {\labelwidth}%
      \addtolength{\leftmargin} {\labelsep}%
      \setlength {\parsep} {1ex}%
      \setlength {\itemsep} {2ex}%
     }
  }
\newcommand{\ethisbibliography}{\end{list}}
\newcommand{\refitem}[2]
  {\bibitem[#1]{#2}}

% Requirements environment
\newenvironment{reqs}{%
\begin{enumerate}
%\renewcommand{\labelenumi}{\textbf{R\theenumi}}
\renewcommand{\theenumi}{\textbf{R\arabic{enumi}}}
}{%
\end{enumerate}}

%\newcommand{\Exercise}[1]{%
%    \textbf{Exercise \thechapter.\theexerciseno}
%   \refstepcounter{exerciseno} #1 $\Box$\\ }%}
%\newcommand{\initexercise}{\setcounter{exerciseno}{0}}

\newcommand{\experience}[1]{%
\begin{center}
\fbox{
\begin{minipage}[t]{.8\textwidth}
#1
\end{minipage}}
\end{center}}

\usepackage{fancyhdr}

\pagestyle{fancy}
\fancyhead{}
\fancyhead[LO]{\leftmark}
\fancyhead[RE]{Tutorial to Overture/VDM-RT}
\fancyhead[RO,LE]{\resizebox{0.05\textwidth}{!}{\includegraphics{OMLlogoattempt.jpg}}}
\fancyfoot[C]{\thepage}

\begin{document}
 
\title{2010 Release Planning for Overture and DESTECS}
\author{Peter Gorm Larsen\\
        %\and
        Kenneth Lausdahl\\
        Augusto Ribeiro}

\date{February 2010}

%\frontmatter
\reportno{TR-2010-04}     

\pagenumbering{roman}
\maketitle
%\addtocounter{page}{2}
\tableofcontents
\newpage
% \include{foreword/foreword}

%\cleardoublepage
%\mainmatter
%\lhead{\nouppercase{\rightmark}}
%\rhead{\nouppercase{\leftmark}}


%\pagestyle{fancy}
%\fancyhead{}
%\fancyhead[LO]{\leftmark}
%\fancyhead[RE]{Validated Designs for Object-oriented Systems}
%\fancyhead[RO,LE]{\thepage}
%\fancyfoot{}
%%\fancyfoot[LE,RO]{\thepage}

\pagenumbering{arabic} 
\setcounter{page}{1}
\addtocounter{chapter}{2}

\section{Introduction}

This document is intended to provide an overview of the release
planning for the open source project called Overture as well as the
European research project DESTECS. Efforts will be made to ensure that
stable releases of both the executables for Overture and DESTECS in a
systematic fashion. It is recommended that official public releases of
these executables are synchronised with each other. 

Official major public releases will be made yearly and will be made
available through the Overture and DESTECS official websites
respectively.  One month before the major release, a Release Candidate
will be made for the purpose of testing and locate bugs. In the
meantime, snapshot builds will be provided via SourceForge. In this
period only bug fixes needed for the features planning in that release
will be checked in. Minor official public releases will be released
approximately once every three months and for each of these two weeks
of release candidate periods will be used. A full release procedure
will be followed for each of the public releases such that the users
can expect all public releases to be stable.  In between these public
releases ad-hoc releases will be produced on a needs basis but it is
suggested that at least a weekly build that contains the latest source
code will be put in system. For such ad-hoc releases the
documenatation cannot be guranteed to be updated and it will not be
tested at the same level of detail as for official releases so no
gurantee is provided for all ad-hoc releases. All the builds will
consist of stand alone branded version for the Windows, Mac and Linux
platforms.

This document start off with providing an overview of the release
numbering scheme in Section~\ref{sec:numbering}. This is followed by
Section~\ref{sec:Overture2010} which provides the current plan for the
releases of Overture executables during 2010. Finally
Section~\ref{sec:DESTECS2010} provides the current plan for the
additional features released in the DESTECS executable releases in
2010. Later on the document will be updated with plans for subsequent
years. 

\section{Release Build Numbering}\label{sec:numbering}

The Overture and DESTECS executable build numbering will be made with
the following format:
\begin{center}
\fbox{\textbf{X.Y.Z}}
\end{center}

The \textbf{X} indicates which is the major public release version of
the tool. During the DESTECS project there will be three major
releases of the DESTECS Tool (month 12, 24 and 36, i.e.\ December
2010, 2011 and 2012 respectively). It is suggested that these builds
are numbered 1.0.0, 2.0.0 and 3.0.0 respectively.

An increase of \textbf{Y} indicates a minor public release including
significant update since last version, for example, a new feature
which have been through a proper release procedure. Thus the users can
expect it to be in a stable state.

An increase of  \textbf{Z} indicates an ad-hoc release with minor
update since the last version, for example, bug fixes. No gurantee for
stability is provided for these ad-hoc releases because they have not
been going through the release procedure used for public releases.

\section{Contents of Overture releases in 2010}\label{sec:Overture2010}

This section presents a plan for the contents of the 2010 releases of
the Overture executable.

\begin{description}
\item[Release 0.2.0 (March 2010):] The aim for this release is to
  include the following features:  
\begin{itemize}
\item Complete the new ASTGen utility and add all sources for it to
  the tools utility in the Overture SF SVN. \fbox{NickBattle}
\item Change to the AST generated from the new version of ASTGen all
  over the Overture sources. \fbox{NickBattle,KennethLausdahl}
\item The combinatorial testing feature will be updated in the IDE
  with the features for reducing the size of the test suites in
  intelligent ways made available in VDMJ for all VDM
  dialects. \fbox{KennethLausdahl}
\item In the edit perspective automatic type checking will only be
  performed in case the syntax is correct for all definitions such
  that type errors does not get mixed with syntax errors. \fbox{??}
\item When the Overture tool is first started up and the user kills
  the welcome window the VDM++ edit perspective will be started
  automatically. \fbox{??}
\item When creating a new debug configuration it shall be named after
  the name of the project (and not just ``New
  configuration''). \fbox{??} 
\item Making sure that the run and debug buttons in the IDE work
  appropriately and does not crash whenever the user tries to start
  them. Ideally these should simply change to the debug perspective if
  no debug configuration is present and be ready to type an expression
  into a quick console where simple expressions can be typed and
  executed (and up and down in the list of commands given is
  supported). \fbox{??} 
\item Syntax and type check the arguments used in the debug
  configuration. \fbox{??}
\item Remove all communication exceptions from VDMJ which causes the
  console to change. \fbox{NickBattle}
\item Solve the existing problem with the debugger in the IDE whenever
  an invariant for instance invariants are
  broken. \fbox{KennethLaudahl,NickBattle} 
\item Documentation of the Overture IDE in order to enable more
  developers to make updates for the IDE and for the DESTECS
  development to follow similar principles. This shall document the
  implementation of the editor, the parser, the builder, the outline
  and the AST access. \fbox{AugustoRibeiro}
\item Specify the top level scheduling strategy for the interpreter
  dealing with multiple threads in VDM++ and VDM-RT (to subsequently
  be implemented in the interpreter).
  \fbox{KennethLausdahl,++} 
\item Make the pretty printer with test coverage usable for all VDM
  dialects. \fbox{KennethLausdahl}
\end{itemize}

\item[Release 0.3.0 (June 2010):] The aim for this release is to
  include the following features:  
\begin{itemize}
\item A button for moving a test case from the combinatorial testing
  perspective into the debugger such that it will stop at the right
  place for the run-time error reported for the test case
  selected. \fbox{KennethLausdahl} 
\item Stop all threads when a breakpoint or a run-time error has
  occured in one of the threads. \fbox{NickBattle,++} 
\item In the debug perspective it shall be possible to look into the
  contents of objects in the variables view. \fbox{??}
\item Include all definitions in the Outline view for all VDM
  dialects. \fbox{AugustoRibeiro}
\item When creating new projects in the IDE include the possibility
  for including standard libraries (i.e.\ IO and Math) in the
  project. These are then to be included in a special ``lib''
  directory and proof obligations for these are ignorred. \fbox{??}
\item Add DL classes and modules to Overture where the user shall
  place corresponding jar files in the lib directory of the
  project. \fbox{KennethLaudahl,NickBattle} 
\end{itemize}

\item[Release 0.4.0 (September 2010):] The aim for this release is to
  include the following features:   
\begin{itemize}
\item First stable bi-directional connection from Overture/VDM++
  to/from JML. \fbox{CarlosVilhena}
\item New version of the interpreter for scheduling of multiple
  threads in VDM++ and VDM-RT prepared for co-simulkation also able to
  break after a fixed amount of time. \fbox{KennthLausdahl,++}
\item User documentation in pdf available as a part of the Overture
  executable so the user can search on-line in these sources
  (including a VDM-10 language manual). \fbox{PeterGormLarsen,++}
\end{itemize}

\item[Release 1.0.0 (December 2010):] The aim for this release is to
  include the following features:   
\begin{itemize}
\item Quick fix incorporated in the editor view of the IDE for all VDM
  dialects. \fbox{??}
\item Goto definition in the edit view of the IDE across all files in
  a project for all VDM dialects. \fbox{??}
\item First stable release of proof support feature for proving a
  subset of the POs generated. \fbox{NickBattle}
\item Support in the Proof Obligation Explorer view for new icons
  indicating whether the user have proved a PO or have manually
  inspected it and approved it. This also means that the icon with the
  red cross shall only be used for those where proof have failed and a
  less intemidating iconj shall be used for the POs that one have not
  yet dealt with. \fbox{??}
\item First version of refactoring support for all VDM
  dialects. \fbox{??}
\item Include VDMDoc (inspired by JavaDoc) for all VDM dialects. \fbox{??}
\item Stable version of the UML mapper enabling proper round-trip
  engineering between VDM and UML class diagrams and enable generation
  of traces from a subset of UML sequence
  diagrams. \fbox{KennethLausdahl,PeterGormLarsen} 
\item First release of code generator from VDM to a programming
  language. \fbox{MarcelVerhoef}
\item Increased \LaTeX\ support in the editor such that parts outside
  the \texttt{vdm\_al} environments are highlighed as TeXClipse would
  do it. \fbox{??}
\end{itemize}
\end{description}

\section{Contents of DESTECS releases in 2010}\label{sec:DESTECS2010}

This section presents a plan for the contents of the 2010 releases of
the DESTECS executable.

\begin{description}
\item[Release 0.1.0 (March 2010):] The aim for this release is to
  include the following features:  
\begin{itemize}
\item Stand-alone DESTECS executable with the DESTECS icon
  \fbox{KennethLausdahl} 
\item Develop wizard for creating a new DESTECS project. This shall
  create 3 different directories in the project for ``Discrete
  Event'', ``Continuous Time'' and ``Fault Modelling''
  respectively. \fbox{??} 
\item Ability to invoke the 20-sim editor for 20-sim models in a
  DESTECS project. \fbox{ControlLabs}
\end{itemize}

\item[Release 0.2.0 (June 2010):] The aim for this release is to
  include the following features:  
\begin{itemize}
\item TBD
\end{itemize}

\item[Release 0.3.0 (September 2010):] The aim for this release is to
  include the following features:   
\begin{itemize}
\item TBD
\end{itemize}

\item[Release 1.0.0 (December 2010):] The aim for this release is to
  include the following features:   
\begin{itemize}
\item Co-simulation between Overture VDM models and 20-sim models is
  fully enabled. \fbox{KennethLausdahl,++}
\item Synchronising the log viewer from Overture with the plot of
  variables from 20-sim in an Eclipse setting. \fbox{ControlLabs,++}
\end{itemize}
\end{description}


\bibliographystyle{newalpha}

\bibliography{dan}

\end{document}
