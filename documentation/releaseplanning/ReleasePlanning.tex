\documentclass{overturerep}
\usepackage{url}
\usepackage{graphics}
\usepackage{times}
\usepackage{listings}
%\usepackage{color}
\usepackage[usenames,dvipsnames]{color}
\usepackage{graphicx}
\usepackage{latexsym}
\usepackage{longtable} % ,multirow}

\newcommand{\vdmtools}{VDMTools}
\newcommand{\vdmstyle}[1]{\texttt{#1}}
\newcounter{exerciseno}

%\newcommand{\Exercise}[1]{%
%    \textbf{Exercise \thechapter.\theexerciseno}
%   \refstepcounter{exerciseno} #1 $\Box$\\ }%}
%\newcommand{\initexercise}{\setcounter{exerciseno}{0}}
%\newcounter{exerciseno}

\newcommand\thebookexercise{\thechapter.\arabic{exerciseno}}
\newenvironment{myexercise}{\par
  \refstepcounter{exerciseno}%
  \indent\textbf{Exercise\ \thebookexercise}\enskip}{$\Box$\\
}
\newenvironment{myhardexercise}{\par
  \refstepcounter{exerciseno}%
  \indent\textbf{Exercise\ \thebookexercise $\star$}\enskip}{$\Box$\\
}
\newcommand{\initexercise}{\setcounter{exerciseno}{0}}
\newenvironment{mysolution}{
} %% This will be replaced by a perl script extracting the solutions
  %% and inserting them automatically into the solutions chapter.
 
%\newcommand{\insertcommentedvdm*}[2]{}
  %% This macro is identified by perl script which will move the
  %% parameter to the solutions chapter.
% definition of VDM++, JavaCC, JJTree, JTB, ANTLR and SableCC for listings
\lstdefinelanguage{VDM++}
  {morekeywords={\#act, \#active, \#fin, \#req, \#waiting, abs, all, allsuper, always, and, answer, 
     assumption, async, atomic, be, bool, by, card, cases, char, class, comp, compose, conc, cycles,
     dcl, def, del, dinter, div, do, dom, dunion, duration, effect, elems, else, elseif, end,
     error, errs, exists, exists1, exit, ext, floor, for, forall, from, functions, 
     general, hd, if, in, inds, init, inmap, input, instance, int, inter, inv, inverse, iota, is, 
     isofbaseclass, isofclass, inv, inverse, lambda, let, map, mu, mutex, mod, nat, nat1, new, merge, 
     munion, not, of, operations, or, others, per, periodic, post, power, pre, pref, 
     private, protected, public, qsync, rd, responsibility, return, reverse, samebaseclass, 
     sameclass, psubset, rem, rng, sel, self, seq, seq1, set, skip, specified, st, 
     start, startlist, static, subclass, subset, subtrace, sync, synonym, then, thread, 
     threadid, time, tixe, tl, to, token, traces, trap, types, undefined, union, using, values, 
     variables, while, with, wr, yet, RESULT, false, true, nil, periodic pref, rat, real},
   %keywordsprefix=mk\_,
   %keywordsprefix=a\_,
   %keywordsprefix=t\_,
   %keywordsprefix=w\_,
   sensitive,
   morecomment=[l]--,
   morestring=[b]",
   morestring=[b]',
  }[keywords,comments,strings]
\lstdefinelanguage{JavaCC}
  {morekeywords={options, PARSER\_BEGIN, PARSER\_END, SKIP, TOKEN},
   sensitive=false,
  }[keywords]

% define the layout for listings
\lstdefinestyle{tool}{basicstyle=\ttfamily,
                         frame=trBL, 
			 showstringspaces=false, 
			 frameround=ffff, 
			 framexleftmargin=0mm, 
			 framexrightmargin=0mm}
\lstdefinestyle{mystyle}{basicstyle=\ttfamily,
                         frame=trBL, 
%                         numbers=left, 
%			 gobble=0, 
			 showstringspaces=false, 
%			 linewidth=\textwidth, 
			 frameround=fttt, 
			 aboveskip=5mm,
			 belowskip=5mm,
			 framexleftmargin=0mm, 
			 framexrightmargin=0mm}
%\lstdefinestyle{mystyle}{basicstyle=\sffamily\small,
%			 frame=tb,
%                         numbers=left,
%			 gobble=0,
%			 showstringspaces=false,
%			 linewidth=345pt,
%			 frameround=ffff,
%			 framexleftmargin=8mm,
%			 framexrightmargin=8mm,
%			 framextopmargin=1mm,
%			 framexbottommargin=1mm,
%			 aboveskip=7mm,
%			 belowskip=5mm,
%			 xleftmargin=10mm,}

\lstset{style=mystyle}
\lstset{language=VDM++}
\lstset{alsolanguage=Java}
% The command below enables you to escape into normal LaTeX mode inside your 
% VDM chunks by starting with a `?�� character and ending with a `����
%\lstset{escapeinside=?��}

\newcommand{\kw}[1]{{\tt #1}}

%%%%%%%%%%%%%%%%%% Commands for bibtex %%%%%%%%%%%%%%%%%%%%%%%
%************************************************************************
%                                                                       *
%       Bibliography and Terminology supporting commands                *
%                                                                       *
%************************************************************************

\newcommand{\bthisbibliography}[1]{\chapter*{References}%
   \begin {list} {}%
     {\settowidth {\labelwidth} {[#1]XX}%
      \setlength {\leftmargin} {\labelwidth}%
      \addtolength{\leftmargin} {\labelsep}%
      \setlength {\parsep} {1ex}%
      \setlength {\itemsep} {2ex}%
     }
  }
\newcommand{\ethisbibliography}{\end{list}}
\newcommand{\refitem}[2]
  {\bibitem[#1]{#2}}

% Requirements environment
\newenvironment{reqs}{%
\begin{enumerate}
%\renewcommand{\labelenumi}{\textbf{R\theenumi}}
\renewcommand{\theenumi}{\textbf{R\arabic{enumi}}}
}{%
\end{enumerate}}

%\newcommand{\Exercise}[1]{%
%    \textbf{Exercise \thechapter.\theexerciseno}
%   \refstepcounter{exerciseno} #1 $\Box$\\ }%}
%\newcommand{\initexercise}{\setcounter{exerciseno}{0}}

\newcommand{\experience}[1]{%
\begin{center}
\fbox{
\begin{minipage}[t]{.8\textwidth}
#1
\end{minipage}}
\end{center}}

\newcommand{\markDone}[1]{{\color{Gray}#1}}
\newcommand{\developer}[1]{{\scriptsize \fbox{#1}}}


\usepackage{fancyhdr}

\pagestyle{fancy}
\fancyhead{}
\fancyhead[LO]{\leftmark}
\fancyhead[RE]{Release plan Overture/DESTECS}
\fancyhead[RO,LE]{\resizebox{0.05\textwidth}{!}{\includegraphics{OMLlogoattempt.jpg}}}
\fancyfoot[C]{\thepage}

\begin{document}
 
\title{2010 Release Planning for Overture and DESTECS}
\author{Peter Gorm Larsen\\
        %\and
        Kenneth Lausdahl\\
        Augusto Ribeiro}

\date{February 2010}

%\frontmatter
\reportno{TR-2010-04}     

\pagenumbering{roman}
\maketitle
%\addtocounter{page}{2}
\tableofcontents
\newpage
% \include{foreword/foreword}

%\cleardoublepage
%\mainmatter
%\lhead{\nouppercase{\rightmark}}
%\rhead{\nouppercase{\leftmark}}


%\pagestyle{fancy}
%\fancyhead{}
%\fancyhead[LO]{\leftmark}
%\fancyhead[RE]{Validated Designs for Object-oriented Systems}
%\fancyhead[RO,LE]{\thepage}
%\fancyfoot{}
%%\fancyfoot[LE,RO]{\thepage}

\pagenumbering{arabic} 
\setcounter{page}{1}

\chapter{Introduction}

This document is intended to provide an overview of the release
planning for the open source project called Overture as well as the
European research project DESTECS. Efforts will be made to ensure that
stable releases of both the executables for Overture and DESTECS in a
systematic fashion. It is recommended that official public releases of
these executables are synchronised with each other. The planning
presented here is aggressive and probably on the optimistic side but
naturally this also depends upon the time the different stakeholders
are able to deliver to the projects during 2010.

Official major public releases will be made yearly and will be made
available through the Overture and DESTECS official websites
respectively.  One month before the major release, a Release Candidate
will be made for the purpose of testing and locate bugs. In the
meantime, snapshot builds will be provided via SourceForge. In this
period only bug fixes needed for the features planning in that release
will be checked in. Minor official public releases will be released
approximately once every three months and for each of these two weeks
of release candidate periods will be used. A full release procedure
will be followed for each of the public releases such that the users
can expect all public releases to be stable.  In between these public
releases ad-hoc releases will be produced on a needs basis but it is
suggested that at least a weekly build that contains the latest source
code will be put in system. For such ad-hoc releases the
documentation cannot be guaranteed to be updated and it will not be
tested at the same level of detail as for official releases so no
guarantee is provided for all ad-hoc releases. All the builds will
consist of stand alone branded version for the Windows, Mac and Linux
platforms.

This document start off with providing an overview of the release
numbering scheme in Section~\ref{sec:numbering}. This is followed by
Section~\ref{sec:Overture2010} which provides the current plan for the
releases of Overture executables during 2010. Finally
Section~\ref{sec:DESTECS2010} provides the current plan for the
additional features released in the DESTECS executable releases in
2010. Later on the document will be updated with plans for subsequent
years. 

\chapter{Release Build Numbering}\label{sec:numbering}

The Overture and DESTECS project deliverables (both executables as
well as associated documentation) build numbering will be made with
the following format:
\begin{center}
\fbox{\textbf{X.Y.Z}}
\end{center}

The \textbf{X} indicates which is the major public release version of
the tool. During the DESTECS project there will be three major
releases of the DESTECS Tool (month 12, 24 and 36, i.e.\ December
2010, 2011 and 2012 respectively). It is suggested that these builds
are numbered 1.0.0, 2.0.0 and 3.0.0 respectively.

An increase of \textbf{Y} indicates a minor public release including
significant update since last version, for example, a new feature
which have been through a proper release procedure. Thus the users can
expect it to be in a stable state.

An increase of  \textbf{Z} indicates an ad-hoc release with minor
update since the last version, for example, bug fixes. No guarantee for
stability is provided for these ad-hoc releases because they have not
been going through the release procedure used for public releases.

When release candidates are made available at the X and Y level there
shall be a list of checks that needs to be performed by different
named stakeholders before the
release candidate is lifted up to before an official X or Y
release. This check list shall include:

\begin{enumerate}
\item All examples for all dialects must be up to date with the
  software from the release.
\item The tutorials for all dialects must be updated to fit the newest
  screen dumps and the features in the new release.
\item Checks to be performed on the executable at each OS platform
  (Mac, Linux and Windows). The explicit list of what to check for the
  executables must be created.
\end{enumerate}

\chapter{Contents of Overture releases in 2010}\label{sec:Overture2010}

This section presents a plan for the contents of the 2010 releases of
the Overture executable.

\begin{description}
\item[Release 0.2.0 (March 2010):] The aim for this release is to
  include the following features:  
\begin{enumerate}
\item \markDone{Analyse whether we can go for a debugger without
  DLTK. \developer{AugustoRibeiro},\textbf{Dependencies: None}\label{analyseDLTKDebug}}
 
\item Follow standard way to use pictures in Eclipse and document this
  for all developers. \developer{AugustoRibeiro}. \textbf{Dependencies: None}\label{picturesEclipse}



\item \markDone{ The combinatorial testing feature will be updated in the IDE
  with the features for reducing the size of the test suites in
  intelligent ways made available in VDMJ for all VDM
  dialects. \developer{KennethLausdahl}, \textit{Done(Prototype):
    2. feb 2010 - Kenneth}} Unit test, user manual and tutorial update
  still outstanding.

\item \markDone{In the edit perspective automatic type checking will only be
  performed in case the syntax is correct for all definitions such
  that type errors does not get mixed with syntax
  errors. \developer{KennethLausdahl}, \textit{Done: 1. feb 2010 - Kenneth} } \textbf{Remarks:} missing doc

\item \markDone{ When the Overture tool is first started up and the user kills
  the welcome window the VDM++ edit perspective will be started
  automatically. \developer{AugustoRibeiro}, \textit{Done: 1. feb 2010 - Kenneth}} 

\item \markDone{Syntax and type check the arguments used in the debug
  configuration. \developer{KennethLausdahl}}, \textbf{Dependencies: None} \textbf{Remarks:} missing doc

\item \markDone{ Remove all communication exceptions from VDMJ which causes the
  console to change (e.g.\ the existing problem with the debugger in
  the IDE whenever an invariant for instance invariants are
  broken). \developer{NickBattle, KennethLausdahl}, \textbf{Dependencies: None}}

\item Documentation of the Overture IDE in order to enable more
  developers to make updates for the IDE and for the DESTECS
  development to follow similar principles. This shall document the
  implementation of the editor, the parser, the builder, the outline
  and the AST access. \textbf{Status:} Ongoing \developer{AugustoRibeiro}, \textbf{Dependencies: \ref{completeAST}, \ref{analyseDLTKDebug}}

\item Specify the top level scheduling strategy for the interpreter
  dealing with multiple threads in VDM++ and VDM-RT (to subsequently
  be implemented in the interpreter).\\
  \hfill \developer{KennethLausdahl,++} \label{toplevelscheduling}

\item \markDone{Make the pretty printer with test coverage usable for all VDM
  dialects. \developer{KennethLausdahl, NickBattle}}  \textbf{Remarks:} missing doc



\end{enumerate}

%
% Release 0.3.0
%
\item[Release 0.3.0 (June 2010):] The aim for this release is to
  include the following features:  
\begin{enumerate}
\item  \markDone{Decide if we want to include the Update Manager or embed a number of videly used features like Sub Version control.: We will include the update manager and self update for RCP Update to reduce the effort needed to update the tool}



\item Provide help (F1) inside Eclipse; \developer{AugustoRibeiro}

\item provide goto definition and completion; \developer{AugustoRibeiro}

\item Complete the new ASTGen utility and add all sources for it to
  the tools utility in the Overture SF SVN. \developer{NickBattle} \textbf{Dependencies: None},\label{completeAST}

\item Change to the AST generated from the new version of ASTGen all
  over the Overture sources (i.e.\ this will affect the development of
  UML, JML and POtrans components). \developer{NickBattle, KennethLausdahl}, \textbf{Dependencies: \ref{completeAST}}

\item  \markDone{Replace DLTK. This includes a complete recoding of all existing Editors, Wizards, Outlines etc including a build from scratch Debug interface inspired from DLTK but only using the DBGProtocol as DLTK do.\footnote{The decision for replacing DLTK has been made based on the investigation of Eclipse where it have been found more feasible to use Eclipse directly rather than using DLTK since a lot of the existing code do work rounds to get back to eclipse. The benefit of this change in the long run is seen larger than the work to recreate some of the DLTK features.} (For lower level description of the goals please refer to appendix \ref{app:lowlevelgoals}). \developer{AugustoRibeiro,KennethLausdahl}, \textbf{Dependencies: \ref{picturesEclipse}} }

\item  \markDone{Launch configuration setup: 
	\begin{enumerate}
		\item When creating a new debug configuration it shall be named after
  the name of the project (and not just ``New configuration'').
		\item The configuration should be linked to the project so it is available from \textit{Debug As}
		\item General plugin.xml problem.
	\end{enumerate} \developer{AugustoRibeiro,KennethLausdahl}, \textbf{Dependencies: R020:\ref{analyseDLTKDebug}}\label{launchconfig}}

\item \markDone{Making sure that the run and debug buttons in the IDE work
  appropriately and does not crash whenever the user tries to start
  them. Ideally these should simply change to the debug perspective if
  no debug configuration is present and be ready to type an expression
  into a quick console where simple expressions can be typed and
  executed (and up and down in the list of commands given is
  supported). \developer{KennethLausdahl}, \textbf{Dependencies: \ref{launchconfig}} }

\item A button for moving a test case from the combinatorial testing
  perspective into the debugger such that it will stop at the right
  place for the run-time error reported for the test case
  selected. \developer{KennethLausdahl see launch shortcut to see how this can be done}, \textbf{Dependencies: \ref{launchconfig}} 

\item  \markDone{Stop all threads when a breakpoint or a run-time error has
  occured in one of the threads. \developer{NickBattle,KEL, ARI}, \textbf{Dependencies: R020:\ref{toplevelscheduling}} }

\item  \markDone{In the debug perspective it shall be possible to look into the
  contents of objects in the variables view. \developer{KEL}, \textbf{Dependencies: R020:\ref{analyseDLTKDebug},\ref{completeAST}} }

\item  \markDone{Include all definitions in the Outline view for all VDM
  dialects. \developer{AugustoRibeiro}, \textbf{Dependencies: R020:\ref{completeAST}}}  Missing RT System and Threads

\item \markDone{ When creating new projects (by wizard) in the IDE include the possibility
  for including standard libraries (i.e.\ (VDM headers) IO and Math) in the
  project. These are then to be included in a special ``lib''
  directory. \developer{??}, \textbf{Dependencies: None}} 

\item  \markDone{ Make sure that proof obligations for standard libraries are
  ignorred. \developer{NickBattle}, \textbf{Dependencies: None}  }

\item \markDone{Make it possible for users to provide static implementations to ``is not yet specified'' functions and
  operations by adding a Jar file to a lib folder in the project. \developer{KennethLaudahl,NickBattle}\label{staticIsNotYetSpecified} \textit{Prototype done}}

\item \markDone{ Make it possible for users to add a java
  implementation of a VDM class by adding a jar to the lib folder. The
  implementation should be done so one instance in java corresponds to
  one instance in VDM. No
  static. \developer{KennethLausdahl,NickBattle},
  \textbf{Dependencies: \ref{staticIsNotYetSpecified}} }
\item Make use of {\bf\ttfamily measure}'s inside the the interpreter
  so it would be checked in the same way as invariants, pre and post
  conditions. \developer{NickBattle}
\end{enumerate}





\item[Release 0.4.0 (September 2010):] The aim for this release is to
  include the following features:   
\begin{enumerate}

\item \markDone{ Investigate the possibility of Eclipse p2. (Adding Self-Update to an RCP Application) \url{http://wiki.eclipse.org/RCP_FAQ}} Will do.

\item First stable bi-directional connection from Overture/VDM++
  to/from JML. \developer{CarlosVilhena}
\item New version of the interpreter for scheduling of multiple
  threads in VDM++ and VDM-RT prepared for co-simulkation also able to
  break after a fixed amount of time. \developer{KennethLausdahl,++}
\item User documentation in pdf available as a part of the Overture
  executable so the user can search on-line in these sources
  (including a VDM-10 language manual). \developer{PeterGormLarsen,++}
\end{enumerate}

\item[Release 1.0.0 (December 2010):] The aim for this release is to
  include the following features:   
\begin{enumerate}
\item Quick fix incorporated in the editor view of the IDE for all VDM
  dialects. \developer{??}
\item Goto definition in the edit view of the IDE across all files in
  a project for all VDM dialects. \developer{??}
\item First version of refactoring support for all VDM
  dialects. \developer{??}
\item Possibly first release of proof support feature for proving a
  subset of the POs generated. \developer{NickBattle}
\item Support in the Proof Obligation Explorer view for new icons
  indicating whether the user have proved a PO or have manually
  inspected it and approved it. This also means that the icon with the
  red cross shall only be used for those where proof have failed and a
  less intemidating iconj shall be used for the POs that one have not
  yet dealt with. \developer{??}
\item Include VDMDoc (inspired by JavaDoc) for all VDM dialects. \developer{??}
\item Stable version of the UML mapper enabling proper round-trip
  engineering between VDM and UML class diagrams and enable generation
  of traces from a subset of UML sequence
  diagrams. \developer{KennethLausdahl,PeterGormLarsen} 
\item First release of code generator from VDM to a programming
  language. \developer{MarcelVerhoef}
\item Increased \LaTeX\ support in the editor such that parts outside
  the \texttt{vdm\_al} environments are highlighed as TeXClipse would
  do it. \developer{??}
\end{enumerate}
\end{description}



\begin{description}
\item[Release X (?):] The aim for this release is to
  include the following features:  
\begin{enumerate}
\item VDM Tools
	\begin{enumerate}
		\item Start VDM Tools command line prompt from a Overture project. Where the prompt is a console inside Eclipse.
		
		\item  \markDone{Create a VDM Tools project file from a Overture project and launch the VDM Tools GUI.\developer{KEL}}
		
		\item Create a number of buttons for VDM Tools code generation through the VDM Tools comman dline interface, with output in the console.
		
		\item Make a VDM Tools corba DBGPReader to enable a VDM Tools debugger to connect to the Overture IDE and use the basic debug facility.
		\item Preferences for VDM Tools
	\end{enumerate}

\item Make Java code generator

\item Make C code generator

\item Make C++ code generator

\item Make VDM Doc parser

\item Make VDM ast pretty printer for math syntax

\item Make VDM source format

\item  \markDone{Coverage Editor\developer{KEL}}

\item  \markDone{DLTK replacement

	\begin{enumerate}
		\item Recreate Debugger. New debug protocol separation from VDMJC. Launch, Debug runner, 
		\item Recreate the parser
		\item Recreate the builder
		\item Recreate the Editor
		\item Recreate the outline
		\item Recreate the package explorer
	\end{enumerate}}

\item Ant scripts to build a product
\item Tests of the plug-ins
\item Test program which can parse/type/interpret all examples and compare results with expected.
\item Extract VDM is not specified classes from Java implementation of external interpreter modules.
\item Make a VDM model of Multi-core CPU distribution/scheduling.

\end{enumerate}
\end{description}

\chapter{Contents of DESTECS releases in 2010}\label{sec:DESTECS2010}

This section presents a plan for the contents of the 2010 releases of
the DESTECS executable.

\begin{description}
\item[Release 0.1.0 (March 2010):] The aim for this release is to
  include the following features:  
\begin{enumerate}
\item Develop development environment for DESTECS enabling different
  plug-ins to be added and documenting this. \developer{KennethLausdahl}
\item Stand-alone DESTECS executable with the DESTECS icon
  \developer{KennethLausdahl} 
\item Develop wizard for creating a new DESTECS project. This shall
  create 3 different directories in the project for ``Discrete
  Event'', ``Continuous Time'' and ``Fault Modelling''
  respectively. \developer{??} 
\item Ability to invoke the 20-sim editor for 20-sim models in a
  DESTECS project. \developer{ControlLabs}
\end{enumerate}

\item[Release 0.2.0 (June 2010):] The aim for this release is to
  include the following features:  
\begin{enumerate}
\item DESTECS tool (Core + GUI) 
\item Define communication interfaces \developer{IHA+CLP}
\item 20-sim Implementation Interface \developer{CLP}
\item Overture communicating with the interface stubs \developer{IHA}
\item 20-sim communicating with the interface stubs \developer{CLP}
\end{enumerate}

\item[Release 0.3.0 (September 2010):] The aim for this release is to
  include the following features:   
\begin{enumerate}
\item Overture implement interface \developer{IHA}
\item 20-sim implement interface \developer{CLP}
\item Watertank example working \developer{IHA+CLP}
\end{enumerate}

\item[Release 1.0.0 (December 2010):] The aim for this release is to
  include the following features:   
\begin{enumerate}
\item Higher complexity models working \developer{IHA+CLP}
\item Testing and bughunting \developer{IHA+CLP}
\item Co-simulation between Overture VDM models and 20-sim models is
  fully enabled. \developer{KennethLausdahl,++}
\item Synchronising the log viewer from Overture with the plot of
  variables from 20-sim in an Eclipse setting. \developer{ControlLabs,++}
\end{enumerate}
\end{description}


\bibliographystyle{newalpha}

\bibliography{dan}

\appendix*


\chapter{Low level goals for IDE release plan.}
\label{app:lowlevelgoals}
Low level goals for IDE release plan.
\markDone{
\begin{enumerate}
\item Core:
	\begin{itemize}
	\item Utilities (VdmProject)
	\item AST
	\item VDMJ (export)
	\item Core Parser
	\item Core Builder
	\item Commands
	\end{itemize}

\item Core.Vdmpp
	\begin{itemize}
	\item Nature
	\item Content Type
	\end{itemize}

\item VDMJ Builder/Parser Update

\item UI:
	\begin{itemize}
	\item Abstract Editor
		\begin{itemize}
		\item Syntax Highlight
		\item Navigation
		\item Auto Completion
		\item ...
		\end{itemize}
	\item Outline
	\item Navigator
	\item Properties (Language Version, etc)
	\item Abstract Wizards (New Project/File)
	\item Perspective?
	\end{itemize}

\item UI.Vdmpp
	\begin{itemize}
	\item Editor
	\item Wizards
	\item Templates
	\item Perspective?
	\end{itemize}

\item Debug
	\begin{itemize}
	\item Launch Core
	\item Debug Model Core
	\item Launch UI - (Tabs, shortcuts)
	\item Debug UI - (source highlight, source lookup)
	\end{itemize}

\item Debug.Vdmpp - (Launch tabs)?


\end{enumerate}
}
\chapter{Overture Source status}
\section{Component status}

\subsection{Core}
\begin{description}

\item[ast] Scheduled for replacement
\item[astgenerator] Scheduled for replacement
\item[jmltrans] Carlos has taken an action to redevelop the initial JML transformation.
\item[parser] Scheduled for replacement
\item[potrans] Unknown
\item[proofsupport] Unknown
\item[showtrace] Stable, besides that unknown
\item[stdlib] Has beem split up and now only a few VDM libraries it still located here.
\item[traces] Stable contains CT for all dialects including filtering. 
\item[umltrans] Stable for VDMPP and VDMRT. Only from VDM to UML. The other direction is incomplete do to upgrades for VDM to UML.
\item[vdmj] Stable. New check for history counters, Delegation to Java added including a RemoteControl feature.
\item[vdmjc] Unknown
\item[vdmtools] Unknown
\item[vdmunit] Unknown

\end{description}

\subsection{IDE}

\subsubsection{Plug-ins}

\begin{description}


\item[Idedebughelper] discontinues.
\item[Latex] Supports Latex document generation of a complete specification including coverage tables and colouring. Support is provided for all dialects: SL, PP and RT.

\item[PO Viewer] Viewer/Generator of PO's for all dialects SL, PP and RT.
\item[Proofsupport] Stable but not implemented to an extend where it is usable in the IDE. (To limited)
\item[Showtrace / Real Time Log viewer] Viewer of RT log files. Stable but has an performance issue with large files. Navigation buttons also needs some work, it is not clear where they are located.
\item[Traces / Combinatorial Testing] Implemented for all dialects SL, PP and RT. Stable for now but needs to be redeveloped because of they way it evaluates traces using the internal IDE AST. Error null is currently being printed in the console when an evaluation is impossible because of a AST out of state.
\item[Umltrans] Implementation is stable for the PP and RT except the system class.
\end{description}

\subsubsection{IDE}
\begin{description}
\item[ast] Stable, scheduled to be moved to \textit{org.overutre.ide.core}
\item[builders/core]
\item[builders/vdmj]
\item[debug/core]
\item[debug/launching] Stable, VM arguments updated to a set of arguments can be parsed to the debuggibg VM.

\item[parsers/vdmj]
\item[platform]

\item[ui]
\item[utility]Stable, scheduled to be moved to \textit{org.overutre.ide.core}
\item[vdmpp/core]
\item[vdmpp/ui]
\item[vdmpp/debug/core]
\item[vdmpp/debug/ui]

\item[vdmrt/core]
\item[vdmrt/ui]
\item[vdmrt/debug/core]
\item[vdmrt/debug/ui]

\item[vdmsl/core]
\item[vdmsl/ui]
\item[vdmsl/debug/core]
\item[vdmsl/debug/ui]

\end{description}



\end{document}
