\documentclass[11pt,twoside]{report}
\usepackage{vdmlisting}
\usepackage{multicol}
\usepackage{pict2e}

\newcommand{\lbparen}{%
  \mathopen{%
    \sbox0{$()$}%
    \setlength{\unitlength}{\dimexpr\ht0+\dp0}%
    \raisebox{-\dp0}{%
      \begin{picture}(.32,1)
      \linethickness{\fontdimen8\textfont3}
      \roundcap
      \put(0,0){\raisebox{\depth}{$($}}
      \polyline(0.32,0)(0,0)(0,1)(0.32,1)
      \end{picture}%
    }%
  }%
}

\newcommand{\rbparen}{%
  \mathclose{%
    \sbox0{$()$}%
    \setlength{\unitlength}{\dimexpr\ht0+\dp0}%
    \raisebox{-\dp0}{%
      \begin{picture}(.32,1)
      \linethickness{\fontdimen8\textfont3}
      \roundcap
      \put(-0.08,0){\raisebox{\depth}{$)$}}
      \polyline(0,0)(0.32,0)(0.32,1)(0,1)
      \end{picture}%
    }%
  }%
}


\usepackage{color}        
\usepackage{alltt}  
\usepackage{hyperref}
\usepackage{amsmath}
\usepackage{unicode-math}
\usepackage{xcolor}
\usepackage{listings}
\usepackage{color}
\usepackage[toc,page]{appendix}
\usepackage{soul}
\sethlcolor{lightgray}
\lstdefinelanguage{Velocity}{
  keywords={\$Isa, true, false, \#foreach, \#if, \#end, null, \#else, \#elseif, \$node, \#, \#\#},
  keywordstyle=\color{orange}\bfseries,
  ndkeywords={\.trans, \\\<, \>},
  ndkeywordstyle=\color{gray}\bfseries,
  identifierstyle=\color{gray},
  sensitive=false,
  comment=[l]{\#},
  morecomment=[s]{\\<}{>},
  commentstyle=\color{blue}\ttfamily,
}

\lstdefinelanguage{Isabelle}{
  keywords={\\\<and\>, true, false, \#foreach, \#if, \#end, null, \#else, \#elseif, \$node, type_synonym, definition, where, abbreviation, record},
  keywordstyle=\color{blue}\bfseries,
  ndkeywords={\.trans, \\\<, \>},
  ndkeywordstyle=\color{pink}\bfseries,
  identifierstyle=\color{black},
  sensitive=false,
  comment=[l]{//},
  morecomment=[s]{\\<}{>},
  commentstyle=\color{black}\ttfamily,
  stringstyle=\color{purple}\ttfamily,
  morestring=[b]',
  morestring=[b]"
}

\definecolor{dkgreen}{rgb}{0,0.6,0}
\definecolor{gray}{rgb}{0.5,0.5,0.5}
\definecolor{mauve}{rgb}{0.58,0,0.82}

\lstset{frame=tb,
  language=Java,
  aboveskip=3mm,
  belowskip=3mm,
  showstringspaces=false,
  columns=flexible,
  basicstyle={\small\ttfamily},
  numbers=none,
  numberstyle=\tiny\color{gray},
  keywordstyle=\color{orange},
  commentstyle=\color{dkgreen},
  stringstyle=\color{mauve},
  breaklines=true,
  breakatwhitespace=true,
  tabsize=3
}
\NewDocumentCommand{\codeword}{v}{%
\texttt{\textcolor{cyan}{#1}}%
}
\newcommand{\syntaxNULL}[1]{\textcolor[rgb]{0.0,0.0,0.0}{#1}}
\newcommand{\syntaxCOMMENTA}[1]{\textcolor[rgb]{0.8,0.0,0.0}{#1}}
\newcommand{\syntaxCOMMENTB}[1]{\textcolor[rgb]{1.0,0.5176470588235295,0.0}{#1}}
\newcommand{\syntaxCOMMENTC}[1]{\textcolor[rgb]{0.4,0.0,0.8}{#1}}
\newcommand{\syntaxCOMMENTD}[1]{\textcolor[rgb]{0.8,0.4,0.0}{#1}}
\newcommand{\syntaxDIGIT}[1]{\textcolor[rgb]{1.0,0.0,0.0}{#1}}
\newcommand{\syntaxFUNCTION}[1]{\textcolor[rgb]{0.6,0.4,1.0}{#1}}
\newcommand{\syntaxINVALID}[1]{\colorbox[rgb]{1.0,1.0,0.8}{\textcolor[rgb]{1.0,0.0,0.4}{#1}}}
\newcommand{\syntaxKEYWORDA}[1]{\textcolor[rgb]{0.0,0.4,0.6}{\textbf{#1}}}
\newcommand{\syntaxKEYWORDB}[1]{\textcolor[rgb]{0.0,0.6,0.4}{\textbf{#1}}}
\newcommand{\syntaxKEYWORDC}[1]{\textcolor[rgb]{0.0,0.6,1.0}{\textbf{#1}}}
\newcommand{\syntaxKEYWORDD}[1]{\textcolor[rgb]{0.4,0.8,1.0}{\textbf{#1}}}
\newcommand{\syntaxLABEL}[1]{\textcolor[rgb]{0.00784313725490196,0.7254901960784313,0.00784313725490196}{#1}}
\newcommand{\syntaxLITERALA}[1]{\textcolor[rgb]{1.0,0.0,0.8}{#1}}
\newcommand{\syntaxLITERALB}[1]{\textcolor[rgb]{0.8,0.0,0.8}{#1}}
\newcommand{\syntaxLITERALC}[1]{\textcolor[rgb]{0.6,0.0,0.8}{#1}}
\newcommand{\syntaxLITERALD}[1]{\textcolor[rgb]{0.4,0.0,0.8}{#1}}
\newcommand{\syntaxMARKUP}[1]{\textcolor[rgb]{0.0,0.0,1.0}{#1}}
\newcommand{\syntaxOPERATOR}[1]{\textcolor[rgb]{0.0,0.0,0.0}{\textbf{#1}}}

\newcommand{\gutter}[1]{\textcolor[rgb]{0,0,0}{{|}#1}}
\newcommand{\gutterH}[1]{\textcolor[rgb]{1,0,0}{{|}#1}}

\definecolor{bgcolor}{rgb}{1.0,1.0,1.0}
\newsavebox{\opencurlybracket}%
\newsavebox{\closecurlybracket}%
\newsavebox{\lessthan}%
\newsavebox{\greaterthan}%
\newsavebox{\dollarbox}%
\newsavebox{\underscorebox}%
\newsavebox{\andbox}%
\newsavebox{\hashbox}%
\newsavebox{\backslashbox}%
\newsavebox{\atbox}%
\newsavebox{\percentbox}%
\newsavebox{\hatbox}%
\setbox\opencurlybracket=\hbox{\verb.{.}%
\setbox\closecurlybracket=\hbox{\verb.}.}%
\setbox\lessthan=\hbox{\verb.<.}%
\setbox\dollarbox=\hbox{\verb.$.}%
\setbox\underscorebox=\hbox{\verb._.}%
\setbox\andbox=\hbox{\verb.&.}%
\setbox\hashbox=\hbox{\verb.#.}%
\setbox\atbox=\hbox{\verb.@.}%
\setbox\backslashbox=\hbox{\verb.\.}%
\setbox\greaterthan=\hbox{\verb.>.}%
\setbox\percentbox=\hbox{\verb.\%.}%
\setbox\hatbox=\hbox{\verb.^.}%
\def\urltilda{\kern -.15em\lower .7ex\hbox{\~{}}\kern .04em}

\def\vdmlisting{\texttt{vdmlisting}}
\def\envvdmsl{\texttt{vdmsl}}
\usepackage{color}
\definecolor{dkgreen}{rgb}{0,0.6,0}
\definecolor{gray}{rgb}{0.5,0.5,0.5}
\definecolor{mauve}{rgb}{0.58,0,0.82}
\usepackage[citestyle=alphabetic,bibstyle=authortitle]{biblatex}
\addbibresource{Dissertation.bib}
\usepackage[utf8]{inputenc}
\usepackage{graphicx}
\usepackage[a4paper,width=150mm,top=25mm,bottom=25mm]{geometry}
\graphicspath{ {images/} }
\usepackage{fancyhdr}
\pagestyle{fancy}
\fancyhead{}
\fancyhead[RO,LE]{Creating a tool for the translation of VDM to Isabelle HOL}
\fancyfoot{}
\fancyfoot[LE,RO]{\thepage}
\fancyfoot[LO,CE]{\leftmark}
\fancyhead[LE,LO]{Jamie Simm}
\title{
{Creating a tool for the translation of VDM to Isabelle HOL}\\
{\vspace{0.5cm}}
{\large BSc Computer Science}\\
{\vspace{0.5cm}}
{\large Newcastle University}\\
{\vspace{0.5cm}}
{\large Project Supervisor Dr Leonardo J.S Freitas}\\
{\vspace{3cm}}
{\small Word count : 13751 in text, 245 outside text.}\\
{\vspace{2.5cm}}
{\includegraphics[width=5cm, height=2cm ]{newcastle.jpg}}
}
\author{Jamie Simm}
\date{2019}

\begin{document}
\maketitle

\chapter*{Abstract}
This dissertation aims to explain and describe the implementation of and need for a VDM (Vienna Development Method) to Isabelle HOL (Higher Order Logic) translation tool. The tool is applied to a VDM model of the POLAR organ persufflation machine's alarm system. As Isabelle theories can be proven mathematically by the modeller, it is hoped that this tool will be useful in the verification of software by eliminating the need for manual VDM to Isabelle translation.
\chapter*{Declaration}
I declare that this dissertation represents my own work except where otherwise stated. The length of this document is only so large for its code snippets and figures, the majority of which take up a considerable amount of the page. Excluding them, this document would approximately measure thirty pages in length. 

\chapter*{Acknowledgments}
I would like to thank my supervisor, Dr Leo Freitas: for guiding me toward the project of developing this tool, which I have thoroughly enjoyed; for introducing me to the world of formal specification, Isabelle and VDM; and for fostering my interest in automated translation and proof tools. Secondly, I would like to thank Casper Thule Hansen of Aarhus University, Denmark, for setting up the environment upon which this tool was built and teaching me how to write transformation visitors. I believe that this project may not have been so successful without such an essential and flexible platform for it to be built upon. 

\tableofcontents

\chapter{Introduction}
\subsection{The POLAR machine and persufflation}
Organ preservation is a vital factor in the success and availability of allotransplantation\footnote{The transplantation of organs, tissue, or cells from a genetically distinct individual of the same species.} as a treatment for illness, conventional methods like Static Cold Storage (SCS) have a number of issues - advancements in this field could serve to transform medicine\parencite{Giwa2017}. One such issue is that SCS is incapable of sufficient oxygen delivery during preservation\parencite{Pappas2005}. The POLAR project aims to address this issue by designing a machine to automate another method of organ preservation, persufflation, which gives a far higher oxygen supply per gram of tissue. 

\subsection{Verifying the correctness of the machine to ensure reliability}
To achieve verification of the system we apply formal modelling techniques to its specification in order to derive an unambiguous representation of its components and its operation. For POLAR, this has been done by translating the machine's various alarm conditions into a VDM model. VDM is a formal modelling language, it allows the modeller to abstract the system into a mathematical representation, its syntax rules and semantics are so precisely defined that there is no room for disagreement about the model’s properties. This means that the POLAR VDM model can be analysed using mathematical proof. In principal we can: prove that a program, in this case the alarm system, is correct with respect to a specification and prove that the POLAR VDM model embodies a property such as safety - critical in a medical system. While VDM has tools for generating proof obligations, it has no support for mathematical proof of those obligations.

One way to prove a VDM model is to translate it into HOL\footnote{Higher order logic, essentially higher-order simple predicate logic, quantifies over an arbitrary number of nested sets i.e. where first-order logic quantifies only variables that range over individuals and second-order logic quantifies variables that range over both individuals and sets; third-order also quantifies over sets of sets etc. Higher Order Logic is a union of first, second third up to any number of nested sets.} so that it can then be proven using the proof assistant Isabelle, this is how the POLAR model will be proven. Ordinarily, translation is done manually, this can be an arduous process for the modeller/prover who should only be concerned with the proof of the model rather than the process of translation. 

Currently, there exists an effort to create a tool that will automate translation so that the modeller/prover can focus their efforts on proof of the system, as this is the pertinent part of the verification process\parencite{VDM2ISAGit}. In its current state at the time of beginning this dissertation the tool does not have the functionality to translate the POLAR model and can only translate integer basic types as well as arithmetic and predicate. With more work, applying the automated translation tool to the POLAR model could save a considerable amount of time in its verification, work on proving the systems alarm system mathematically could begin sooner and therefore be more thorough. Human errors made during manual translation would be removed, reducing the likelihood that the modeller wastes time proving an incorrect translation.

\section{Aims and Objectives}
\subsection{Aim}
The aim of this dissertation is to extend the current VDM to Isabelle translation tool so that it can translate more components of the POLAR model.
\subsection{Objectives}
\begin{itemize}
	\item Understand the Java visitor design pattern.
	\item Understand the existing tool architecture.
	\item Create Velocity templates and Java visitors for more VDM constructs.
	\item Apply the translator to the POLAR model.
\end{itemize}

\section{Document structure}
In the following sections, I will provide background by explaining and discussing the area/domain that the project affects and explores as well as clarifying existing works on the translation tool. In the following section, I will present the implementation of my contribution to the tool with a description of implementation issues and its architecture. Before concluding this dissertation I will apply the tool to the POLAR example and review my time working on the project.


\chapter{Background}
This section attempts to explain and discuss the key concepts that are the foundation of this project: the POLAR organ persufflation machine and its formal model in VDM; mathematically proving a model in Isabelle and implementing a tool to translate a VDM model into Isabelle HOL; an outline of the problems that this dissertation aims to tackle, as well as the proposed solutions to those problems.

\section{Persufflation}
Organ persufflation is a method for organ preservation in which a donor organ is submerged in a cold preservation solution, a catheter\footnote{A flexible tube inserted through a narrow opening into a body cavity.} is then inserted into the vasculature\footnote{The arrangement of blood vessels in the body, or within an organ.} of the organ and oxygen is pumped into the organ through the catheter, thereby oxygenating\footnote{Supply, treat, charge, or enrich with oxygen.} the organ, see Fig\ref{fig:Persufflation}.\begin{figure}
        \center{\includegraphics[width=\textwidth]
        {Images/Persufflation.jpeg}}
        \caption{\label{fig:Persufflation} Persufflation \parencite{Dholakia2018}}
      \end{figure} Through this method, more oxygen is delivered to the organ than the other most popular methods and by doing so preservation period can be extended by up to 48-72 hours\parencite{Suszynski2013}. 

However, Persufflation is a delicate process. Oxygen flow, temperature and pressure both inside and outside of the organ must be kept within a precise margin; slightly anomalous variations in any of these variables, could result in damage to the organ. 
Until now, humans have controlled these variables manually: typically, an alarm will sound if a variable exceeds its boundary and an attendant will adjust the value by hand. This method is not precise enough under such safety critical circumstances as human error may result in damage to the organ pre-transplant or failure of the organ post-transplant, due to poor condition.

Through software, the POLAR project's persufflation machine would enable effective, long term persufflation by removing the human element. By exploiting the speed, precision and accuracy of a computer system, POLAR hopes to be able to control these variables with high precision and within a minuscule margin of error. However, as is the nature of software of such complexity, the alarm system that informs changes to variables may be rife with errors, with high potential for unintended behaviour\parencite{JamiaOCW154}. In a safety critical system such as this, it is imperative that software components do not malfunction.


\section{POLAR System Complexity}
The POLAR machine is a Cyber-Physical System - collaborating computational elements controlling physical entities, which interact with humans and their environment. For the sake of keeping safety in mind, it is important to note that the software that will be developed for POLAR is part of a system - a combination of interacting elements organized to achieve one or more stated purposes\parencite{walden_roedler_forsberg_hamelin_shortell_2007}. As such, the system will have several dimensions to take into account - building complexity. Physical variables such as external and internal pressure, temperature, atmospheric composition, movement and more. Also computational variables such as security protocols, fail safes, varying software; Human variables, how will the machine be handled, used, what errors may humans introduce into the system? 

Specifically, human contribution is arguably the central source of error and complexity in a system. Separate discipline-specific development processes result in components being developed in a disjointed manner. In the case of POLAR, software is developed separate to the hardware, a machine engineer is not able to know precisely how their chip's resources will be handled by kernel processes written by the operating system developer, and so they are unable to understand which features might cause error, which limitations the software will have. All of this risks late discovery of defects, namely at the integration stage, when it is highly cumbersome to go back. 

\section{Modelling}
The best way to bridge the gap between varying components, development teams, dimensions of the system, is to create a model of the system, an abstraction of all of these variables together in one unambiguous representation so that the variables of the system can be tested and scrutinised as a whole, their interaction monitored and alterations made where error is found. Models allow us to explore a design space before we build, and physically integrate, the components within it. Interactions are modelled as contracts, assumptions and guarantees of what will or should happen rather than how. As an architect models a structure to smaller scale before it is built - so is a software and hardware system modelled before it is integrated and implemented. This provides evidence for trust, reduces risk of bugs and can even identify improvements to the current design. In POLAR, a system that will someday contribute to human well-being, it is easy to see how an optimal design and faster development will be beneficial. For an increasing amount of software, this model takes the form of a formal specification.

\section{Formal Specification}
Formal specifications are used to describe a system, analyse its behaviour and to aid in its design by verifying key properties of interest through rigorous and effective reasoning tools\parencite{FORMAL1}\parencite{FORMAL2}. These specifications are formal in the sense that they have a syntax, their semantics fall within one domain, and they are able to be used to infer useful information\parencite{Lamsweerde2000}.\parencite{wikiFS} Ordinarily, a formal specification of a system goes no further than a UML diagram of its components and mostly all projects have some form of it. This is a good first approximation, but it is imprecise. For safety critical systems, a mathematical, unambiguous abstraction of the system, which states the effect of computation, is the only acceptable level of specification. A mathematical representation can be analysed, altered and proven using deduction and reasoning techniques - and therefore, so can the design of the system with respect to its specification. A mathematical specification is encoded in an appropriate formal modelling programming language, for POLAR this is VDM-SL.


\section{Vienna Development Method Specification Language (VDM-SL)}
Though readers of this dissertation might already be familiar with VDM, a brief overview of the language is necessary to provide motivation for this project. VDM is a state-based modelling language which allows the modeller to formally specify structure, behaviour and logical constraints. The Vienna Development Method (VDM) was established in the 1970's originating from IBM Laboratory, Vienna. The language has syntax to represent mathematical representation of a specification. Mathematical constructs including, sets, sequences and integers, to name a small few, define types of data that are maintained and transformed in the system. How this data is represented, what restrictions are placed on the data, represented as invariants on types and values, and what data forms the persistent state of the system, define its data set. Behaviour of the system is encoded as functions that represent functionality, and as operations which modify its persistent state. Pre and post conditions for operations and functions are purposeful in both restricting and checking their operation. Errors in the model identify errors in the specification of the system, the specification can then be altered, and the design improved to eliminate errors before the new specification is written in VDM-SL again so that all of this can be repeated. Many iterations of this - model, check, alter, model again - cycle incrementally identify and eliminate errors until none, or as few as possible within a reasonable threshold, can be found.



\subsection{VDM-SL Example, Alarm System}
Below, VDM code represents the alarm system for a nuclear power plant reactor. Experts are paged when certain variables' values cross safe boundaries. Certain experts are on shift at certain times and paging is decided based on period of time over boundaries as well as other important factors such as an expert’s qualification.


\begin{vdmsl}[label=lst:AlarmSL.vdmsl,caption=Types of data used in the alarm system in VDM-SL]
types

Schedule = map Period to set of Expert;

Period = token;

Expert :: expertid : ExpertId
quali : set of Qualification
inv ex == ex.quali <> {};

ExpertId = token;
Qualification = <Elec> | <Mech> | <Bio> | <Chem>;
Alarm :: alarmtext : seq of char
quali : Qualification;
\end{vdmsl}

Above, an invariant on the Expert type is defined to ensure that the expert does not have an empty set of qualifications i.e. the expert is qualified, and their qualifications are recorded.

\begin{vdmsl}[label=lst:AlarmSL.vdmsl,caption=Alarm system's values in VDM-SL]
values
 
  p1:Period = mk_token("Monday day");
  ps : set of Period = {p1,p2,p3,p4,p5};

  eid8:ExpertId = mk_token(190);
  
  e1:Expert = mk_Expert(eid1,{<Elec>});
  exs : set of Expert = {e1,e2,e3,e4,e5,e6,e7,e8};
\end{vdmsl}

Some of the values in the system are shown above. A set of experts is created, one such expert having a particular id and electrical qualification.

\begin{vdmsl}[label=lst:AlarmSL.vdmsl,caption=An alarm system function in VDM-SL which takes a set of experts\, a qualification and returns a boolean value.]
functions
  QualificationOK: set of Expert * Qualification -> bool
  QualificationOK(exs2,reqquali) ==
    exists ex in set exs2 & reqquali in set ex.quali;
\end{vdmsl}
\hfill\break
\hfill\break
\begin{vdmsl}[label=lst:AlarmSL.vdmsl,caption=Alarm system's persistent state in VDM-SL\, t he state is represented as a record type with the important persistent data as fields\, here schedule and alarms.]
state Plant of
schedule : Schedule
alarms : set of Alarm

\end{vdmsl}

\begin{vdmsl}[label=lst:AlarmSL.vdmsl,caption=Operations on the state of the alarm system in VDM-SL\, operations manipulate data stored in the persistent state and mimic the operation of the system.]
operations

NumberOfExperts: Period ==> nat
NumberOfExperts(peri) == is not yet specified
pre peri in set dom schedule;

ExpertIsOnDuty: Expert ==> set of Period

ExpertIsOnDuty(ex) == is not yet specified;

ExpertToPage: Alarm * Period ==> Expert

ExpertToPage(a,peri) == is not yet specified;
\end{vdmsl}

\section{Isabelle Translation}
VDM provides an unambiguous mathematical representation of the system, \textbf{however it does not provide any support for mathematical proof of the specification}. Mathematically proving or disproving correctness of the VDM model and its intended algorithms, allows us to say without doubt that they are correct with respect to their formal specification. Isabelle allows us to write code contracts in Higher Order Logic statements and provides an automated theorem proving assistant to prove them. Code contracts are assumptions and guarantees in the model, features written as VDM constructs. Contracts include but are not limited to: \begin{itemize}
  \item Properties that always hold (i.e. invariants).
  \item Assumptions (pre conditions) and commitments (post conditions).
  \item Proof obligations of interest such as satisfiability\footnote{A logical check to verify that operations are feasible} and reification\footnote{Do representations of the data in the system agree with one another, are they compatible?}, which if proved would establish the consistency of the model.
  \item Proof obligations (POs) of what correctness means, verifying that we have built the correct model.
  \item Sanity checks on functionality, validating that the model has been built correctly.
\end{itemize}
Isabelle is a functional programming language, constructs are represented as fields and curried functions, all collected together in a \syntaxKEYWORDA{theory}, \syntaxKEYWORDA{.thy}, file. A theory is a named collection of types, functions, and theorems, much like a module in a programming language or a specification in a specification language like VDM. Isabelle contains a theory file \syntaxKEYWORDA{"Main"}, which is a union of all the basic, predefined theories like arithmetic, lists, sets, etc. which are used by the automated theorem prover to inform proof assistants. Below, a module \syntaxLITERALA{"VDMToolkit"}\footnote{VDMToolkit is written and maintained by Dr Leonardo Jose Simoes Freitas. Newcastle University.} is included in the imports. This is very important in the VDM to Isabelle translation steps as, similar to \syntaxKEYWORDA{"Main"}, it provides type-checked VDM constructs with pre, post conditions and invariants for VDM types like VDMNat1 and VDMSet, as VDM represents these things differently to Isabelle.

To use Isabelle to prove a model, that model must first be translated from VDM into Isabelle HOL. For the above example, translations are below.

\hfill\break
\hfill\break
\ttfamily
\syntaxNULL{}\gutter{\ \ \ \ 1{|}\ }\syntaxKEYWORDA{theory}{\ }Alarm\hspace*{\fill}\\
\gutter{\ \ \ \ 2{|}\ }\syntaxKEYWORDB{imports}{\ }\syntaxLITERALA{"../../lib/VDMToolkit"}\hspace*{\fill}\\
\gutter{\ \ \ \ 3{|}\ }\syntaxKEYWORDB{begin}\hspace*{\fill}\\
\gutter{\ \ \ 42{|}\ }{\ }{\ }\hspace*{\fill}\\
\gutter{\ \ \ 42{|}\ }{\ }{\ }\hspace*{\fill}\\
\gutter{\ \ \ \ 7{|}\ }\syntaxKEYWORDA{type\usebox{\underscorebox}synonym}{\ }Period{\ }{\ }{\ }\syntaxOPERATOR{=}{\ }VDMToken\hspace*{\fill}\\
\gutter{\ \ \ 14{|}\ }\syntaxKEYWORDA{definition}\hspace*{\fill}\\
\gutterH{\ \ \ 15{|}\ }{\ }{\ }inv\usebox{\underscorebox}Period{\ }\syntaxOPERATOR{::}{\ }\syntaxLITERALA{"Period{\ }⇒{\ }𝔹"}\hspace*{\fill}\\
\gutter{\ \ \ 16{|}\ }{\ }{\ }\syntaxKEYWORDB{where}\hspace*{\fill}\\
\gutter{\ \ \ 17{|}\ }{\ }{\ }\syntaxLITERALA{"inv\usebox{\underscorebox}Period{\ }≡{\ }inv\usebox{\underscorebox}True"}\hspace*{\fill}\\
\gutter{\ \ \ 18{|}\ }{\ }{\ }\hspace*{\fill}\\
\gutter{\ \ \ 19{|}\ }\syntaxKEYWORDA{type\usebox{\underscorebox}synonym}{\ }ExpertId{\ }\syntaxOPERATOR{=}{\ }VDMToken\hspace*{\fill}\\
\gutterH{\ \ \ 20{|}\ }\hspace*{\fill}\\
\gutter{\ \ \ 21{|}\ }\syntaxKEYWORDA{datatype}{\ }Qualification{\ }\syntaxOPERATOR{=}{\ }Elec{\ }\syntaxOPERATOR{|}{\ }Mech{\ }\syntaxOPERATOR{|}{\ }Bio{\ }\syntaxOPERATOR{|}{\ }Chem\hspace*{\fill}\\
\gutter{\ \ \ 22{|}\ }\hspace*{\fill}\\
\gutter{\ \ \ 23{|}\ }\syntaxKEYWORDA{definition}\hspace*{\fill}\\
\gutter{\ \ \ 24{|}\ }{\ }{\ }inv\usebox{\underscorebox}Qualification{\ }\syntaxOPERATOR{::}{\ }\syntaxLITERALA{"Qualification{\ }⇒{\ }𝔹"}\hspace*{\fill}\\
\gutterH{\ \ \ 25{|}\ }{\ }{\ }\syntaxKEYWORDB{where}\hspace*{\fill}\\
\gutter{\ \ \ 26{|}\ }{\ }{\ }\syntaxLITERALA{"inv\usebox{\underscorebox}Qualification{\ }≡{\ }inv\usebox{\underscorebox}True"}\hspace*{\fill}\\
\gutter{\ \ \ 27{|}\ }{\ }{\ }\hspace*{\fill}\\
\gutter{\ \ \ 28{|}\ }\syntaxKEYWORDA{record}{\ }Alarm{\ }\syntaxOPERATOR{=}\hspace*{\fill}\\
\gutter{\ \ \ 29{|}\ }{\ }{\ }alarm\usebox{\underscorebox}alarmtext{\ }\syntaxOPERATOR{::}{\ }\syntaxLITERALA{"char{\ }VDMSeq"}{\ }{\ }\hspace*{\fill}\\
\gutterH{\ \ \ 30{|}\ }{\ }{\ }alarm\usebox{\underscorebox}quali{\ }{\ }{\ }{\ }{\ }\syntaxOPERATOR{::}{\ }Qualification\hspace*{\fill}\\
\gutter{\ \ \ 27{|}\ }{\ }{\ }\hspace*{\fill}\\
\gutter{\ \ \ 38{|}\ }\syntaxKEYWORDA{definition}\hspace*{\fill}\\
\gutter{\ \ \ 39{|}\ }{\ }{\ }inv\usebox{\underscorebox}Alarm{\ }\syntaxOPERATOR{::}{\ }\syntaxLITERALA{"Alarm{\ }⇒{\ }𝔹"}\hspace*{\fill}\\
\gutterH{\ \ \ 40{|}\ }{\ }{\ }\syntaxKEYWORDB{where}\hspace*{\fill}\\
\gutter{\ \ \ 41{|}\ }{\ }{\ }\syntaxLITERALA{"inv\usebox{\underscorebox}Alarm{\ }≡{\ }inv\usebox{\underscorebox}True"}\hspace*{\fill}\\
\gutter{\ \ \ 42{|}\ }{\ }{\ }\hspace*{\fill}\\
\gutter{\ \ \ 43{|}\ }\syntaxKEYWORDA{record}{\ }Expert{\ }\syntaxOPERATOR{=}\hspace*{\fill}\\
\gutter{\ \ \ 44{|}\ }{\ }{\ }expert\usebox{\underscorebox}expertid{\ }{\ }\syntaxOPERATOR{::}{\ }ExpertId\hspace*{\fill}\\
\gutterH{\ \ \ 45{|}\ }{\ }{\ }expert\usebox{\underscorebox}quali{\ }{\ }{\ }{\ }{\ }\syntaxOPERATOR{::}{\ }\syntaxLITERALA{"Qualification{\ }VDMSet"}\hspace*{\fill}\\
\gutter{\ \ \ 46{|}\ }{\ }{\ }\hspace*{\fill}\\
\gutter{\ \ \ 47{|}\ }\syntaxKEYWORDA{definition}\hspace*{\fill}\\
\gutter{\ \ \ 48{|}\ }{\ }{\ }inv\usebox{\underscorebox}Expert{\ }\syntaxOPERATOR{::}{\ }\syntaxLITERALA{"Expert{\ }⇒{\ }𝔹"}\hspace*{\fill}\\
\gutter{\ \ \ 49{|}\ }{\ }{\ }\syntaxKEYWORDB{where}\hspace*{\fill}\\
\gutterH{\ \ \ 50{|}\ }{\ }{\ }\syntaxLITERALA{"inv\usebox{\underscorebox}Expert{\ }e{\ }≡}\hspace*{\fill}\\
\gutter{\ \ \ 51{|}\ }\syntaxLITERALA{{\ }{\ }{\ }{\ }{\ }{\ }let{\ }eq{\ }={\ }(expert\usebox{\underscorebox}quali{\ }e){\ }in{\ }}\hspace*{\fill}\\
\gutter{\ \ \ 52{|}\ }\syntaxLITERALA{{\ }{\ }{\ }{\ }{\ }{\ }{\ }{\ }inv\usebox{\underscorebox}SetElems{\ }inv\usebox{\underscorebox}True{\ }eq{\ }∧}\hspace*{\fill}\\
\gutter{\ \ \ 53{|}\ }\syntaxLITERALA{{\ }{\ }{\ }{\ }{\ }{\ }{\ }{\ }eq{\ }≠{\ }\usebox{\opencurlybracket}\usebox{\closecurlybracket}"}\hspace*{\fill}\\
\gutter{\ \ \ 54{|}\ }\hspace*{\fill}\\
\gutter{\ \ \ 74{|}\ }\syntaxKEYWORDA{type\usebox{\underscorebox}synonym}{\ }Schedule{\ }\syntaxOPERATOR{=}{\ }\syntaxLITERALA{"Period{\ }⇀{\ }Expert{\ }set"}{\ }\hspace*{\fill}\\
\gutterH{\ \ \ 75{|}\ }{\ }{\ }\hspace*{\fill}\\
\gutter{\ \ \ 77{|}\ }\syntaxKEYWORDA{definition}\hspace*{\fill}\\
\gutter{\ \ \ 78{|}\ }{\ }{\ }inv\usebox{\underscorebox}Schedule{\ }\syntaxOPERATOR{::}{\ }\syntaxLITERALA{"Schedule{\ }⇒{\ }𝔹"}\hspace*{\fill}\\
\gutter{\ \ \ 79{|}\ }{\ }{\ }\syntaxKEYWORDB{where}\hspace*{\fill}\\
\gutterH{\ \ \ 80{|}\ }{\ }{\ }\syntaxLITERALA{"inv\usebox{\underscorebox}Schedule{\ }s{\ }≡{\ }}\hspace*{\fill}\\
\gutter{\ \ \ 81{|}\ }\syntaxLITERALA{{\ }{\ }{\ }{\ }{\ }{\ }inv\usebox{\underscorebox}Map{\ }inv\usebox{\underscorebox}Period{\ }(inv\usebox{\underscorebox}SetElems{\ }inv\usebox{\underscorebox}Expert){\ }s{\ }}\hspace*{\fill}\\
\gutter{\ \ \ 82{|}\ }\syntaxLITERALA{{\ }∧}\hspace*{\fill}\\
\gutter{\ \ \ 83{|}\ }\syntaxLITERALA{{\ }{\ }{\ }{\ }{\ }{\ }(∀{\ }exs1{\ }∈{\ }rng{\ }s{\ }.{\ }}\hspace*{\fill}\\
\gutter{\ \ \ 84{|}\ }\syntaxLITERALA{{\ }{\ }{\ }{\ }{\ }{\ }{\ }{\ }{\ }{\ }exs1{\ }≠{\ }\usebox{\opencurlybracket}\usebox{\closecurlybracket}{\ }∧}\hspace*{\fill}\\
\gutterH{\ \ \ 85{|}\ }\syntaxLITERALA{{\ }{\ }{\ }{\ }{\ }{\ }{\ }{\ }{\ }{\ }(∀{\ }ex1{\ }∈{\ }exs1{\ }.{\ }∀{\ }ex2{\ }∈{\ }exs1{\ }.{\ }}\hspace*{\fill}\\
\gutter{\ \ \ 86{|}\ }\syntaxLITERALA{{\ }{\ }{\ }{\ }{\ }{\ }{\ }{\ }{\ }{\ }{\ }{\ }{\ }{\ }ex1{\ }≠{\ }ex2{\ }⟶{\ }(expert\usebox{\underscorebox}quali{\ }ex1){\ }≠{\ }(expert\usebox{\underscorebox}quali{\ }ex2)))"}\hspace*{\fill}\\
\gutter{\ \ \ 87{|}\ }{\ }{\ }\hspace*{\fill}\\
\gutter{\ \ 716{|}\ }\syntaxKEYWORDB{end}\hspace*{\fill}\\
\hfill\break
\hfill\break
\rmfamily
The translation 'recipe' detailing the general method for translation will be detailed in the Implementation section of this document later, see \ref{itr}. It is evident that translations might become very intricate and cumbersome for more complex models, human error during repetitive tasks is common, and manual translation takes time. The complexity comes from the level of detail required for each VDM construct's translation: each type must have an invariant checking all of its subsequent types; when it comes to function translations, which we will see later, each function needs a pre and a post condition also; state must be initialised in a separate function which must also have a pre and a post condition and so on. The tool, whose development is detailed in the next chapter, automates translation to leave the modeller free to concentrate on proving the translated model rather than translating it manually - and consequently spending time fixing minor errors and filling in gaps in that translation. \footnote{See the Isabelle manual for more information: https://isabelle.in.tum.de/dist/Isabelle2018/doc/tutorial.pdf}

\chapter{Implementation}
\chapter{Implementation}

In this chapter is explained how different parts of Overture were implemented as well as important information on those different parts concerning extension, mantainence, tips, etc.

\section{Launcher}
The launcher is the part of the eclipse framework responsible to "launch" the code to run, in our case a model to be interpreted. Associated with a launch is a launch configuration that contains information used to launch the code. The inspiration for the Overture implementation can be found \href{http://www.eclipse.org/articles/Article-Launch-Framework/launch.html}{here}\footnote{\url{http://www.eclipse.org/articles/Article-Launch-Framework/launch.html}} \cite{Szurszewski03}. 
Also connected with the launcher is its UI part which is the window that pops up when a defining a new launch configuration.

\subsection{The players}

\begin{description}
\item[launchConfigurationType:] it is an extension point where new launch configurations can be declared. A launch configuration describes a way to launch a model; 

\item[launchConfigurationDelegate:] it is a delegate associated with a launchConfigurationType. The delegate is in charge of, using the configuration set for launch, starting up the interpreter process. The configuration contains which is the launch mode (i.e. "run", "debug") among other settings;

\item[launchConfigurationTypeImages:] it is an extension point to select an image associated with a launch configuration type;

\item[launchConfigurationTabGroups:] it is an extension point that defines a tab group associated with a certain configurationType. The tab group is a group of tabs presented when creating a new launch configuration which contain the enables the user to graphically set the settings to be used in the launch (configuring it);

\item[launchShortcuts:] it is an extension point that enables the definition of shortcuts to launch models without configuring the launch, i.e. without bringing up the launch configuration tabs.

\item[launchGroups:] didnt get there yet... not sure if it will be needed.

\end{description}

\subsection{launchConfigurationType}
To define a new launchConfigurationType this is the extension point.

\lstset{language=XML, caption=lauchConfigurationTypes extension point}
\begin{lstlisting}
 <extension
         point="org.eclipse.debug.core.launchConfigurationTypes">
      <launchConfigurationType
            delegate="org.overture.ide.debug.core.launch.VdmLaunchConfiguration"
            id="org.overture.ide.debug.launchConfigurationType"
            modes="run, debug"
            name="Overture Launch Configuration Type"
            public="true">
      </launchConfigurationType>
   </extension>
\end{lstlisting}

Important fields here are the
\begin{description}


\item[delegate:] a class that implements \class{ILaunchConfigurationDelegate}. The \class{VdmLauchConfigurationDelegate} is located in the package \epoint{org.overture.debug.core.launching} and it contains the code that checks the given configuration and that launches the interpreter thread;

\item[delegateName:] humman readable name for the delegate;

\item[id:] a suitable "id";

\item[modes:] the modes which the launcher support, in our case: "run" without debug and debug mode;

\item[name:] the human readable name for the configuration;

\item[public:] setting this attribute to true enables the launch configuration to be presented in the UI;

\item[sourceLocatorId:] we will come back to this attribute later in the debug section;

\item[sourcePathComputerId:] same as the one above.
\end{description}

There is only method to be written with the launch configuration, in the \class{VdmLauchConfigurationDelegate} and it is the  method \java{launch}.
\lstjava{Snippets of VdmLauchConfigurationDelegate}
\begin{lstlisting}
public void launch(ILaunchConfiguration configuration, String mode,
    ILaunch launch, IProgressMonitor monitor) throws CoreException {
	 ...
	 }
\end{lstlisting}

The arguments are:
\begin{description}
\item[configuration:]  the configuration that the user selects in the UI is passed to the launch delegate through this variable;

\item[mode:] the mode in which the interpreter is supposed to run;

\item[launch:] created process and debug targets are added to this variable;

\item[monitor:] UI progress monitor of the launch;
\end{description}

In the \java{launch} method, the first thing to be done is to check if the actual configuration is correct. After these checks, the interpreter process is launched, if the debug mode is selected then the process should be attached to the debug target.

\lstjava{Snippets of VdmLauchConfigurationDelegate}
\begin{lstlisting}
public void launch(ILaunchConfiguration configuration, String mode,
	ILaunch launch, IProgressMonitor monitor) throws CoreException {
	
	// initial checks
	...
	// building process string
	String commandLine = "java.exe -cp vdmj-2.0.0.jar" + ...;
	...
	if (mode.equals(ILaunchManager.DEBUG_MODE)) {
		// start intepreter in debug mode
		commandLine = commandLine + "debug options";
		// start the debugger
		Process process = Runtime.getRuntime().exec(commandLine);
		// attach process to debug target
		IProcess p = DebugPlugin.newProcess(launch, process, path);
		IDebugTarget target = new VdmDebugTarget(launch,p,s);
		launch.addDebugTarget(target);
		}
	}
\end{lstlisting}
This is the simplified version of the launch delegate, in practise, because the Vdm interpreter tries to connect via a socket to the IDE when executed in debug mode, we have to start a socket server before launching the process.


\textbf{Tip 1:} the launch configuration (\java{configuration}) is still accessible while the debugger is running via the DebugPlugin. It is possible to set attributes that can be used later in the debug.
\lstjava{Changing the configuration}
\begin{lstlisting}
	...
	ILaunchConfigurationWorkingCopy workConfiguration = 
		configuration.getWorkingCopy();		
	workConfiguration.setAttribute("attribute","value");
	configuration = workConfiguration.doSave();
	...
\end{lstlisting}

\textbf{Tip 2:} use the constants in ILaunchManager to compare with the launch mode
\lstjava{Comparing the launch mode}
\begin{lstlisting}
	...
	if (mode.equals(ILaunchManager.DEBUG_MODE)) {
	...
	}
	...
\end{lstlisting}



\subsection{launchConfigurationTypeImages}
The \epoint{launchConfigurationTypeImages} extension point is a way to assign a icon to a launch configuration.


\lstset{language=XML, caption=launchConfigurationTypeImages extension point}
\begin{lstlisting}
<extension
         point="org.eclipse.debug.ui.launchConfigurationTypeImages">
      <launchConfigurationTypeImage
            configTypeID="org.overture.ide.debug.launchConfigurationType"
            icon="icons/cview16/overture_nav.gif"
            id="org.overture.ide.debug.launchConfigurationTypeImage">
      </launchConfigurationTypeImage>
</extension>
\end{lstlisting}




Fields:
\begin{description}
\item[configTypeID:] the id of the \epoint{launchConfigurationType} the image is used for;
\item[icon:] the icon to be used;
\item[id:] an id for this \epoint{launchConfigurationTypeImages}.
\end{description}

There is not much more to say about this extension point. After defining a \epoint{launchConfigurationType} and a  \epoint{launchConfigurationTypeImages} for it, the UI when the \textit{Debug Configurations...} button is pressed shows like in figure \ref{fig:launchConfigurationType}.

\begin{figure}[htb]
\centering
\includegraphics[width=\textwidth]{figures/launchConfigurationType}
\caption{VDM Launch Configuration}
\label{fig:launchConfigurationType}
\end{figure}

\subsection{Laucher UI part}

After defining the 2 extension points above, we have a launch type and a launch icon, but normally we would like to receive some user input (configuration) of the model to launch, such as: which is the entry point method of the model, if the user wants test coverage and so on. These options should be presented in the \textit{Debug Configuration} (figure \ref{fig:launchConfigurationType}) window when pressing our type of configuration. This is made by declaring a \epoint{launchConfigurationTabGroup}.
\begin{program}
\scriptsize
\begin{verbatim}
<extension
         point="org.eclipse.debug.ui.launchConfigurationTabGroups">
      <launchConfigurationTabGroup
            class="org.overture.debug.ui.tabs.VdmLaunchConfigurationTabGroup"
            description="Vdm Launch Config"
            id="org.overture.debug.vdmLaunchConfigurationTabGroup"
            type="org.overture.debug.vdmLaunchConfigurationType">
      </launchConfigurationTabGroup>
</extension>
\end{verbatim}
\caption{launchConfigurationTabGroup extension point}
\normalsize
\end{program}
Fields:
\begin{description}
\item[class:] the class implementing the tab group;

\item[description:] a human readable description;

\item[id:] a id for the tab group;

\item[type:] the id of the \epoint{launchConfigurationType} we want to define the tabs for.
\end{description}

Here there is also the task of defining one new class \class{VdmLaunchConfigurationTabGroup} in which only one method needs to be defined if it extends the \class{AbstractLaunchConfigurationTabGroup} class that eclipse provides.

\lstjava{createTabs method in \class{VdmLaunchConfigurationTabGroup}}
\begin{lstlisting}
	ILaunchConfigurationTab[] tabs = new ILaunchConfigurationTab[]{
				new VdmppMainLaunchConfigurationTab(mode),
				new VdmInterpreterTab(),
				new SourceLookupTab(),
				new CommonTab()
		};
		setTabs(tabs);	
\end{lstlisting}
Then the only task is to define the tabs that we would like to provide to the user. These user defined tabs should either extend the \class{AbstractLaunchConfigurationTab} or implement \class{ILaunchConfigurationTab}. Eclipse guidelines also advise to include always the \class{CommonTab} as one of the tabs and also the \class{SourceLookupTab} if we use source lookup, which will omit by now but will introduced on the debug section. The tabs defined in this class appear in the \textit{Debug Configuration} window as shown in figure \ref{fig:launchConfigurationTabs}.

\begin{figure}[htb]
\centering
\includegraphics[width=\textwidth]{figures/launchConfigurationTabs}
\caption{VDM Launch Configuration Tabs}
\label{fig:launchConfigurationTabs}
\end{figure}

\subsubsection{Launch configuration shortcuts}
TBD.



\chapter{Results}
The following section will describe the outcome of development, it will discuss mainly, but among other things: a measure of what work has been done, which constructs were translated for example; metrics of the development, how long development took, how many lines of code were written; limitations of the current state of the tool, what is missing from the tool and the things that the tool cannot achieve; which aims and objectives were achieved? which were not? As well as any new aims and objectives born from the development process; the complexity of the project.

\section{Work Done}
As mentioned, the purpose of development was to create IR transformation functionality that would facilitate translation from the IR into Isabelle for as many VDM constructs as possible. The test framework provided a large list of tests for the translation of VDM constructs to Isabelle, the test strategy was that they should all pass, there were tests for each construct, as well as tests for a combination of constructs. By the end of development every one of these tests passed successfully, each translation matched a correct manual translation for each construct test and construct combination test.
\begin{vdmsl}[label=lst:MapIntChar.vdmsl, caption=The VDM-SL test file\, MapIntChar.vdmsl. This test file tests that the translation of a VDM-SL map\, and a combination of int and char constructs\, matches a previously manually translated correct translation\, seen below in MapIntChar.vdmsl.result. If the translation matches\, the tool successfully translates this construct.]
types

t = map int to char
\end{vdmsl} 
\begin{lstlisting}[language=Isabelle, label=lst:MapIntChar.vdmsl.result, caption=The MapIntChar.vdmsl.result file specifies that the below is the correct translation\, and the output of the tool applied to MapIntChar.vdmsl should match it. The file specifies that not only should the type be translated but so should also have generated invariant for it. The output\, under the word "Got:" matches\, and so this test passes\, this construct is successfully translated.]
 --- Expected: ---
theory DEFAULT
  imports VDMToolkit
begin

type_synonym t = "VDMInt \<rightharpoonup> char"




definition
    inv_t :: "(t) \<Rightarrow> \<bool>"
    where
    "inv_t t \<equiv> isa_invTrue t"

end
 --- Got: ---
theory DEFAULT
  imports VDMToolkit
begin

type_synonym t = "VDMInt
 \<rightharpoonup> char"



definition
	inv_t :: "(t) \<Rightarrow> \<bool>"
    where
    "inv_t t \<equiv> isa_invTrue t"

end
\end{lstlisting}

Additional to the pre-defined tests, I created some further tests that should pass, testing things like nested collection translation and translation of multiple parameter functions. The partial list of tests and their contents is in the appendices along with their result files, in appendix \ref{testing}, section \ref{vdmsl}. The collective code is far too long to include each test.
\begin{multicols}{2}
\textbf{\emph{Functions}}\hfill\break
\emph{Explicit}
\begin{itemize}
\item FuncApply1Param.vdmsl
\item FuncApply3Params.vdmsl
\item FuncApplyNoParam.vdmsl
\item FuncDecl1Param.vdmsl
\item FuncDecl2Params.vdmsl
\item FuncDeclNoParam.vdmsl
\item FuncDepSimple.vdmsl
\item FuncPost.vdmsl
\item FuncPre.vdmsl
\item FuncPrePost.vdmsl
\item NotYetSpecified.vdmsl
\end{itemize}
\emph{Implicit}
\begin{itemize}
\item 1ParamNoPre.vdmsl
\item 1ParamPrePost.vdmsl
\item 2ParamsPrePost.vdmsl
\item 2ParamsNoPre.vdmsl
\item 2ParamsNoPost.vdmsl
\item NoParamNoPre.vdmsl
\item NoParamPrePost.vdmsl
\end{itemize}
\textbf{\emph{State}}
\begin{itemize}
\item EqualsInit.vdmsl
\item PredicateInit.vdmsl
\end{itemize}
\textbf{\emph{Types}}\hfill\break
\emph{InvTypes}
\begin{itemize}
\item InvInt.vdmsl
\item InvRecordDummyInv.vdmsl
\item InvSet.vdmsl
\end{itemize}
\emph{NoInv}
\begin{itemize}
\item Char.vdmsl
\item CharNatTokenTuple.vdmsl
\item CharSeqIntSetTuple.vdmsl
\item Int.vdmsl
\item IntCharTuple.vdmsl
\item IntIntTuple.vdmsl
\item MapIntChar.vdmsl
\item MapIntInt.vdmsl
\item Nat.vdmsl
\item Nat1.vdmsl
\item Rat.vdmsl
\item Real.vdmsl
\item Rec1Field.vdmsl
\item Rec2Fields.vdmsl
\item Rec2FieldsDiffTypes.vdmsl
\item SeqInt.vdmsl
\item SeqNat.vdmsl
\item \item SetInt.vdmsl
\item Token.vdmsl
\end{itemize}
\textbf{\emph{Values}}
\begin{itemize}
\item BoolType.vdmsl
\item ExplicitInt.vdmsl
\item ExplicitNat.vdmsl
\item ExplicitNat1.vdmsl
\item ExplicitReal.vdmsl
\item ImplicitNumericExp.vdmsl
\item IndependentDefsOrder.vdmsl
\item IntExpVarExp.vdmsl
\item NestedVarExp.vdmsl
\item VarExp.vdmsl
\end{itemize}
\end{multicols}
This means that every VDM construct, with the exception of a few special cases, discussed in \ref{lott}. Basic constructs such as \lstinline[language=Isabelle]{Int}, \lstinline[language=Isabelle]{char}, \lstinline[language=Isabelle]{set}, essentially the type construct, were translated without much hassle, relative to function or invariant translations. Successful translation of basic constructs presented a nice quirk of the development process that I decided to exploit early on, if visitors and their methods are built flexibly, translations almost pass like dominoes when one of these tests pass. Translation of the basic types and their invariants means that any consequent combination of types is translated successfully. For example, successful translation of \lstinline[language=Isabelle]{Map}, \lstinline[language=Isabelle]{Int} and \lstinline[language=Isabelle]{char}, powered by a flexible visitor for generating invariants from any type, means that any combination of them also translate successfully - see \ref{mapintchar}. This feature means that translation of a most type constructs were relatively trivial after one good general transformation visitor was written. 

This was not true for all types however, and more tricky transformations were required for record types due to the uniqueness of their field structure, however even then, due to the fact that a flexible invariant and type generation visitor was set up, transformation functionality was already majorly established for each field and only required some massaging. Due to the way that invariant generation visitors had been written, different invariants for different fields were generated, the difficult part here was to combine them into one invariant field.
\begin{vdmsl}[caption=Rec2FieldsDiffTypes.vdmsl\, two fields needed to have invariant checks in the invariant but were generated in separate invariant functions.]
types

RecType :: x : char
           y: real
\end{vdmsl}

\begin{lstlisting}[language=Isabelle, caption=After removing malformed individual invariants from the AST\, one was generated and combined\, no additional code had to be written to translate each field's type as this had already been done when translating type constructs.]
  imports VDMToolkit
begin

record RecType =
        recType_x :: char
        recType_y :: \<real>
    



definition
	inv_RecType :: "RecType \<Rightarrow> \<bool>"
    where
    "inv_RecType r \<equiv> (isa_invTrue (recType_x r) \<and> isa_invTrue (recType_y r))"

end
\end{lstlisting}

The more difficult translations were for functions - invariant or otherwise. Functions have many more layers of complexity to be concerned with during translation, the tool had to worry about more than just one name, one type and keywords generated by velocity. Now there were special cases for implicit functions, multiple parameter functions, no parameter functions, functions with existing pre and post conditions with their own identifier variables. For implicit functions, the malformed function body needed to be removed from the AST leaving only correctly translated pre and post conditions. Pre conditions carried the issue of having to traverse the AST and pull out only the parameters of a function for their method type. Post conditions required more fiddling by adding the result of their parent function to their parameters. Most of all though, the most difficult thing in the development was generating expressions. Generating expressions was so sophisticated for the fact that they were needed to be generated by almost every construct in the AST, and as such, the methods involved needed to be highly flexible. Figuring out the recursion of methods such as \lstinline[language=Isabelle]{genAnd} and \lstinline[language=Isabelle]{buildInvForType} was a difficult task with so many different inputs to consider and took up the majority of my time with the project. Again though, once this infrastructure had been set up, all invariant check expression generation was successful for functions and invariants regardless of their type, structure or parameters.

Briefly, translation of state declaration followed the same formula as other transformations before it, and all that needed be done was to create a new function that would hold initialisation of state. 

To summarise, thoughtful generation of flexible transformations from the start, made it trivial to translate what should have been highly complex constructs later in the development, and this I think is what makes the tool successful.

\subsection{Complexity of This Project}
This project had a high complexity because of the various frameworks and environments that drive it.
For the Java side alone, there was an AST to work with and efficiently traverse and manipulate. I had to learn the structure of the AST, the families of nodes within it and their interaction. I also had to learn the visitor design pattern and get to grips with the various adaptors, each one with its own subtlety and aberration. Familiarity with a number of frameworks was also essential, the code generation platform, overture, and the ability to write and use Apache Velocity. Before development could begin, it was necessary to be able to write, use and understand two formal specification languages, both Isabelle and VDM-SL. Finally, and perhaps most importantly, the tool required me to learn and be able to perform translation from VDM into HOL and Isabelle so that the tool produced the correct translation.

\subsection{Limitations of the Tool} \label{lott}
Due to time restrictions, a few translations are not sophisticated, the tool simply prints out the translations of the node that it can cope with, and for the most part this is acceptable, however for more intricate Isabelle, the tool might struggle to produce an error free translation. These include the template type, and the \lstinline[language=Isabelle]{ARecordPatternIR} pattern template.

\section{Reflection and Development Process Metrics}
The entire development process lasted for 46 days in total, and each day averaged successful translation of two constructs, this is because later in the project development slowed down to debug null pointer exceptions and develop utility methods for more complex function translations. To the merit of the tool, Velocity templates were kept as small as possible, at 2.6, rounded to 3, lines on average. It is worth noting that the code in the appendices and in this project were at three times their size on the first development iteration, but have been refined and cut down for efficiency. Given more time with the project, it would be rewarding to revise the code further, some control flow blocks could be transformed into more complex separate state methods and would therefore become better programs for it. In all, the functionality of the tool is split across only 14 visitor classes, development contributed 2,049 additions and 171 deletions. See https://github.com/overturetool/overture/compare/cth/isagen...SamieJim:cth/isagen, to compare the progress of development.

\section{Application to POLAR}




\chapter{Conclusion}
To conclude, the issues laid out in the introduction and background chapters are eliminated by this tool, the aims and objectives of this dissertation achieved. As explained in the previous section; the development of the tool was difficult due the vast complexity of its environment. Considering this, and considering that the tool works correctly, this project should be considered a success, this conclusion will attempt justify this statement. 

\section{Measuring The Success Of This Project}
From the aims and objectives defined in the Introduction chapter all were achieved. The aim of this dissertation was to extend the current VDM to Isabelle translation tool so that it can translate more components of the POLAR model, this has been achieved and arguably exceeded, as more constructs than are present in the VDM-SL POLAR model can be translated. The objectives of this project were:
\begin{itemize}
	\item Understand the Java visitor design pattern.
	\item Understand the existing tool architecture.
	\item Create Velocity templates and Java visitors for more VDM constructs.
	\item Apply the translator to the POLAR model.
\end{itemize}

All of which were achieved. I fully understand the visitor design pattern and the existing tool architecture. Though I have created two Velocity templates and modified more, it was not necessary, and in fact would have been irresponsible, to modify or create too much Velocity as a good template is just that, a template, it should be abstract, short, and minimal. Large amounts of Java was written in existing visitors and a few more created, both for transformation and utility. Application to the POLAR model was a success, it is translated correctly with the exception of some minor foibles already covered, caused by a few kinks in the tool.

\section{Future Development}
This tool lays the foundations for more sophisticated prototypes to be developed on top of it. Flexible methods and translation of basic VDM constructs have made the tool extensible, but in some areas, like the way that symbols are reorganised to their correct positions in set and sequence initialisation, the tool uses a "hacky" method to iterate through where they should have been in their correct place to begin with.\footnote{This quirk of the tool is a side effect of reverse recursion to generate symbols.} Though they present no practical obstacle and the tool works well for the tests that have been run, given more time, these inelegant methods could be transformed or removed entirely after some thought so that they do not present potential problems down the line for larger models. For actual additional functionality however, it would be highly beneficial if this tool could also generate lemmas, the units of auxiliary information used by Isabelle to prove a model. This would be done in an approximately similar way to the way that it has been done for invariant generations during the development of this tool, although they would have to be generated \emph{after} the generation of the translated model. The reason for this, is that the lemmas are based on other constructs in the Isabelle file as well as theories present in the "Main" Isabelle theory directory mentioned previously. 

Should lemmas be generated successfully, the tool would become more than a translation engine and code generator. The tool would then, with the assistance of the Isabelle automated theorem assistant, become an automated proof tool that could have vastly beneficial implementations in proving the specification of safety critical systems.  


\section{Impact of This Project}
The introduction explains that before the development of this tool, manually translate an Isabelle model. This is cumbersome for two reasons, the first is that the modeller is human and makes mistakes, despite the high intelligence true of those working in the field of mathematical formal specification, human error is ever present. Small errors made in translation could result in faults with proof, e.g. proving the wrong thing. This means that the soundness of proof of an Isabelle proven specification could come into question. The second reason is the massive additional time required to translate a model into Isabelle. Each construct must be manually typed and considered, this means that it simply is not feasible to translate large Isabelle models, like the one for POLAR into Isabelle. The important thing to note is that the man hours required to translate a model could not only be spent proving it, which is the difficult part, but if we are to use Isabelle to prove all safety critical systems, as we should to ensure the upmost safety assurance, then a potentially life saving/changing technology could be delayed while such an automatable step is performed. This tool remedies both, for the first, automation means that we only have to prove that a translation is sound once, when we create the translation recipe that the tool follows, then any subsequent translations done by the tool are always likely to be correct, practically eliminating the concern of modeller error. The second encumbrance of time is removed entirely thanks to the speed of the computer, automated translation could speed up the development of safety critical systems with the beneficial side effect of leaving the modeller more time to focus on proof.

\printbibliography

\begin{appendices}
\chapter{Tool Code Before Development} \label{beforecode}
\section{Transformations} \label{transformationsbefore}

\subsection{IsaBasicTypesConv} \label{IsaBasicTypesConvbefore}
\begin{lstlisting}
package org.overturetool.cgisa.transformations;

import org.overture.cgisa.isair.analysis.DepthFirstAnalysisIsaAdaptor;
import org.overture.codegen.ir.*;
import org.overture.codegen.ir.declarations.*;
import org.overture.codegen.ir.types.AIntNumericBasicTypeIR;
import org.overture.codegen.trans.assistants.TransAssistantIR;

import java.util.Map;
import java.util.stream.Collectors;

/***
 * Visitor to convert basic VDM types to VDMToolkit types
 */
public class IsaBasicTypesConv extends DepthFirstAnalysisIsaAdaptor {

    private final Map<String, ATypeDeclIR> isaTypeDeclIRMap;
    private final TransAssistantIR t;
    private final AModuleDeclIR vdmToolkitModuleIR;
    private final IRInfo info;
    private final static String isa_VDMInt = "isa_VDMInt";

    public IsaBasicTypesConv(IRInfo info, TransAssistantIR t, AModuleDeclIR vdmToolkitModuleIR) {
        this.t = t;
        this.info = info;
        this.vdmToolkitModuleIR = vdmToolkitModuleIR;

        this.isaTypeDeclIRMap = this.vdmToolkitModuleIR.getDecls()
                .stream()
                .filter(d -> {
                    if (d instanceof ATypeDeclIR)
                        return true;
                    else
                        return false;
                }).map(d -> (ATypeDeclIR) d)
                .collect(Collectors.toMap(x -> ((ANamedTypeDeclIR) x.getDecl()).getName().getName(), x -> x));
    }

    //Transform int to isa_VDMInt
    public void caseAIntNumericBasicTypeIR(AIntNumericBasicTypeIR x){
        if(x.getNamedInvType() == null)
        {
            AIntNumericBasicTypeIR a = new AIntNumericBasicTypeIR();
            // Retrieve isa_VDMInt from VDMToolkit
            ATypeDeclIR isa_td = isaTypeDeclIRMap.get(this.isa_VDMInt);

            x.setNamedInvType((ANamedTypeDeclIR)isa_td.getDecl().clone());
        }

    }
}
\end{lstlisting}
\subsection{IsaInvGenTrans.java} \label{IsaInvGenTransbefore}
\begin{lstlisting}[language=Java]


public class IsaInvGenTrans extends DepthFirstAnalysisIsaAdaptor {

    private final AModuleDeclIR vdmToolkitModule;
    private final Map<String, ATypeDeclIR> isaTypeDeclIRMap;
    private IRInfo info;
    private final Map<String, AFuncDeclIR> isaFuncDeclIRMap;

    public IsaInvGenTrans(IRInfo info, AModuleDeclIR vdmToolkitModuleIR) {
        this.info = info;
        this.vdmToolkitModule = vdmToolkitModuleIR;

        this.isaFuncDeclIRMap = this.vdmToolkitModule.getDecls().stream().filter(d ->
        {
            if (d instanceof AFuncDeclIR)
                return true;
            else
                return false;
        }).map(d -> (AFuncDeclIR) d).collect(Collectors.toMap(x -> x.getName(), x -> x));

        this.isaTypeDeclIRMap = this.vdmToolkitModule.getDecls().stream().filter(d -> {
            if (d instanceof ATypeDeclIR)
                return true;
            else
                return false;
        }).map(d -> (ATypeDeclIR) d).collect(Collectors.toMap(x -> ((ANamedTypeDeclIR) x.getDecl()).getName().getName(), x -> x));


    }

    @Override
    public void caseATypeDeclIR(ATypeDeclIR node) throws AnalysisException {
        super.caseATypeDeclIR(node);

        String typeName = IsaInvNameFinder.findName(node.getDecl());
        SDeclIR decl = node.getDecl();
        SDeclIR invFun = node.getInv();

        if(invFun == null)
        {
            // Invariant function
            AFuncDeclIR invFun_ = new AFuncDeclIR();
            invFun_.setName("inv_" + typeName);

            // Define the type signature
            //TODO: Type should be XTypeInt - correct?
            AMethodTypeIR methodType = new AMethodTypeIR();
            STypeIR t = IsaDeclTypeGen.apply(node.getDecl());
            methodType.getParams().add(t.clone());
            methodType.setResult(new ABoolBasicTypeIR());
            invFun_.setMethodType(methodType);

            // Generate the pattern
            AIdentifierPatternIR identifierPattern = new AIdentifierPatternIR();
            identifierPattern.setName("x");
            AFormalParamLocalParamIR afp = new AFormalParamLocalParamIR();
            afp.setPattern(identifierPattern);
            afp.setType(t.clone()); // Wrong to set entire methodType?
            invFun_.getFormalParams().add(afp);

            // Generate the expression
            SExpIR expr = IsaInvExpGen.apply(decl, identifierPattern, methodType.clone(), isaFuncDeclIRMap);
            invFun_.setBody(expr);

            // Insert into AST
            AModuleDeclIR encModule = node.getAncestor(AModuleDeclIR.class);
            if(encModule != null)
            {
                encModule.getDecls().add(invFun_);
            }

            System.out.println("");

        }
    }

    public String GenInvTypeDefinition(String arg){
        return "Definition\n" +
                "   inv_" + arg+ " :: \"" + arg + " \\<Rightarrow> \\<bool>\"\n" +
                "   where\n" +
                "";
    }

}
\end{lstlisting}

\subsection{IsaBasicTypesConv} \label{IsaBasicTypesConvbefore}
\begin{lstlisting}

/***
 * Visitor to convert basic VDM types to VDMToolkit types
 */
public class IsaBasicTypesConv extends DepthFirstAnalysisIsaAdaptor {

    private final Map<String, ATypeDeclIR> isaTypeDeclIRMap;
    private final TransAssistantIR t;
    private final AModuleDeclIR vdmToolkitModuleIR;
    private final IRInfo info;
    private final static String isa_VDMInt = "isa_VDMInt";

    public IsaBasicTypesConv(IRInfo info, TransAssistantIR t, AModuleDeclIR vdmToolkitModuleIR) {
        this.t = t;
        this.info = info;
        this.vdmToolkitModuleIR = vdmToolkitModuleIR;

        this.isaTypeDeclIRMap = this.vdmToolkitModuleIR.getDecls()
                .stream()
                .filter(d -> {
                    if (d instanceof ATypeDeclIR)
                        return true;
                    else
                        return false;
                }).map(d -> (ATypeDeclIR) d)
                .collect(Collectors.toMap(x -> ((ANamedTypeDeclIR) x.getDecl()).getName().getName(), x -> x));
    }

    //Transform int to isa_VDMInt
    public void caseAIntNumericBasicTypeIR(AIntNumericBasicTypeIR x){
        if(x.getNamedInvType() == null)
        {
            AIntNumericBasicTypeIR a = new AIntNumericBasicTypeIR();
            // Retrieve isa_VDMInt from VDMToolkit
            ATypeDeclIR isa_td = isaTypeDeclIRMap.get(this.isa_VDMInt);

            x.setNamedInvType((ANamedTypeDeclIR)isa_td.getDecl().clone());
        }

    }
}
\end{lstlisting}
\subsection{IsaDeclTypeGen} \label{IsaDeclTypeGenbefore}
\begin{lstlisting}
public class IsaDeclTypeGen extends AnswerIsaAdaptor<STypeIR> {

    public static STypeIR apply(INode node) throws AnalysisException {
        IsaDeclTypeGen finder = new IsaDeclTypeGen();
        return node.apply(finder);
    }

    public STypeIR caseANamedTypeDeclIR(ANamedTypeDeclIR n)
    {
        AIntNumericBasicTypeIR a = new AIntNumericBasicTypeIR();
        a.setNamedInvType(n.clone());
        return a;
    }



    public STypeIR caseARecordTypeDeclIR(ARecordDeclIR n)
    {
        return null;
    }

    @Override
    public STypeIR createNewReturnValue(INode node) throws AnalysisException {
        return null;
    }

    @Override
    public STypeIR createNewReturnValue(Object node) throws AnalysisException {
        return null;
    }
}

\end{lstlisting}
\subsection{IsaInvExpGen} \label{IsaInvExpGenbefore}
\begin{lstlisting}

/*
Generates the expression for an invariant.
Example:
    VDM spec:   types
                test = nat
    Invariant expression: isa_inv_VDMNat i
    where i is a parameter to this visitor.
 */
public class IsaInvExpGen extends AnswerIsaAdaptor<SExpIR> {

    AIdentifierPatternIR ps;
    AMethodTypeIR methodType;
    private final Map<String, AFuncDeclIR> isaFuncDeclIRMap;

    public IsaInvExpGen(AIdentifierPatternIR ps, AMethodTypeIR methodType, Map<String, AFuncDeclIR> isaFuncDeclIRMap)
    {
        this.ps = ps;
        this.methodType = methodType;
        this.isaFuncDeclIRMap = isaFuncDeclIRMap;
    }

    public static SExpIR apply(SDeclIR decl, AIdentifierPatternIR afp, AMethodTypeIR methodType, Map<String, AFuncDeclIR> isaFuncDeclIRMap) throws AnalysisException {
        IsaInvExpGen finder = new IsaInvExpGen(afp, methodType, isaFuncDeclIRMap);
        return decl.apply(finder);
    }

    @Override
    public SExpIR caseANamedTypeDeclIR(ANamedTypeDeclIR node) throws AnalysisException {
        STypeIR type = node.getType();

        // Find invariant function
        AFuncDeclIR fInv = this.isaFuncDeclIRMap.get("isa_invTrue");

        // Create ref to function
        AIdentifierVarExpIR fInvIdentifier = new AIdentifierVarExpIR();
        fInvIdentifier.setName(fInv.getName());
        fInvIdentifier.setSourceNode(fInv.getSourceNode());
        fInvIdentifier.setType(fInv.getMethodType());

        // Crete apply expr
        AApplyExpIR exp = new AApplyExpIR();
        exp.setType(new ABoolBasicTypeIR());
        AIdentifierVarExpIR iVarExp = new AIdentifierVarExpIR();
        iVarExp.setName(this.ps.getName());
        iVarExp.setType(this.methodType);
        exp.getArgs().add(iVarExp);
        exp.setRoot(fInvIdentifier);

        return exp;
    }

    @Override
    public SExpIR caseARecordDeclIR(ARecordDeclIR node) throws AnalysisException {
        throw new AnalysisException();
    }

    @Override
    public SExpIR createNewReturnValue(INode node) throws AnalysisException {
        return null;
    }

    @Override
    public SExpIR createNewReturnValue(Object node) throws AnalysisException {
            return null;
    }


    public SExpIR caseASeqSeqType(ASeqSeqTypeIR node)
            throws AnalysisException {
        if(node.getSeqOf().getTag()!= null)
        {
            Object t = node.getSeqOf().getTag();

            // We are referring to another type, and therefore we stop here. This is the instantiation of the polymorphic function.
            /*
            For VDM:
             */
            // Return expression corresponding to: isa_invSeqElemens[token](isa_true[token], p)
        }
        else {
            //We need to keep going
        }
        throw new AnalysisException();
    }

    public SExpIR caseATokenBasicTypeIR(ATokenBasicTypeIR n) throws AnalysisException
    {

        AApplyExp e = new AApplyExp();

        throw new AnalysisException();

    }


    public SExpIR caseASetSetTypeIR(ASetSetTypeIR node) throws AnalysisException {
        throw new AnalysisException();

    }


}
\end{lstlisting}

\subsection{IsaInvNameFinder} \label{IsaInvNameFinderbefore}
\begin{lstlisting}
public class IsaInvNameFinder extends AnswerIsaAdaptor<String>
{
    public static String findName(INode node) throws AnalysisException {
        IsaInvNameFinder finder = new IsaInvNameFinder();
        return node.apply(finder);
    }

    @Override
    public String caseANamedTypeDeclIR(ANamedTypeDeclIR node) throws AnalysisException {
        return node.getName().getName();
    }

    @Override
    public String caseARecordDeclIR(ARecordDeclIR node) throws AnalysisException {
        return node.getName();
    }

    @Override
    public String createNewReturnValue(INode node) throws AnalysisException {
        return null;
    }

    @Override
    public String createNewReturnValue(Object node) throws AnalysisException {
        return null;
    }
}
\end{lstlisting}

\section{CodeGen Platform}
\subsection{IsaGen} \label{IsaGenbefore}
\begin{lstlisting}
/*
 * #%~
 * VDM to Isabelle Translation
 * %%
 * Copyright (C) 2008 - 2015 Overture
 * %%
 * This program is free software: you can redistribute it and/or modify
 * it under the terms of the GNU General Public License as
 * published by the Free Software Foundation, either version 3 of the
 * License, or (at your option) any later version.
 * 
 * This program is distributed in the hope that it will be useful,
 * but WITHOUT ANY WARRANTY; without even the implied warranty of
 * MERCHANTABILITY or FITNESS FOR A PARTICULAR PURPOSE.  See the
 * GNU General Public License for more details.
 * 
 * You should have received a copy of the GNU General Public
 * License along with this program.  If not, see
 * <http://www.gnu.org/licenses/gpl-3.0.html>.
 * #~%
 */

package org.overturetool.cgisa;

import java.io.File;
import java.io.StringReader;
import java.io.StringWriter;
import java.util.ArrayList;
import java.util.HashSet;
import java.util.LinkedList;
import java.util.List;

import org.apache.velocity.Template;
import org.apache.velocity.app.Velocity;
import org.apache.velocity.runtime.RuntimeServices;
import org.apache.velocity.runtime.RuntimeSingleton;
import org.apache.velocity.runtime.parser.ParseException;
import org.apache.velocity.runtime.parser.node.SimpleNode;
import org.overture.ast.analysis.AnalysisException;
import org.overture.ast.definitions.SClassDefinition;
import org.overture.ast.expressions.PExp;
import org.overture.ast.modules.AModuleModules;
import org.overture.codegen.ir.*;
import org.overture.codegen.ir.declarations.AModuleDeclIR;
import org.overture.codegen.merging.MergeVisitor;
import org.overture.codegen.utils.GeneratedData;
import org.overture.codegen.utils.GeneratedModule;
import org.overture.typechecker.util.TypeCheckerUtil;
import org.overturetool.cgisa.transformations.*;

/**
 * Main facade class for VDM 2 Isabelle IR
 *
 * @author ldc
 */
public class IsaGen extends CodeGenBase {

    public IsaGen()
    {
        this.addInvTrueMacro();

        this.getSettings().setAddStateInvToModule(false);
        this.getSettings().setGenerateInvariants(true);
    }
    //TODO: Auto load files in macro directory
    public static void addInvTrueMacro(){
        StringBuilder sb = new StringBuilder("#macro ( invTrue $node )\n" +
                "    definition\n" +
                "        inv_$node.Name :: $node.Name \\<RightArrow> \\<bool>\n" +
                "        where\n" +
                "        \"inv_$node.Name \\<equiv> inv_True\"\n" +
                "#end");
        addMacro("invTrue",new StringReader(sb.toString()));
        Template template = new Template();
    }

    public static void addMacro(String name, StringReader reader){
        try {
            Template template = new Template();
            RuntimeServices runtimeServices = RuntimeSingleton.getRuntimeServices();

            SimpleNode simpleNode = runtimeServices.parse(reader, name);
            template.setRuntimeServices(runtimeServices);
            template.setData(simpleNode);
            template.initDocument();
        } catch (ParseException e)
        {
            System.out.println("Failed with: " + e);
        }
    }

    public static String vdmExp2IsaString(PExp exp) throws AnalysisException,
            org.overture.codegen.ir.analysis.AnalysisException {
        IsaGen ig = new IsaGen();
        GeneratedModule r = ig.generateIsabelleSyntax(exp);
        if (r.hasMergeErrors()) {
            throw new org.overture.codegen.ir.analysis.AnalysisException(exp.toString()
                    + " cannot be generated. Merge errors:"
                    + r.getMergeErrors().toString());
        }
        if (r.hasUnsupportedIrNodes()) {
            throw new org.overture.codegen.ir.analysis.AnalysisException(exp.toString()
                    + " cannot be generated. Unsupported in IR:"
                    + r.getUnsupportedInIr().toString());
        }
        if (r.hasUnsupportedTargLangNodes()) {
            throw new org.overture.codegen.ir.analysis.AnalysisException(exp.toString()
                    + " cannot be generated. Unsupported in TargLang:"
                    + r.getUnsupportedInTargLang().toString());
        }

        return r.getContent();
    }


    /**
     * Main entry point into the Isabelle Translator component. Takes an AST and returns corresponding Isabelle Syntax.
     *
     * @param statuses The IR statuses holding the nodes to be code generated.
     * @return The generated Isabelle syntax
     * @throws AnalysisException
     *
     */
    @Override
    protected GeneratedData genVdmToTargetLang(List<IRStatus<PIR>> statuses) throws AnalysisException {


        // Typecheck the VDMToolkit module and generate the IR
        TypeCheckerUtil.TypeCheckResult<List<AModuleModules>> listTypeCheckResult1 =
                TypeCheckerUtil.typeCheckSl(new File("src/test/resources/VDMToolkit.vdmsl"));
        AModuleModules isaToolkit = listTypeCheckResult1.result.
                stream().
                filter(mod -> mod.getName().getName().equals("VDMToolkit")).
                findAny().
                orElseThrow(() -> new AnalysisException("Failed to find VDMToolkit module"));
        super.genIrStatus(statuses, isaToolkit);

        // Get the VDMToolkit module IR
        IRStatus<PIR> vdmToolkitIR = statuses.stream().filter(x -> x.getIrNodeName().equals("VDMToolkit")).findAny().orElseThrow(() -> new AnalysisException("Failed to find VDMToolkit IR node"));
        AModuleDeclIR vdmToolkitModuleIR = (AModuleDeclIR) vdmToolkitIR.getIrNode();


        GeneratedData r = new GeneratedData();
        try {


            // Apply transformations
            for (IRStatus<PIR> status : statuses) {
                if(status.getIrNodeName().equals("VDMToolkit")){
                    System.out.println("");
                } else {


                    // make init expression an op
                    StateInit stateInit = new StateInit(getInfo());
                    generator.applyPartialTransformation(status, stateInit);

                    // transform away any recursion cycles
                    GroupMutRecs groupMR = new GroupMutRecs();
                    generator.applyTotalTransformation(status, groupMR);

                    if (status.getIrNode() instanceof AModuleDeclIR) {
                        AModuleDeclIR cClass = (AModuleDeclIR) status.getIrNode();
                        // then sort remaining dependencies
                        SortDependencies sortTrans = new SortDependencies(cClass.getDecls());
                        generator.applyPartialTransformation(status, sortTrans);
                    }

                    
                    // Transform all token types to isa_VDMToken
                    // Transform all nat types to isa_VDMNat
                    // Transform all nat1 types to isa_VDMNat
                    // Transform all int types to isa_VDMInt

                    IsaBasicTypesConv invConv = new IsaBasicTypesConv(getInfo(), this.transAssistant, vdmToolkitModuleIR);
                    generator.applyPartialTransformation(status, invConv);

                    IsaInvGenTrans invTrans = new IsaInvGenTrans(getInfo(), vdmToolkitModuleIR);
                    generator.applyPartialTransformation(status, invTrans);
                }
            }

            r.setClasses(prettyPrint(statuses));
        } catch (org.overture.codegen.ir.analysis.AnalysisException e) {
            throw new AnalysisException(e);
        }
        return r;

    }

    public GeneratedModule generateIsabelleSyntax(PExp exp)
            throws AnalysisException,
            org.overture.codegen.ir.analysis.AnalysisException {
        IRStatus<SExpIR> status = this.generator.generateFrom(exp);

        if (status.canBeGenerated()) {
            return prettyPrint(status);
        }

        throw new org.overture.codegen.ir.analysis.AnalysisException(exp.toString()
                + " cannot be code-generated");
    }


    private List<GeneratedModule> prettyPrint(List<IRStatus<PIR>> statuses)
            throws org.overture.codegen.ir.analysis.AnalysisException {
        // Apply merge visitor to pretty print Isabelle syntax
        IsaTranslations isa = new IsaTranslations();
        MergeVisitor pp = isa.getMergeVisitor();

        List<GeneratedModule> generated = new ArrayList<GeneratedModule>();

        for (IRStatus<PIR> status : statuses) {
            if(status.getIrNodeName().equals("VDMToolkit")){
                System.out.println("");
            } else {
                generated.add(prettyPrintNode(pp, status));
            }

        }

        // Return syntax
        return generated;
    }

    private GeneratedModule prettyPrint(IRStatus<? extends INode> status)
            throws org.overture.codegen.ir.analysis.AnalysisException {
        // Apply merge visitor to pretty print Isabelle syntax
        IsaTranslations isa = new IsaTranslations();
        MergeVisitor pp = isa.getMergeVisitor();
        return prettyPrintNode(pp, status);
    }

    private GeneratedModule prettyPrintNode(MergeVisitor pp,
                                            IRStatus<? extends INode> status)
            throws org.overture.codegen.ir.analysis.AnalysisException {
        INode irClass = status.getIrNode();

        StringWriter sw = new StringWriter();

        irClass.apply(pp, sw);

        if (pp.hasMergeErrors()) {
            return new GeneratedModule(status.getIrNodeName(), irClass, pp.getMergeErrors(), false);
        } else if (pp.hasUnsupportedTargLangNodes()) {
            return new GeneratedModule(status.getIrNodeName(), new HashSet<VdmNodeInfo>(), pp.getUnsupportedInTargLang(), false);
        } else {
            // Code can be generated. Ideally, should format it
            GeneratedModule generatedModule = new GeneratedModule(status.getIrNodeName(), irClass, sw.toString(), false);
            generatedModule.setTransformationWarnings(status.getTransformationWarnings());
            return generatedModule;
        }
    }
}
\end{lstlisting}

\chapter{Tool Code After Development} \label{aftercode}
\section{Transformations} \label{transformationsafter}

\subsection{IsaInvNameFinder} \label{IsaInvNameFinderafter}
\begin{lstlisting}
public class IsaInvNameFinder extends AnswerIsaAdaptor<String>
{
    public static String findName(INode node) throws AnalysisException {
        IsaInvNameFinder finder = new IsaInvNameFinder();
        return node.apply(finder);
    }

    @Override
    public String caseANamedTypeDeclIR(ANamedTypeDeclIR node) throws AnalysisException {
        return node.getName().getName();
    }
    
    @Override
    public String caseANotImplementedExpIR(ANotImplementedExpIR node) {
    return "True";
      
    }
    
    @Override
    public String caseAStateDeclIR(AStateDeclIR node) throws AnalysisException {
        return node.getName();
    }
    
    @Override
    public String caseASetSetTypeIR(ASetSetTypeIR node) throws AnalysisException {
        return "SetElems";
    }
    
    @Override
    public String caseASeqSeqTypeIR(ASeqSeqTypeIR node) throws AnalysisException {
      ANamedTypeDeclIR n = new ANamedTypeDeclIR();
      return "SeqElems";
    }
    @Override
    public String caseANatNumericBasicTypeIR(ANatNumericBasicTypeIR node) throws AnalysisException {
      return "VDMNat";
    }
    
    @Override
    public String caseAIntNumericBasicTypeIR(AIntNumericBasicTypeIR node) throws AnalysisException {
      return "True";
    }
    
    @Override
    public String caseARealNumericBasicTypeIR(ARealNumericBasicTypeIR node) throws AnalysisException {
      return "True";
    }
    
    @Override
    public String caseARatNumericBasicTypeIR(ARatNumericBasicTypeIR node) throws AnalysisException {
      return "True";
    }
    
    
    @Override
    public String caseABoolBasicTypeIR(ABoolBasicTypeIR node) throws AnalysisException {
      return "True";
    }
    
    @Override
    public String caseACharBasicTypeIR(ACharBasicTypeIR node) throws AnalysisException {
      return "True";
    }
    
    @Override
    public String caseAMapMapTypeIR(AMapMapTypeIR node) throws AnalysisException {
      return "True";
    }
    
    @Override
    public String caseATokenBasicTypeIR(ATokenBasicTypeIR node) throws AnalysisException {
      return "True";
    }
    
    @Override
    public String caseANat1NumericBasicTypeIR(ANat1NumericBasicTypeIR node) throws AnalysisException {
      return "VDMNat1";
    }
    
    
    
    
    @Override
    public String caseARecordDeclIR(ARecordDeclIR node) throws AnalysisException {
        return node.getName();
    }

    @Override
    public String createNewReturnValue(INode node) throws AnalysisException {
      String typeName;
        STypeIR n = (STypeIR) node;
        //if not a toolkit or IR node type
      if (n.getNamedInvType() == null) typeName = "True";
      else typeName = n.getNamedInvType().getName().getName();
      return typeName;  
    }

    @Override
    public String createNewReturnValue(Object node) throws AnalysisException {
      String typeName;
        STypeIR n = (STypeIR) node;
        //if not a toolkit or IR node type
      if (n.getNamedInvType() == null) typeName = "True";
      else typeName = n.getNamedInvType().getName().getName();
      return typeName;  
    }
}
\end{lstlisting}

\subsection{IsaFuncDeclConv} \label{IsaFuncDeclConv}
\begin{lstlisting}
public class IsaFuncDeclConv extends DepthFirstAnalysisIsaAdaptor {


    private final AModuleDeclIR vdmToolkitModuleIR;
    private final Map<String, AFuncDeclIR> isaFuncDeclIRMap;
    
    public IsaFuncDeclConv(IRInfo info, TransAssistantIR t, AModuleDeclIR vdmToolkitModuleIR) {
        this.vdmToolkitModuleIR = vdmToolkitModuleIR;

        this.vdmToolkitModuleIR.getDecls()
                .stream()
                .filter(d -> {
                    if (d instanceof ATypeDeclIR)
                        return true;
                    else
                        return false;
                }).map(d -> (ATypeDeclIR) d)
                .collect(Collectors.toMap(x -> ((ANamedTypeDeclIR) x.getDecl()).getName().getName(), x -> x));
        
        this.isaFuncDeclIRMap = this.vdmToolkitModuleIR.getDecls().stream().filter(d ->
        {
            if (d instanceof AFuncDeclIR)
                return true;
            else
                return false;
        }).map(d -> (AFuncDeclIR) d).collect(Collectors.toMap(x -> x.getName(), x -> x));

        
    }
    
   
    // Transform AFuncDeclIR
    @Override
    public void caseAFuncDeclIR(AFuncDeclIR x) throws AnalysisException {
      super.caseAFuncDeclIR(x);
      //we need to stop post conditions of postconditions of post conditions... being formed
      if (!x.getName().contains("inv") && 
          !x.getName().contains("post") && !x.getName().contains("pre"))
      {
        
        if (x.parent() instanceof AStateDeclIR)
        {
          transStateInit(x);
          
        }
        else 
        {
          transformPreConditions(x);
          
          transformPostConditions(x);
          
          // If no parameter function set params to null to make this more concrete for velocity
          if (x.getFormalParams().size() == 0) 
          {
            x.getMethodType().setParams(null);
          }
          
          formatIdentifierPatternVars(x);
          if (x.getImplicit()) removeFromAST(x);
        }
      
      
      
      }
    }
    
    private void transStateInit(AFuncDeclIR node) {
      AStateDeclIR st = node.getAncestor(AStateDeclIR.class);
      
      AMethodTypeIR methodType = new AMethodTypeIR();
      
      
      st.getFields().forEach(f -> methodType.getParams().add(f.getType().clone()));
      methodType.setResult(new ABoolBasicTypeIR());
      
      AFuncDeclIR postInit = new AFuncDeclIR();
      postInit.setMethodType(methodType.clone());     
      postInit.setName("post_"+node.getName());
      
      AApplyExpIR app = new AApplyExpIR();
      AIdentifierVarExpIR root = new AIdentifierVarExpIR();
      root.setName("inv_"+st.getName());
      System.out.println(IsaGen.funcGenHistoryMap.keySet());
      root.setType(IsaGen.funcGenHistoryMap.get("inv_"+st.getName()).getMethodType().clone());
      app.setRoot(root);
      
      AIdentifierVarExpIR arg = new AIdentifierVarExpIR();
      arg.setName(node.getName());
      arg.setType(node.getMethodType().clone());
      app.getArgs().add(arg);
      
      postInit.setBody(app);
        addToAST(postInit, node);

        System.out.println("Post condition has been added");
    
  }


  private void removeFromAST(AFuncDeclIR x) {
      // Insert into AST
        AModuleDeclIR encModule = x.getAncestor(AModuleDeclIR.class);
        if(encModule != null)
        {
            encModule.getDecls().remove(x);
        }
    
  }

    private void addToAST(INode node, INode parent) {
      // Insert into AST
        AModuleDeclIR encModule = parent.getAncestor(AModuleDeclIR.class);
        if(encModule != null)
        {
            encModule.getDecls().add((SDeclIR) node);
        }

    
  }
    

  private void transformPreConditions (AFuncDeclIR node) throws AnalysisException {
      AMethodTypeIR mt = node.getMethodType().clone();
      
      /*The final pre condition that will be populated with a generated pre condition,
      a modeller written pre condition or both or neither.*/
      AFuncDeclIR finalPreCondition = null;
      
      //Generated pre condition will be populated if one can be generated
      AFuncDeclIR generatedPre = null;
      
        // If there are parameters with which to build a pre condition then build one
      if (!mt.getParams().isEmpty())
      {
        generatedPre = createPre(node.clone());     
        //Copy across all generated properties into final pre condition.
        finalPreCondition = generatedPre;
        
      }
        
        // If there are pre written pre conditions and one was generated add them both
      if (node.getPreCond() != null && generatedPre != null)
      {
        AFuncDeclIR preCond_ = (AFuncDeclIR) node.getPreCond();
        
        AAndBoolBinaryExpIR andExisting = new AAndBoolBinaryExpIR();
        andExisting.setLeft(generatedPre.getBody());
        andExisting.setRight(preCond_.getBody());
        finalPreCondition.setBody(andExisting);
      }
      
      //If there is only a pre written pre condition add that
      else if (node.getPreCond() != null && generatedPre == null)
      {
        //Copy across all pre written properties into final pre condition.
        finalPreCondition = new AFuncDeclIR();
        
        //No need to add formal params again they're all already put there above
        AFuncDeclIR preCond_ = (AFuncDeclIR) node.getPreCond();
        finalPreCondition.setBody(preCond_.getBody());
        finalPreCondition.setFormalParams(preCond_.getFormalParams());
        finalPreCondition.setMethodType(preCond_.getMethodType());
        finalPreCondition.setName(preCond_.getName());
      }
       /* If no pre condition is written, none has been generated then
        there are no parameter types to use as invariant checks and no relevant checks provided
        by modeller, so pre condition is added but left empty as a reminder to the modeller to add one      later.*/
      else if (node.getPreCond() == null && generatedPre == null)
      {
        //Copy across all pre written properties into final pre condition.
        finalPreCondition = new AFuncDeclIR();
        ANotImplementedExpIR n = new ANotImplementedExpIR();
        n.setTag("TODO");
        finalPreCondition.setBody(n);
        // Set up method type for post condition
          AMethodTypeIR mty = new AMethodTypeIR();
          mty.setResult(new ABoolBasicTypeIR());
      mty.setParams(null);
          finalPreCondition.setMethodType(mty);
        finalPreCondition.setName("unimplemented_pre_"+node.getName());
      }
    
      formatIdentifierPatternVars(finalPreCondition);
      node.setPreCond(finalPreCondition);
      IsaGen.funcGenHistoryMap.put(finalPreCondition.getName(), finalPreCondition.clone());
      addToAST(finalPreCondition, node);
      
      System.out.println("Pre condition has been added");
   
    }
    
    

    
    private void transformPostConditions (AFuncDeclIR node) throws AnalysisException {
      AMethodTypeIR mt = node.getMethodType().clone();
      
      /*The final post condition that will be populated with a generated post condition,
      a modeller written post condition or both or neither.*/
      AFuncDeclIR finalPostCondition = null;
      
      //Generated post condition will be populated if one can be generated
      AFuncDeclIR generatedPost = null;
      
        // If there are parameters and results with which to build a post condition then build one
      if (!mt.getParams().isEmpty() && mt.getResult() != null)
      {
        generatedPost = createPost(node.clone()); 
        //Copy across all generated properties into final post condition.
          finalPostCondition = generatedPost;
          
      }
      
      // If there are pre written post conditions and one was generated add them both
        if (node.getPostCond() != null && generatedPost != null)
        {
          AFuncDeclIR postCond_ = (AFuncDeclIR) node.getPostCond();   
          
          
          AAndBoolBinaryExpIR andExisting = new AAndBoolBinaryExpIR();
          andExisting.setLeft(generatedPost.getBody());
          andExisting.setRight(postCond_.getBody());
          finalPostCondition.setBody(andExisting);
          
        }
        
        //If there is only a pre written post condition add that
        else if (node.getPostCond() != null && generatedPost == null)
        {
          //Copy across all pre written properties into final post condition.
          finalPostCondition = new AFuncDeclIR();
          
          //No need to add formal params again they're all already put there above
          AFuncDeclIR postCond_ = (AFuncDeclIR) node.getPostCond();
          finalPostCondition.setBody(postCond_.getBody());
          finalPostCondition.setFormalParams(postCond_.getFormalParams());
          finalPostCondition.setMethodType(postCond_.getMethodType());
          finalPostCondition.setName(postCond_.getName());
        }
       /* If no post condition is written, none has been generated then
        there are no parameter types to use as invariant checks and no relevant checks provided
        by modeller, so post condition is added but left empty as a reminder to the modeller to add one later.*/
        else if (node.getPostCond() == null && generatedPost == null)
        {
          //Copy across all pre written properties into final post condition.
          finalPostCondition = new AFuncDeclIR();
          ANotImplementedExpIR n = new ANotImplementedExpIR();
          n.setTag("TODO");
          finalPostCondition.setBody(n);
       
          AMethodTypeIR mty = new AMethodTypeIR();
          mty.setResult(new ABoolBasicTypeIR());
      mty.getParams().add(mt.getResult());
          finalPostCondition.setMethodType(mty);
          finalPostCondition.setName("unimplemented_post_"+node.getName());
        }
      
        formatIdentifierPatternVars(finalPostCondition);
        node.setPostCond(finalPostCondition);
      IsaGen.funcGenHistoryMap.put(finalPostCondition.getName(), finalPostCondition.clone());
        addToAST(finalPostCondition, node);

        System.out.println("Post condition has been added");
    }
    
  
    


    private AFuncDeclIR createPre(AFuncDeclIR node) throws AnalysisException {
      // Post condition function
        AFuncDeclIR preCond = new AFuncDeclIR();
      AMethodTypeIR mt = node.getMethodType();
        SExpIR expr;
        
       // Set post_[function name] as post function name
      preCond.setName("pre_" + node.getName()); 
        
      // Set up method type for post condition
        AMethodTypeIR type = new AMethodTypeIR();
        type.setResult(new ABoolBasicTypeIR());
    type.setParams(mt.getParams());
        preCond.setMethodType(type);
        
        
        
        AIdentifierPatternIR identifierPattern = new AIdentifierPatternIR();
        identifierPattern.setName("");
        if (node.getFormalParams() != null && !node.getFormalParams().isEmpty())
        {
          // Loop through all but result type
          for (int i = 0; i < preCond.getMethodType().getParams().size(); i++)
          {
            identifierPattern = new AIdentifierPatternIR();
              identifierPattern.setName(node.getFormalParams().get(i).getPattern().toString());
            AFormalParamLocalParamIR afp = new AFormalParamLocalParamIR();
            afp.setPattern(identifierPattern);
            afp.setType(preCond.getMethodType().getParams().get(i).clone()); 
            preCond.getFormalParams().add(afp);
          }
        }
    expr = IsaInvExpGen.apply(preCond.clone(), identifierPattern, preCond.getMethodType().clone(), isaFuncDeclIRMap);
      preCond.setBody(expr);
        return preCond;
    }
    
    
  private AFuncDeclIR createPost(AFuncDeclIR node) throws AnalysisException {
      // Post condition function
        AFuncDeclIR postCond = new AFuncDeclIR();
      AMethodTypeIR mt = node.getMethodType();
        SExpIR expr;
        
       // Set post_[function name] as post function name
      postCond.setName("post_" + node.getName()); 
        
      // Set up method type for post condition
        AMethodTypeIR type = new AMethodTypeIR();
        type.setResult(new ABoolBasicTypeIR());
        List<STypeIR> params = mt.getParams();
        params.add(mt.getResult().clone());
    type.setParams(params);
        postCond.setMethodType(type);
        
        
        
        AIdentifierPatternIR identifierPattern = new AIdentifierPatternIR();
        identifierPattern.setName("");
        if (node.getFormalParams() != null && !node.getFormalParams().isEmpty())
        {
          // Loop through all but result type
          for (int i = 0; i < postCond.getMethodType().getParams().size() -1; i++)
          {
            identifierPattern = new AIdentifierPatternIR();
              identifierPattern.setName(node.getFormalParams().get(i).getPattern().toString());
            AFormalParamLocalParamIR afp = new AFormalParamLocalParamIR();
            afp.setPattern(identifierPattern);
            afp.setType(postCond.getMethodType().getParams().get(i).clone()); 
            postCond.getFormalParams().add(afp);
          }
        }
        
        // Add RESULT pattern if the function has a result
        if (mt.getResult() != null)
        {
          identifierPattern = new AIdentifierPatternIR();
          if (node.getPostCond() != null)
            identifierPattern.setName(((AFuncDeclIR) 
                node.getPostCond()).getFormalParams().getLast().getPattern().toString());
          else
            identifierPattern.setName("RESULT");
          AFormalParamLocalParamIR afp = new AFormalParamLocalParamIR();
          afp.setPattern(identifierPattern);
          afp.setType(mt.getResult()); 
          postCond.getFormalParams().add(afp);
        }
    //an and expression of all of the parameter invariants do nothing if the body is not implemented
        expr = IsaInvExpGen.apply(postCond.clone(), identifierPattern, postCond.getMethodType().clone(), isaFuncDeclIRMap);
        postCond.setBody(expr);
        
        
        return postCond;
  }
    
    
   /*space out identifier variables, e.g. for two variable t and x we should have t x not tx. 
      "inv_t t x \<equiv> isa_invTrue t \<and> isa_invTrue x"*/
    private AFuncDeclIR formatIdentifierPatternVars (AFuncDeclIR node) {
      /*  This puts a space between different parameters in the Isabelle function body
      , xy is misinterpreted as one variable whereas x y is correctly interpreted as two
      */
      node.getFormalParams().forEach
      (   
          p -> { 
            
            AIdentifierPatternIR ip = new AIdentifierPatternIR();
            ip.setName(p.getPattern().toString() + " ");
            p.setPattern(ip);
            
          }
      );
      
      return node;
    }
    
}
\end{lstlisting}

\subsection{IsaBasicTypesConv} \label{IsaBasicTypesConvafter}
\begin{lstlisting}
/***
 * Visitor to convert sequence or set VDM types to VDMToolkit types
 */
public class IsaTypeTypesConv extends DepthFirstAnalysisIsaAdaptor {

    private final Map<String, ATypeDeclIR> isaTypeDeclIRMap;
    private final TransAssistantIR t;
    private final AModuleDeclIR vdmToolkitModuleIR;
    private final IRInfo info;

    private final static String isa_VDMSet = "isa_VDMSet";

    private final static String isa_VDMSeq = "isa_VDMSeq";

    public IsaTypeTypesConv(IRInfo info, TransAssistantIR t, AModuleDeclIR vdmToolkitModuleIR) {
        this.t = t;
        this.info = info;
        this.vdmToolkitModuleIR = vdmToolkitModuleIR;

        this.isaTypeDeclIRMap = this.vdmToolkitModuleIR.getDecls()
                .stream()
                .filter(d -> {
                    if (d instanceof ATypeDeclIR)
                        return true;
                    else
                        return false;
                }).map(d -> (ATypeDeclIR) d)
                .collect(Collectors.toMap(x -> ((ANamedTypeDeclIR) x.getDecl()).getName().getName(), x -> x));
    }
    
   //transform seq into VDMSeq
    public void caseASeqSeqTypeIR(ASeqSeqTypeIR x) {
      if(x.getNamedInvType() == null)
        {
            
            // Retrieve isa_VDMSeq from VDMToolkit
            ATypeDeclIR isa_td = isaTypeDeclIRMap.get(IsaTypeTypesConv.isa_VDMSeq);

            x.setNamedInvType((ANamedTypeDeclIR)isa_td.getDecl().clone());
            
        }
    }
  //transform set into VDMSet
    public void caseASetSetTypeIR(ASetSetTypeIR x) {
      if(x.getNamedInvType() == null)
        {
            
            // Retrieve isa_VDMSet from VDMToolkit
            ATypeDeclIR isa_td = isaTypeDeclIRMap.get(IsaTypeTypesConv.isa_VDMSet);

            x.setNamedInvType((ANamedTypeDeclIR)isa_td.getDecl().clone());
        }
    }
    
    
    
    
}
\end{lstlisting}
\subsection{IsaTypeTypesConv} \label{IsaTypeTypesConv}

\subsection{IsaInvExpGen} \label{IsaInvExpGenafter}
\begin{lstlisting}
/*
Generates the expression for an invariant.
Example:
    VDM spec:   types
                test = nat
    Invariant expression: isa_inv_VDMNat i
    where i is a parameter to this visitor.

 */
public class IsaInvExpGen extends AnswerIsaAdaptor<SExpIR> {

    AIdentifierPatternIR ps;
    AMethodTypeIR methodType;
    
    private final Map<String, AFuncDeclIR> isaFuncDeclIRMap;
  private AIdentifierVarExpIR targetIP;
  private final LinkedList<ANamedTypeDeclIR> invArr = new LinkedList<ANamedTypeDeclIR>();


    public IsaInvExpGen(AIdentifierPatternIR ps, AMethodTypeIR methodType, Map<String, AFuncDeclIR> isaFuncDeclIRMap)
    {
        this.ps = ps;
        this.methodType = methodType;
        this.isaFuncDeclIRMap = isaFuncDeclIRMap;
    }

    public static SExpIR apply(SDeclIR decl, AIdentifierPatternIR afp, AMethodTypeIR methodType, Map<String, AFuncDeclIR> isaFuncDeclIRMap) throws AnalysisException {
        IsaInvExpGen finder = new IsaInvExpGen(afp, methodType, isaFuncDeclIRMap);
        return decl.apply(finder);
    }

    @Override
    public SExpIR caseANamedTypeDeclIR(ANamedTypeDeclIR node) throws AnalysisException {
        node.getType();
        //TODO make for different types invariants
        // Find invariant function
        AFuncDeclIR fInv = this.isaFuncDeclIRMap.get("isa_invTrue");
        // Create ref to function
        AIdentifierVarExpIR fInvIdentifier = new AIdentifierVarExpIR();
        fInvIdentifier.setName(fInv.getName());
        fInvIdentifier.setSourceNode(fInv.getSourceNode());
        fInvIdentifier.setType(fInv.getMethodType());

        // Crete apply expr
        AApplyExpIR exp = new AApplyExpIR();
        exp.setType(new ABoolBasicTypeIR());
        AIdentifierVarExpIR iVarExp = new AIdentifierVarExpIR();
        iVarExp.setName(this.ps.getName());
        iVarExp.setType(this.methodType);
        exp.getArgs().add(iVarExp);
        exp.setRoot(fInvIdentifier);

        return exp;
    }

    
    @Override
    public SExpIR caseAStateDeclIR(AStateDeclIR node) throws AnalysisException {
      //TODO e.g. where "inv_recType r \<equiv> isa_invVDMSeq isa_invVDMNat1 (x r) etc. 
      LinkedList<AFieldDeclIR> fields = new LinkedList<AFieldDeclIR>();
    node.getFields().forEach(f -> fields.add(f.clone()));
        AApplyExpIR completeExp = new AApplyExpIR();
        LinkedList<AApplyExpIR> fieldInvariants = new LinkedList<AApplyExpIR>();
        
        for (int i = 0; i < fields.size(); i++) 
          {
            STypeIR type = fields.get(i).getType();
          AIdentifierVarExpIR invExp = new AIdentifierVarExpIR();
              invExp.setName("("+node.getName().substring(0,1).toLowerCase()+
                  node.getName().toString().substring(1, node.getName().toString().length())+"_"+
                  fields.get(i).getName()+" "+this.ps.toString()+")");
              invExp.setType(this.methodType);
              this.targetIP = invExp;
        
              completeExp.setType(new ABoolBasicTypeIR());
              //Recursively build curried inv function e.g.  (inv_VDMSet (inv_VDMSet inv_Nat1)) inv_x
             
              try {
          fieldInvariants.add(buildInvForType(type.clone()));
        } catch (AnalysisException e) {
          e.printStackTrace();
        }
          
          }
      
     // Link numerous apply expressions together in an and expression
        if (fieldInvariants.size() >= 2)
          return genAnd(fieldInvariants);
        else
        // Just one field return it as an apply expression
          return fieldInvariants.get(0);
    }
    
    
    
    
    
    
    @Override
    public SExpIR caseARecordDeclIR(ARecordDeclIR node) throws AnalysisException {
      //TODO e.g. where "inv_recType r \<equiv> isa_invVDMSeq isa_invVDMNat1 (x r) etc. 
      LinkedList<AFieldDeclIR> fields = new LinkedList<AFieldDeclIR>();
    node.getFields().forEach(f -> fields.add(f.clone()));
        AApplyExpIR completeExp = new AApplyExpIR();
        LinkedList<AApplyExpIR> fieldInvariants = new LinkedList<AApplyExpIR>();
        
        for (int i = 0; i < fields.size(); i++) 
          {
            STypeIR type = fields.get(i).getType();
          AIdentifierVarExpIR invExp = new AIdentifierVarExpIR();
              invExp.setName("("+node.getName().substring(0,1).toLowerCase()+
                  node.getName().toString().substring(1, node.getName().toString().length())+"_"+
                  fields.get(i).getName()+" "+this.ps.toString()+")");
              invExp.setType(this.methodType);
              this.targetIP = invExp;
        
              completeExp.setType(new ABoolBasicTypeIR());
              //Recursively build curried inv function e.g.  (inv_VDMSet (inv_VDMSet inv_Nat1)) inv_x
             
              try {
          fieldInvariants.add(buildInvForType(type.clone()));
        } catch (AnalysisException e) {
          e.printStackTrace();
        }
          
          }
      
     // Link numerous apply expressions together in an and expression
        if (fieldInvariants.size() >= 2)
          return genAnd(fieldInvariants);
        else
        // Just one field return it as an apply expression
          return fieldInvariants.get(0);
    }
    
    
    
    @Override
    public SExpIR caseAFieldDeclIR(AFieldDeclIR node) throws AnalysisException {
        STypeIR t = node.getType().clone();
        AApplyExpIR completeExp = new AApplyExpIR();
        // Crete apply to the inv_ expr e.g inv_x inv_y
        AIdentifierVarExpIR invExp = new AIdentifierVarExpIR();
        invExp.setName(node.getName());
        invExp.setType(this.methodType);
        this.targetIP = invExp;
    
        completeExp.setType(new ABoolBasicTypeIR());
        //Recursively build curried inv function e.g.  (inv_VDMSet (inv_VDMSet inv_Nat1)) inv_x
       
    completeExp = buildInvForType(t);
      
      
      
    return completeExp;
    }
    
    
    @Override
    public SExpIR caseAFuncDeclIR(AFuncDeclIR node) throws AnalysisException {
        LinkedList<AFormalParamLocalParamIR> t = node.getFormalParams();
        node.setMethodType(this.methodType);
        LinkedList<AApplyExpIR> paramInvariants = new LinkedList<AApplyExpIR>();
        
        for (int i = 0; i < node.getFormalParams().size(); i++) {
          STypeIR type = t.get(i).getType();      
          AApplyExpIR completeExp = new AApplyExpIR();
          
          // Create apply to the inv_ expr e.g inv_x inv_y
            AIdentifierVarExpIR invExp = new AIdentifierVarExpIR();
            invExp.setName(node.getFormalParams().get(i).getPattern().toString());
            invExp.setType(this.methodType);
            this.targetIP = invExp;
      
            completeExp.setType(new ABoolBasicTypeIR());
            //Recursively build curried inv function e.g.  (inv_VDMSet (inv_VDMSet inv_Nat1)) inv_x
           
      try {
        completeExp = buildInvForType(type);
      } catch (AnalysisException e) {
        e.printStackTrace();
      }
      
      paramInvariants.add(completeExp);
              
            
        }
        
        
        // Link numerous apply expressions together in an and expression
        if (paramInvariants.size() >= 2)
          return genAnd(paramInvariants);
        else
        // Just one parameter return it as an apply expression
          return paramInvariants.get(0);
        
        
      
    }
     
    
    private SExpIR genAnd(LinkedList<AApplyExpIR> paramInvariants) {
      
      AAndBoolBinaryExpIR and = new AAndBoolBinaryExpIR();
      
      //base case
    if (paramInvariants.size() == 2)
        {
        and.setLeft(paramInvariants.get(0));
        and.setRight(paramInvariants.get(1));
        }
    else
      {
        and.setLeft(paramInvariants.get(0));
        paramInvariants.remove(0);
        and.setRight( genAnd(paramInvariants) );
      }
    return and;
    
  }

  //build curried invariant
    public AApplyExpIR buildInvForType(STypeIR seqtNode) throws AnalysisException {
      
      String typeName = IsaInvNameFinder.findName(seqtNode);
      
      AFuncDeclIR fInv;
      if (this.isaFuncDeclIRMap.get("isa_inv"+typeName) != null)
      {
        fInv = this.isaFuncDeclIRMap.get("isa_inv"+typeName).clone();
      }
      else
      {
        fInv = IsaGen.funcGenHistoryMap.get("inv_"+typeName).clone();
        
      }
      if (fInv.getMethodType() == null)
      {
        AMethodTypeIR mt = new AMethodTypeIR();
        mt.setResult(new ABoolBasicTypeIR());
        mt.getParams().add(seqtNode);
        fInv.setMethodType(mt.clone());
      }
      
         // Create ref to function
        AIdentifierVarExpIR curriedInv = new AIdentifierVarExpIR();
        curriedInv.setName(fInv.getName());
        curriedInv.setSourceNode(fInv.getSourceNode());
        curriedInv.setType(fInv.getMethodType().clone());//Must always clone
      AApplyExpIR accum = new AApplyExpIR();
      accum.setRoot(curriedInv);
      
      
      //if this type is not the last in the nested types, then keep rescursing until we get to the final nested type
      if ( seqtNode instanceof ASetSetTypeIR && ((ASetSetTypeIR) seqtNode).getSetOf() != null )
      {
        accum.getArgs().add(buildInvForType(((ASetSetTypeIR) seqtNode).getSetOf().clone()));
      }
      else if (seqtNode instanceof ASeqSeqTypeIR && ((ASeqSeqTypeIR) seqtNode).getSeqOf() != null)
      {
        
        accum.getArgs().add(buildInvForType(((ASeqSeqTypeIR) seqtNode).getSeqOf().clone()));
      }
      else
      {
        accum.getArgs().add(targetIP);
      }
      return accum;
        
  }


    @Override
    public SExpIR createNewReturnValue(INode node) throws AnalysisException {
        return null;
    }

    @Override
    public SExpIR createNewReturnValue(Object node) throws AnalysisException {
            return null;
    }

    public SExpIR caseATokenBasicTypeIR(ATokenBasicTypeIR n) throws AnalysisException
    {

        new AApplyExp();

        throw new AnalysisException();

    }


    public SExpIR caseASetSetTypeIR(ASetSetTypeIR node) throws AnalysisException {
        throw new AnalysisException();

    }


}
\end{lstlisting}

\subsection{IsaDeclTypeGen} \label{IsaDeclTypeGenafter}
\begin{lstlisting}
public class IsaDeclTypeGen extends AnswerIsaAdaptor<STypeIR> {

    public static STypeIR apply(INode node) throws AnalysisException {
        IsaDeclTypeGen finder = new IsaDeclTypeGen();
        return node.apply(finder);
    }

    public STypeIR caseANamedTypeDeclIR(ANamedTypeDeclIR n)
    {
      IsaGen.typeGenHistoryMap.put(n.getType(), n.getName().toString());
        AIntNumericBasicTypeIR a = new AIntNumericBasicTypeIR();
        a.setNamedInvType(n.clone());
        return a;
    }

    public STypeIR caseAStateDeclIR(AStateDeclIR n)
    {
      ARecordTypeIR a = new ARecordTypeIR();
      ATypeNameIR o = new ATypeNameIR();
      o.setName(n.getName());
      a.setName(o);
        return a;
      
    }
    
    public STypeIR caseARecordDeclIR(ARecordDeclIR n)
    {
      ARecordTypeIR a = new ARecordTypeIR();
      ATypeNameIR o = new ATypeNameIR();
      o.setName(n.getName());
      a.setName(o);
        return a;
      
    }

    @Override
    public STypeIR createNewReturnValue(INode node) throws AnalysisException {
        return null;
    }

    @Override
    public STypeIR createNewReturnValue(Object node) throws AnalysisException {
        return null;
    }
}
\end{lstlisting}

\subsection{IsaInvGenTrans} \label{IsaInvGenTransafter}
\begin{lstlisting}
public class IsaInvGenTrans extends DepthFirstAnalysisIsaAdaptor {

    private final AModuleDeclIR vdmToolkitModule;
    private final Map<String, ATypeDeclIR> isaTypeDeclIRMap;
    private IRInfo info;
    private final Map<String, AFuncDeclIR> isaFuncDeclIRMap;
    
    public IsaInvGenTrans(IRInfo info, AModuleDeclIR vdmToolkitModuleIR) {
        this.info = info;
        this.vdmToolkitModule = vdmToolkitModuleIR;

        this.isaFuncDeclIRMap = this.vdmToolkitModule.getDecls().stream().filter(d ->
        {
            if (d instanceof AFuncDeclIR)
                return true;
            else
                return false;
        }).map(d -> (AFuncDeclIR) d).collect(Collectors.toMap(x -> x.getName(), x -> x));

        this.isaTypeDeclIRMap = this.vdmToolkitModule.getDecls().stream().filter(d -> {
            if (d instanceof ATypeDeclIR)
                return true;
            else
                return false;
        }).map(d -> (ATypeDeclIR) d).collect(Collectors.toMap(x -> ((ANamedTypeDeclIR) x.getDecl()).getName().getName(), x -> x));


    }

    
    @Override
    public void caseAStateDeclIR(AStateDeclIR node) throws AnalysisException {
      super.caseAStateDeclIR(node);
      
      SDeclIR decl = node.clone();
      String typeName = IsaInvNameFinder.findName(node.clone());
      SExpIR invExp = node.getInvExp();
        // Invariant function
        AFuncDeclIR invFun_ = new AFuncDeclIR();
        invFun_.setName("inv_" + typeName); //inv_t

        AMethodTypeIR methodType = new AMethodTypeIR();
        
        STypeIR t = IsaDeclTypeGen.apply(decl.clone());
        methodType.getParams().add(t.clone());
        
          
      methodType.setResult(new ABoolBasicTypeIR());
        invFun_.setMethodType(methodType);
         
          
          
  
        // Translation for VDMToolkit and modeller written invariants
        if (invExp != null)
        {
          AAndBoolBinaryExpIR multipleInvs = new AAndBoolBinaryExpIR();
          //change (a_c a) to (c A) for Isabelle field access
            //if (decl instanceof ARecordDeclIR) formatExistingRecordInvExp(inv.getBody());
          
            multipleInvs.setRight(invExp);
          
      AIdentifierPatternIR identifierPattern = new AIdentifierPatternIR();
      identifierPattern.setName(typeName.substring(0, 1).toLowerCase());
      
      //set Inv pattern if one does not exist
      if (node.getInvPattern() != null) node.setInvPattern(identifierPattern);
      
      SExpIR expr = IsaInvExpGen.apply(decl, 
          identifierPattern , 
          methodType.clone(), isaFuncDeclIRMap);
          
      multipleInvs.setLeft(expr);
          
          invFun_.setBody(multipleInvs);
          node.setInvExp(multipleInvs);
        } 
        //translation for no inv types 
        else 
        {
        
          SExpIR expr;
      AIdentifierPatternIR identifierPattern = new AIdentifierPatternIR();
          identifierPattern.setName(typeName.substring(0, 1).toLowerCase());
          AFormalParamLocalParamIR afp = new AFormalParamLocalParamIR();
          afp.setPattern(identifierPattern.clone());
          afp.setType(t.clone()); 
          
          node.setInvPattern(identifierPattern);
          
          invFun_.getFormalParams().add(afp);
          expr = IsaInvExpGen.apply(decl.clone(), identifierPattern, methodType.clone(), isaFuncDeclIRMap);
          
          
          invFun_.setBody(expr.clone());
          node.setInvExp(expr);
        }
        node.setInvDecl(invFun_.clone());
        
        IsaGen.funcGenHistoryMap.put(invFun_.getName(), invFun_.clone());
        System.out.println("");
        }
        
    
    
    
    
  
    @Override
    public void caseATypeDeclIR(ATypeDeclIR node) throws AnalysisException {
        super.caseATypeDeclIR(node);
        
        /*We do not want invariants built for each type declaration field
        instead we would like one invariant for the whole declaration type
        we skip subsequent record fields so that we do not get
        inv_field1 inv_field2 inv_record instead we get inv_record which accesses
        field1 and field2.*/
        
        String typeName = IsaInvNameFinder.findName(node.getDecl());
        SDeclIR decl = node.getDecl().clone();
         
        SDeclIR invFun;
        if (node.getDecl() instanceof ARecordDeclIR)
          invFun = ( (ARecordDeclIR) decl).getInvariant();
        else
          invFun = node.getInv();
        // Invariant function
        AFuncDeclIR invFun_ = new AFuncDeclIR();
        invFun_.setName("inv_" + typeName); //inv_t

        // Define the type signature
        //TODO: Type should be XTypeInt - correct?
        AMethodTypeIR methodType = new AMethodTypeIR();
        
        STypeIR t = IsaDeclTypeGen.apply(decl);
        methodType.getParams().add(t.clone());
        
          
      methodType.setResult(new ABoolBasicTypeIR());
        invFun_.setMethodType(methodType);
         
          
          
  
        // Translation for VDMToolkit and modeller written invariants
        if (invFun != null)
        {
          AFuncDeclIR inv = (AFuncDeclIR) invFun;//cast invariant function declaration to AFuncDeclIR
          AAndBoolBinaryExpIR multipleInvs = new AAndBoolBinaryExpIR();
          
          for (int i = 0; i < inv.getMethodType().getParams().size(); i++)
          {
            AFormalParamLocalParamIR afplp = new AFormalParamLocalParamIR();
              afplp.setPattern(inv.getFormalParams().get(i).getPattern());
              afplp.setType(inv.getMethodType().getParams().get(i).clone());
              invFun_.getFormalParams().add(afplp);
          }
          
          //change (a_c a) to (c A) for Isabelle field access
            //if (decl instanceof ARecordDeclIR) formatExistingRecordInvExp(inv.getBody());
          
            multipleInvs.setRight(inv.getBody());
          
      AIdentifierPatternIR identifierPattern = new AIdentifierPatternIR();
      identifierPattern.setName(typeName.substring(0, 1).toLowerCase());
      SExpIR expr = IsaInvExpGen.apply(decl.clone(), 
          identifierPattern , 
          methodType.clone(), isaFuncDeclIRMap);
          
      multipleInvs.setLeft(expr);
          
          invFun_.setBody(multipleInvs);
        } 
        //translation for no inv types 
        else 
        {
          SExpIR expr;
      AIdentifierPatternIR identifierPattern = new AIdentifierPatternIR();
          identifierPattern.setName(typeName.substring(0, 1).toLowerCase());
          AFormalParamLocalParamIR afp = new AFormalParamLocalParamIR();
          afp.setPattern(identifierPattern);
          afp.setType(t.clone()); 
          invFun_.getFormalParams().add(afp);
          expr = IsaInvExpGen.apply(decl.clone(), identifierPattern, methodType.clone(), isaFuncDeclIRMap);
          
          
          invFun_.setBody(expr);
        }
        

        // Insert into AST and get rid of existing invariant functions forEach field in record type
        AModuleDeclIR encModule = node.getAncestor(AModuleDeclIR.class);
        if (decl instanceof ARecordDeclIR) encModule.getDecls().removeIf(
            d -> d instanceof AFuncDeclIR && d.getChildren(true).get("_name").toString().contains("inv"));
        
        if(encModule != null)
        {
          
            encModule.getDecls().add(invFun_);
        }

        IsaGen.funcGenHistoryMap.put(invFun_.getName(), invFun_.clone());
        
        System.out.println("");

        
    }
    
    @Override
    public void caseAFieldDeclIR(AFieldDeclIR node) throws AnalysisException {
        super.caseAFieldDeclIR(node);
        if (node.parent() instanceof AStateDeclIR){
          System.out.println("Redirecting State Invariants...");
        }
        else {
        STypeIR t = node.getType();// Invariant function
        AFuncDeclIR invFun_ = new AFuncDeclIR();
        invFun_.setName("inv_" + node.getName());
        
        AMethodTypeIR mt = new AMethodTypeIR();
        
      mt.setResult(new ABoolBasicTypeIR()); //set return type to bool
        invFun_.setMethodType(mt.clone());
      
        

        AIdentifierPatternIR identifierPattern = new AIdentifierPatternIR();
        identifierPattern.setName("");//abbreviations have no params so do not use identifier pattern
        
        
        AFormalParamLocalParamIR afp = new AFormalParamLocalParamIR();
        afp.setPattern(identifierPattern);
        afp.setType(t.clone()); 
        invFun_.getFormalParams().add(afp);
      
        
        SExpIR expr = IsaInvExpGen.apply(node, identifierPattern, mt.clone(), isaFuncDeclIRMap);
        
      invFun_.setBody(expr);
      IsaGen.funcGenHistoryMap.put(invFun_.getName(), invFun_);
        // Insert into AST
        AModuleDeclIR encModule = node.getAncestor(AModuleDeclIR.class);
        if(encModule != null)
        {
            encModule.getDecls().add(invFun_.clone());
        }
        System.out.println("");
        }
    }
    
   

    public String GenInvTypeDefinition(String arg){
        return "Definition\n" +
                "   inv_" + arg+ " :: \"" + arg + " \\<Rightarrow> \\<bool>\"\n" +
                "   where\n" +
                "";
    }

}
\end{lstlisting}
\section{CodeGen Platform}
\subsection{IsaGen} \label{IsaGenafter}
\begin{lstlisting}

/**
 * Main facade class for VDM 2 Isabelle IR
 *
 * @author ldc
 */
public class IsaGen extends CodeGenBase {


  public static Map<String, AFuncDeclIR> funcGenHistoryMap = new HashMap<>();;
  public static Map<STypeIR, String> typeGenHistoryMap = new HashMap<>();;
  
    public IsaGen()
    {
        this.addInvTrueMacro();

        this.getSettings().setAddStateInvToModule(false);
        this.getSettings().setGenerateInvariants(true);
    }
    //TODO: Auto load files in macro directory
    public static void addInvTrueMacro(){
        StringBuilder sb = new StringBuilder("#macro ( invTrue $node )\n" +
                "    definition\n" +
                "        inv_$node.Name :: $node.Name \\<Rightarrow> \\<bool>\n" +
                "        where\n" +
                "        \"inv_$node.Name \\<equiv> inv_True\"\n" +
                "#end");
        addMacro("invTrue",new StringReader(sb.toString()));
        Template template = new Template();
    }

    public static void addMacro(String name, StringReader reader){
        try {
            Template template = new Template();
            RuntimeServices runtimeServices = RuntimeSingleton.getRuntimeServices();

            SimpleNode simpleNode = runtimeServices.parse(reader, name);
            template.setRuntimeServices(runtimeServices);
            template.setData(simpleNode);
            template.initDocument();
        } catch (ParseException e)
        {
            System.out.println("Failed with: " + e);
        }
    }

    public static String vdmExp2IsaString(PExp exp) throws AnalysisException,
            org.overture.codegen.ir.analysis.AnalysisException {
        IsaGen ig = new IsaGen();
        GeneratedModule r = ig.generateIsabelleSyntax(exp);
        if (r.hasMergeErrors()) {
            throw new org.overture.codegen.ir.analysis.AnalysisException(exp.toString()
                    + " cannot be generated. Merge errors:"
                    + r.getMergeErrors().toString());
        }
        if (r.hasUnsupportedIrNodes()) {
            throw new org.overture.codegen.ir.analysis.AnalysisException(exp.toString()
                    + " cannot be generated. Unsupported in IR:"
                    + r.getUnsupportedInIr().toString());
        }
        if (r.hasUnsupportedTargLangNodes()) {
            throw new org.overture.codegen.ir.analysis.AnalysisException(exp.toString()
                    + " cannot be generated. Unsupported in TargLang:"
                    + r.getUnsupportedInTargLang().toString());
        }

        return r.getContent();
    }


    /**
     * Main entry point into the Isabelle Translator component. Takes an AST and returns corresponding Isabelle Syntax.
     *
     * @param statuses The IR statuses holding the nodes to be code generated.
     * @return The generated Isabelle syntax
     * @throws AnalysisException
     *
     */
    @Override
    protected GeneratedData genVdmToTargetLang(List<IRStatus<PIR>> statuses) throws AnalysisException {
      
      
        // Typecheck the VDMToolkit module and generate the IR
        TypeCheckerUtil.TypeCheckResult<List<AModuleModules>> listTypeCheckResult1 =
                TypeCheckerUtil.typeCheckSl(new File("src/test/resources/VDMToolkit.vdmsl"));
        AModuleModules isaToolkit = listTypeCheckResult1.result.
                stream().
                filter(mod -> mod.getName().getName().equals("VDMToolkit")).
                findAny().
                orElseThrow(() -> new AnalysisException("Failed to find VDMToolkit module"));
        super.genIrStatus(statuses, isaToolkit);

        // Get the VDMToolkit module IR
        IRStatus<PIR> vdmToolkitIR = statuses.stream().filter(x -> x.getIrNodeName().equals("VDMToolkit")).findAny().orElseThrow(() -> new AnalysisException("Failed to find VDMToolkit IR node"));
        AModuleDeclIR vdmToolkitModuleIR = (AModuleDeclIR) vdmToolkitIR.getIrNode();


        GeneratedData r = new GeneratedData();
        try {


            // Apply transformations
            for (IRStatus<PIR> status : statuses) {
                if(status.getIrNodeName().equals("VDMToolkit")){
                    System.out.println("");
                } else {


                    // transform away any recursion cycles
                    GroupMutRecs groupMR = new GroupMutRecs();
                    generator.applyTotalTransformation(status, groupMR);
                    
                    if (status.getIrNode() instanceof AModuleDeclIR) {
                      
                        AModuleDeclIR cClass = (AModuleDeclIR) status.getIrNode();
                        
                        // then sort remaining dependencies
                        SortDependencies sortTrans = new SortDependencies(cClass.getDecls());
                        generator.applyPartialTransformation(status, sortTrans);
                    }
                    
                    
                    // Transform all token types to isa_VDMToken
                    // Transform all nat types to isa_VDMNat
                    // Transform all nat1 types to isa_VDMNat
                    // Transform all int types to isa_VDMInt
                    IsaBasicTypesConv invConv = new IsaBasicTypesConv(getInfo(), this.transAssistant, vdmToolkitModuleIR);
                    generator.applyPartialTransformation(status, invConv);
                    
                    
                    // Transform Seq and Set types into isa_VDMSeq and isa_VDMSet
                    IsaTypeTypesConv invSSConv = new IsaTypeTypesConv(getInfo(), this.transAssistant, vdmToolkitModuleIR);
                    generator.applyPartialTransformation(status, invSSConv);
                    
                    
                    IsaInvGenTrans invTrans = new IsaInvGenTrans(getInfo(), vdmToolkitModuleIR);
                    generator.applyPartialTransformation(status, invTrans);
                    
                    IsaFuncDeclConv funcConv = new IsaFuncDeclConv(getInfo(), this.transAssistant, vdmToolkitModuleIR);
                    generator.applyPartialTransformation(status, funcConv);
                    
                    
                    
                }
            }

            r.setClasses(prettyPrint(statuses));
        } catch (org.overture.codegen.ir.analysis.AnalysisException e) {
            throw new AnalysisException(e);
        }
        return r;

    }

    public GeneratedModule generateIsabelleSyntax(PExp exp)
            throws AnalysisException,
            org.overture.codegen.ir.analysis.AnalysisException {
        IRStatus<SExpIR> status = this.generator.generateFrom(exp);

        if (status.canBeGenerated()) {
            return prettyPrint(status);
        }

        throw new org.overture.codegen.ir.analysis.AnalysisException(exp.toString()
                + " cannot be code-generated");
    }


    private List<GeneratedModule> prettyPrint(List<IRStatus<PIR>> statuses)
            throws org.overture.codegen.ir.analysis.AnalysisException {
        // Apply merge visitor to pretty print Isabelle syntax
      
        IsaTranslations isa = new IsaTranslations();
        MergeVisitor pp = isa.getMergeVisitor();
        
       
        List<GeneratedModule> generated = new ArrayList<GeneratedModule>();

        for (IRStatus<PIR> status : statuses) {
            if(status.getIrNodeName().equals("VDMToolkit")){
                System.out.println("");
            } else {
                generated.add(prettyPrintNode(pp, status));
            }

        }
        // Return syntax
        return generated;
    }

    
    //feed to velocity monster
    private GeneratedModule prettyPrint(IRStatus<? extends INode> status)
            throws org.overture.codegen.ir.analysis.AnalysisException {
        // Apply merge visitor to pretty print Isabelle syntax
        IsaTranslations isa = new IsaTranslations();
        MergeVisitor pp = isa.getMergeVisitor();
        return prettyPrintNode(pp, status);
    }

    private GeneratedModule prettyPrintNode(MergeVisitor pp,
                                            IRStatus<? extends INode> status)
            throws org.overture.codegen.ir.analysis.AnalysisException {
        INode irClass = status.getIrNode();

        StringWriter sw = new StringWriter();

        irClass.apply(pp, sw);

        if (pp.hasMergeErrors()) {
            return new GeneratedModule(status.getIrNodeName(), irClass, pp.getMergeErrors(), false);
        } else if (pp.hasUnsupportedTargLangNodes()) {
            return new GeneratedModule(status.getIrNodeName(), new HashSet<VdmNodeInfo>(), pp.getUnsupportedInTargLang(), false);
        } else {
            // Code can be generated. Ideally, should format it
            GeneratedModule generatedModule = new GeneratedModule(status.getIrNodeName(), irClass, sw.toString(), false);
            generatedModule.setTransformationWarnings(status.getTransformationWarnings());
            return generatedModule;
        }
    }
}
\end{lstlisting}



\chapter{Testing} \label{testing}
\subsection{IsaGenParamTest}
\begin{lstlisting}
/**
 * Main parameterized test class. Runs tests on modules with minimal
 * definitions to exercise the translation with a single construct
 * at a time.
 *
 * @author ldc
 */
@RunWith(Parameterized.class)
public class IsaGenParamTest extends ParamStandardTest<CgIsaTestResult> {

    public IsaGenParamTest(String nameParameter, String inputParameter,
                           String resultParameter) {
        super(nameParameter, inputParameter, resultParameter);
    }

    private static final String UPDATE = "tests.update.isagen";
    private static final String CGISA_ROOT = "src/test/resources/modules";
    private static final List<String> skippedTests = Arrays.asList();//"NoParamPrePost.vdmsl",
//        "2ParamsPrePost.vdmsl",
//        "NoParamNoPre.vdmsl",
//        "1ParamNoPre.vdmsl","1ParamPrePost.vdmsl",
//        "FuncPrePost.vdmsl",
////        "NotYetSpecified.vdmsl",
//        "FuncPre.vdmsl",
//        "FuncApply3Params.vdmsl",
//        "FuncDecl2Params.vdmsl",
//        "FuncDeclNoParam.vdmsl",
//        "FuncDepSimple.vdmsl",
//        "FuncApplyNoParam.vdmsl",
//        "FuncPost.vdmsl",
//        "FuncApply1Param.vdmsl",
//        "FuncDecl1Param.vdmsl",
//        "EqualsInit.vdmsl","PredicateInit.vdmsl",
//        "IntExpVarExp.vdmsl","ExplicitInt.vdmsl","ExplicitNat.vdmsl","ExplicitNat1.vdmsl",
//        "ExplicitReal.vdmsl","IndependentDefsOrder.vdmsl",
//        "ImplicitNumericExp.vdmsl","VarExp.vdmsl",
//        "SeqNat.vdmsl",
//        "BoolType.vdmsl",
//        "InvSet.vdmsl",
//        "InvRecordDummyInv.vdmsl",
//        "InvInt.vdmsl",
//        "Rec2Fields.vdmsl","SeqInt.vdmsl","Real.vdmsl","CharSeqIntSetTuple.vdmsl","IntIntTuple.vdmsl",
//        "MapIntChar.vdmsl",
//        "Char.vdmsl",
//        "Rec1Field.vdmsl",
//        "IntCharTuple.vdmsl","Token.vdmsl",
//        "CharNatTokenTuple.vdmsl","Rat.vdmsl","SetInt.vdmsl","Nat.vdmsl","Nat1.vdmsl",
//        "Rec2FieldsDiffTypes.vdmsl");//,// "MapIntInt.vdmsl");

    @Override
    public CgIsaTestResult processModel(List<INode> ast) {
        IsaGen gen = new IsaGen();
        GeneratedData genData = null;

        try {
            genData = gen.generate(ast);
        } catch (AnalysisException e) {
            fail("Could not process test file " + testName);
        }

        List<AModuleModules> classes = new LinkedList<>();
        for (INode n : ast) {
            classes.add((AModuleModules) n);
        }

        List<GeneratedModule> result = null;
            result = genData.getClasses();
            if (!result.get(0).canBeGenerated()) {
                StringBuilder sb = new StringBuilder();
                sb.append(result.get(0).getMergeErrors());
                sb.append(result.get(0).getUnsupportedInIr());
                sb.append(result.get(0).getUnsupportedInTargLang());
                fail(sb.toString());
            }

        return CgIsaTestResult.convert(result);
    }

    @Parameters(name = "{index} : {0}")
    public static Collection<Object[]> testData() {
        return PathsProvider.computePaths(CGISA_ROOT);
    }

    @Override
    public Type getResultType() {
        Type resultType = new TypeToken<CgIsaTestResult>() {
        }.getType();
        return resultType;
    }

    @Override
    protected String getUpdatePropertyString() {
        return UPDATE;
    }

    @Override
    public void compareResults(CgIsaTestResult actual, CgIsaTestResult expected) {
        assertTrue("\n --- Expected: ---\n" + expected.translation
                + "\n --- Got: ---\n" + actual.translation, expected.compare(actual));
        if(expected.compare(actual))
        {
            System.out.println("\n --- Got: ---\n" + actual.translation);
        }

    }

    @Override
    protected void checkAssumptions() {
        Assume.assumeTrue("Test in skip list.",notSkipped());
    }

    private boolean notSkipped() {
        return !skippedTests.contains(testName);
    }
}

\end{lstlisting}

\subsection{IsaGenModelTest} \label{IsaGenModelTest}
\begin{lstlisting}
/*
 * #%~
 * VDM to Isabelle Translation
 * %%
 * Copyright (C) 2008 - 2015 Overture
 * %%
 * This program is free software: you can redistribute it and/or modify
 * it under the terms of the GNU General Public License as
 * published by the Free Software Foundation, either version 3 of the
 * License, or (at your option) any later version.
 * 
 * This program is distributed in the hope that it will be useful,
 * but WITHOUT ANY WARRANTY; without even the implied warranty of
 * MERCHANTABILITY or FITNESS FOR A PARTICULAR PURPOSE.  See the
 * GNU General Public License for more details.
 * 
 * You should have received a copy of the GNU General Public
 * License along with this program.  If not, see
 * <http://www.gnu.org/licenses/gpl-3.0.html>.
 * #~%
 */
package org.overturetool.cgisa;

import java.util.Arrays;
import java.util.Collection;
import java.util.List;

import org.junit.Assume;
import org.junit.Ignore;
import org.junit.runner.RunWith;
import org.junit.runners.Parameterized;
import org.junit.runners.Parameterized.Parameters;
import org.overture.core.testing.PathsProvider;

/**
 * Main integration test class. Runs tests on complete models.
 * 
 * @author ldc
 */
@RunWith(Parameterized.class)

public class IsaGenModelTest extends IsaGenParamTest
{

    public IsaGenModelTest(String nameParameter, String inputParameter,
            String resultParameter)
    {
        super(nameParameter, inputParameter, resultParameter);
    }

    private static final String UPDATE = "tests.update.isagen.model";
    private static final String MODELS_ROOT = "src/test/resources/models";
    private static final List<String> skippedTests = Arrays.asList();//"CustomAlarm.vdmsl","dummy.vdmsl","Alarm1.vdmsl");


    @Parameters(name = "{index} : {0}")
    public static Collection<Object[]> testData()
    {
        return PathsProvider.computePaths(MODELS_ROOT);
    }

    @Override
    protected String getUpdatePropertyString()
    {
        return UPDATE;
    }

    protected void checkAssumptions() {
        Assume.assumeTrue("Test in skip list.",notSkipped());
    }

    private boolean notSkipped() {
        return !skippedTests.contains(testName);
    }


}

\end{lstlisting}

\section{VDM-SL Test Files} \label{vdmsl}
\subsection{FuncDecl1Param.vdmsl / .vdmsl.result}
\begin{lstlisting}
module A

definitions

functions

f : nat -> nat
f(x) == x;

end A
\end{lstlisting}
\hfill\break
\begin{lstlisting}
{"translation":"theory A\n  imports VDMToolkit\n
begin\n\n
definition \n
f :: \"VDMNat \\<Rightarrow> VDMNat\"\n
where \n
\"f x \\<equiv> x\"\n

definition \n
pre_f \:\: \"VDMNat \\<Rightarrow> \\<bool>\"\n
where \n
\"pre_f x \\<equiv> isa_invVDMNat x\"\n

definition\n
post_f \:\: \"VDMNat \\<Rightarrow> VDMNat \\<Rightarrow> \\<bool>\"\n
where\n
\" post_f x RESULT \\<equiv> (isa_invVDMNat x \\<and> isa_invVDMNat RESULT)\"\n

 
\n\nend","errors":false}
\end{lstlisting}

\subsection{FuncApply3Params.vdmsl / .vdmsl.result}
\begin{lstlisting}
module A

definitions

functions
f : int * int * int -> int
f (x,y,z) == 0;

values
x = f(1,2,3);

end A
\end{lstlisting}

\begin{lstlisting}
{"translation":"\ntheory A\nimports VDMToolkit\nbegin

definition\nf \:\: \"VDMInt \\<Rightarrow> VDMInt \\<Rightarrow> VDMInt \\<Rightarrow> VDMInt\"\nwhere\n\"f x y z  \\<equiv> 0\"\n\nabbreviation\n x \:\: VDMInt\n where\n\"x \\<equiv> f 1 2 3\"

definition\ninv_x \:\: \"\\<bool>\"\nwhere\n\"inv_x  \\<equiv> isa_invTrue x\"\n\ndefinition\n    pre_f \:\: \"VDMInt \\<Rightarrow> VDMInt \\<Rightarrow> VDMInt \\<Rightarrow> \\<bool>\"\nwhere\n\"pre_f x y z  \\<equiv> (isa_invTrue x \\<and> (isa_invTrue y \\<and> isa_invTrue z))\"\n\ndefinition\npost_f \:\: \"VDMInt\\<Rightarrow> VDMInt\\<Rightarrow> VDMInt\\<Rightarrow> VDMInt\\<Rightarrow> \\<bool>\"\nwhere\n\"post_f x y z RESULT  \\<equiv> (isa_invTrue x \\<and> (isa_invTrue y \\<and> (isa_invTrue z \\<and> isa_invTrue RESULT)))\"\n\nend","errors":false}
\end{lstlisting}

\subsection{NotYetSpecified.vdmsl / .vdmsl.result}
\begin{lstlisting}
functions

f : int -> token
f (x) == is not yet specified
\end{lstlisting}

\begin{lstlisting}
{"translation":"
theory DEFAULT
  imports VDMToolkit
begin


definition
    f \:\: \"VDMInt
 \\<Rightarrow> VDMToken
\"
    where
    \"f x  \\<equiv> undef\"


definition
    pre_f \:\: \"VDMInt
 \\<Rightarrow> \\<bool>\"
    where
    \"pre_f x  \\<equiv> isa_invTrue x\"


definition
    post_f \:\: \"VDMInt
 \\<Rightarrow> VDMToken
 \\<Rightarrow> \\<bool>\"
    where
    \"post_f x RESULT  \\<equiv> (isa_invTrue x \\<and> isa_invTrue RESULT)\"

end","errors":false}
\end{lstlisting}

\subsection{1ParamNoPre.vdmsl / .vdmsl.result}
\begin{lstlisting}
functions

f (x:int) r: int
post r = x
\end{lstlisting}

\begin{lstlisting}
{"translation":"theory DEFAULT\n  imports VDMToolkit\nbegin\n\n

definition \n
pre_f \:\: \"VDMInt \\<Rightarrow> \\<bool>\"\n
where \n
\"pre_f x \\<equiv> isa_invTrue x \"\n

definition\n
post_f \:\: \"VDMInt \\<Rightarrow> VDMInt \\<Rightarrow> \\<bool>\"\n
where\n
\"post_f x r \\<equiv> ((isa_invTrue x \\<and> isa_invTrue r) \\<and> (r = x))\"



\n\nend","errors":false}
\end{lstlisting}

\subsection{NoParamNoPre.vdmsl / .vmdsl.result}
\begin{lstlisting}
functions

f () r: int
post true
\end{lstlisting}

\begin{lstlisting}
{"translation":"theory DEFAULT\n  imports VDMToolkit\nbegin\n\n

definition \n
unimplemented_pre_f \:\: \"\\<bool>\"\n
where \n
\"unimplemented_pre_f \\<equiv> undef\"\n

definition\n
post_f \:\: \"VDMInt \\<Rightarrow> \\<bool>\"\n
where\n
\" post_f r \\<equiv> true\"\n



\n\n end","errors":false}
\end{lstlisting}

\subsection{PredicateInit.vdmsl / .vdmsl.result}
\begin{lstlisting}
state S of
  x : nat
  init s == s.x >0
end
\end{lstlisting}

\begin{lstlisting}
{"translation":"
theory DEFAULT
  imports VDMToolkit
begin

record S =
        s_x \:\: VDMNat

    

definition
    inv_S \:\: \"S \\<Rightarrow> \\<bool>\"
    where
    \"inv_S s \\<equiv> isa_invVDMNat (s_x s)\"


definition
    init_S \:\: \"S \\<Rightarrow> \\<bool>\"
    where
    \"init_S s \\<equiv> ((s_x s) > 0)\"


definition
    post_init_S \:\: \"VDMNat
 \\<Rightarrow> \\<bool>\"
    where
    \"post_init_S  \\<equiv> inv_S init_S\"

end","errors":false}
\end{lstlisting}

\subsection{EqualsInit.vdmsl / .vdmsl.result}
\begin{lstlisting}
state S of
  x : nat
  init s == s = mk_S(0)
end
\end{lstlisting}

\begin{lstlisting}
{"translation":"
theory DEFAULT
  imports VDMToolkit
begin

record S =
        s_x \:\: VDMNat

    

definition
    inv_S \:\: \"S \\<Rightarrow> \\<bool>\"
    where
    \"inv_S s \\<equiv> isa_invVDMNat (s_x s)\"


definition
    init_S \:\: \"S \\<Rightarrow> \\<bool>\"
    where
    \"init_S s \\<equiv> (s = (| x = 0 |))\"


definition
    post_init_S \:\: \"VDMNat
 \\<Rightarrow> \\<bool>\"
    where
    \"post_init_S  \\<equiv> inv_S init_S\"

end","errors":false}
\end{lstlisting}

\subsection{InvSet.vdmsl / .vdmsl.result}
\begin{lstlisting}
types

t = set of int
inv t == t <> {}
\end{lstlisting}

\begin{lstlisting}
{"translation":"theory DEFAULT\n  imports VDMToolkit\nbegin\n\ntype_synonym t \u003d \"VDMInt VDMSet\"\n\n\n\n\ndefinition\n    inv_t :: \"(t) \\\u003cRightarrow\u003e \\\u003cbool\u003e\"\n    where\n    \"inv_t t \\\u003cequiv\u003e (isa_invTrue t \\<and> (t <> {}))\"\n\nend","errors":false}
\end{lstlisting}

\subsection{InvRecordDummyInv.vdmsl / .vdmsl.result}
\begin{lstlisting}
types
    A :: b : seq of char
         c : int
    inv a == a.c > 1
\end{lstlisting}

\begin{lstlisting}
{"translation":"theory DEFAULT
  imports VDMToolkit
begin

record A =
        a_b \:\: char VDMSeq

        a_c \:\: VDMInt

    



definition
    inv_A \:\: \"A \\<Rightarrow> \\<bool>\"
    where
    \"inv_A a \\<equiv> ((isa_invSeqElems isa_invTrue (a_b a) \\<and> isa_invTrue (a_c a)) \\<and> ((a_c a) > 1))\"

end

","errors":false}
\end{lstlisting}

\subsection{Token.vdmsl / .vdmsl.result}
\begin{lstlisting}
types

t = token
\end{lstlisting}

\begin{lstlisting}
{"translation":"theory DEFAULT\n  imports VDMToolkit\nbegin\n\ntype_synonym t \u003d \"VDMToken\"\n\n\n\n\ndefinition\n    inv_t :: \"(t) \\\u003cRightarrow\u003e \\\u003cbool\u003e\"\n    where\n    \"inv_t t \\\u003cequiv\u003e isa_invTrue t\"\n\nend","errors":false}
\end{lstlisting}

\subsection{SeqNat.vdmsl / .vdmsl.result}
\begin{lstlisting}
types

t = seq of nat
\end{lstlisting}

\begin{lstlisting}
{"translation":"theory DEFAULT\n  imports VDMToolkit\nbegin\n\ntype_synonym t \u003d \"VDMNat VDMSeq\"\n\n\n\n\ndefinition\n    inv_t :: \"(t) \\\u003cRightarrow\u003e \\\u003cbool\u003e\"\n    where\n    \"inv_t t \\\u003cequiv\u003e isa_invTrue t\"\n\nend","errors":false}
\end{lstlisting}

\subsection{Rec2FieldsDiffTypes.vdmsl / .vdmsl.result}
\begin{lstlisting}
types

RecType :: x : char
           y: realR
\end{lstlisting}

\begin{lstlisting}
{"translation":"
theory DEFAULT
  imports VDMToolkit
begin

record RecType =
        recType_x \:\: VDMNat

        recType_y \:\: VDMNat

    



definition
    inv_RecType \:\: \"RecType \\<Rightarrow> \\<bool>\"
    where
    \"inv_RecType r \\<equiv> (isa_invVDMNat (recType_x r) \\<and> isa_invVDMNat (recType_y r))\"

end","errors":false}
\end{lstlisting}

\subsection{CharNatTokenTuple.vdmsl / .vdmsl.result} \label{charnat}
\begin{lstlisting}
types
t = char * nat * token
\end{lstlisting}

\begin{lstlisting}
{"translation":"theory DEFAULT\n  imports VDMToolkit\nbegin\n\ntype_synonym t \u003d \"char * VDMNat * VDMToken\"\n\n\n\n\ndefinition\n    inv_t :: \"(t) \\\u003cRightarrow\u003e \\\u003cbool\u003e\"\n    where\n    \"inv_t t \\\u003cequiv\u003e isa_invTrue t\"\n\nend","errors":false}
\end{lstlisting}

\subsection{NestedVarExp.vdmsl / .vdmsl.result}
\begin{lstlisting}
values
x = {[1]};
y = x;

\end{lstlisting}

\begin{lstlisting}
{"translation":"theory DEFAULT\n  imports VDMToolkit
begin

abbreviation
 x \:\: VDMNat1 VDMSeq VDMSet  where 
\" x \\<equiv> {[1]} \" 
\nabbreviation
 y \:\: VDMNat1 VDMSeq VDMSet where 
\" y \\<equiv> x \" 

definition
inv_x \:\: \"\\<bool>\" where
\"inv_x \\<equiv> isa_invSetElems isa_invSeqElems isa_invVDMNat1 x\"

definition
inv_y \:\: \"\\<bool>\" where
\"inv_y \\<equiv> isa_invSetElems isa_invSeqElems isa_invVDMNat1 y\"

end","errors":false}
\end{lstlisting}

\subsection{IndependentDefsOrder.vdmsl / .vdmsl.result}
\begin{lstlisting}
values
a = 10;
b = 20;
c = 30;
\end{lstlisting}

\begin{lstlisting}
{"translation":"theory DEFAULT\n  imports VDMToolkit
begin

abbreviation
 a \:\: VDMNat1 where 
\" a \\<equiv> 10 \" 
\n
abbreviation
 b \:\: VDMNat1 where 
\" b \\<equiv> 20 \" 
\n
abbreviation
 c \:\: VDMNat1 where 
\" c \\<equiv> 30 \" 
\n\n
definition
inv_a \:\: \"\\<bool>\" where
\"inv_a \\<equiv> isa_invVDMNat1 a\"
\n
definition
inv_b \:\: \"\\<bool>\" where
\"inv_b \\<equiv> isa_invVDMNat1 b\"
\n
definition
inv_c \:\: \"\\<bool>\" where
\"inv_c \\<equiv> isa_invVDMNat1 c\"
\n\n
end","errors":false}
\end{lstlisting}

\subsection{ImplicitNumericExp.vdmsl / .vdmsl.result}
\begin{lstlisting}
values
x = 1;
\end{lstlisting}

\begin{lstlisting}
{"translation":"theory DEFAULT\n  imports VDMToolkit
begin

abbreviation
 x \:\: VDMNat1 where 
\" x \\<equiv> 1 \" 
\n
definition
inv_x \:\: \"\\<bool>\" where
\"inv_x \\<equiv> isa_invVDMNat1 x\"
\n\n
end","errors":false}
\end{lstlisting}

\subsection{ExplicitNat.vdmsl / .vdmsl.result}
\begin{lstlisting}
values
x : nat = 1;

\end{lstlisting}

\begin{lstlisting}
{"translation":"theory DEFAULT\n  imports VDMToolkit
begin

abbreviation
 x \:\: VDMNat where 
\" x \\<equiv> 1 \" 

definition
inv_x \:\: \"\\<bool>\" where
\"inv_x \\<equiv> isa_invVDMNat x\"
\n\n
end","errors":false}
\end{lstlisting}

\subsection{BoolType.vdmsl / .vdmsl.result}
\begin{lstlisting}
values
x : bool = true;

\end{lstlisting}

\begin{lstlisting}
{"translation":"theory DEFAULT\n  imports VDMToolkit\nbegin\n\nabbreviation\n x \:\: \\<bool>\nwhere \" x \\<equiv> true \"\n

definition
inv_x \:\: \"\\<bool>\" where
\"inv_x \\<equiv> isa_invTrue x\"
\n\n

 \n\nend","errors":false}
\end{lstlisting}

\subsection{MapIntChar.vdmsl / .vdmsl.result} \label{mapintchar}
\begin{lstlisting}
types

t = map int to char
\end{lstlisting}

\begin{lstlisting}
{"translation":"theory DEFAULT\n  imports VDMToolkit\nbegin\n\ntype_synonym t \u003d \"VDMInt \\<rightharpoonup> char\"\n\n\n\n\ndefinition\n    inv_t :: \"(t) \\\u003cRightarrow\u003e \\\u003cbool\u003e\"\n    where\n    \"inv_t t \\\u003cequiv\u003e isa_invTrue t\"\n\nend","errors":false}
\end{lstlisting}

\chapter{Apache Velocity} \label{velocity}
\section{Velocity Before Development} \label{Velocitybefore}
\subsection{AFuncDeclIR}
\begin{lstlisting}[language=Velocity]
#macro ( transIdentifiers $node )
#foreach($p in $node.FormalParams)
$Isa.trans($p.pattern)##
#end
#end

definition
    $node.Name :: "$Isa.transTypeParams($node.MethodType.Params) \<RightArrow> $Isa.trans($node.MethodType.Result)"
    where
    "$node.Name #transIdentifiers($node) \<equiv> $Isa.trans($node.Body)"

\end{lstlisting}
\section{Velocity After Development} \label{Velocityafter}
\subsection{AFuncDeclIR}
\begin{lstlisting}[language=Java]
#macro ( transIdentifiers $node )
#foreach($p in $node.FormalParams)
$Isa.trans($p.pattern)##
#end
#end

definition
#if ("$Isa.transTypeParams($node.MethodType.Params)" == "")
  $node.Name :: "$Isa.trans($node.MethodType.Result)"
#else
  $node.Name :: "$Isa.transTypeParams($node.MethodType.Params) \<Rightarrow> $Isa.trans($node.MethodType.Result)"
#end
    where
    "$node.Name #transIdentifiers($node) \<equiv> $Isa.trans($node.Body)"
\end{lstlisting}

\chapter{Results} \label{results}
\subsection{Individual Construct Translation} \label{ict}
\begin{lstlisting}[language=Isabelle]
 --- Got: ---
theory DEFAULT
  imports VDMToolkit
begin

type_synonym t = "VDMInt
 \<rightharpoonup> VDMInt
"



definition
    inv_t :: "(t) \<Rightarrow> \<bool>"
    where
    "inv_t t \<equiv> isa_invTrue t"

end






 --- Got: ---
theory DEFAULT
  imports VDMToolkit
begin

record RecType =
        recType_x :: char
        recType_y :: \<real>
    



definition
    inv_RecType :: "RecType \<Rightarrow> \<bool>"
    where
    "inv_RecType r \<equiv> (isa_invTrue (recType_x r) \<and> isa_invTrue (recType_y r))"

end




 --- Got: ---
theory DEFAULT
  imports VDMToolkit
begin

type_synonym XType = "VDMInt
"



definition
    inv_XType :: "(XType) \<Rightarrow> \<bool>"
    where
    "inv_XType x \<equiv> isa_invTrue x"

end




 --- Got: ---
theory DEFAULT
  imports VDMToolkit
begin

type_synonym t = "VDMNat1
"



definition
    inv_t :: "(t) \<Rightarrow> \<bool>"
    where
    "inv_t t \<equiv> isa_invTrue t"

end




 --- Got: ---
theory DEFAULT
  imports VDMToolkit
begin

type_synonym t = "VDMNat
"



definition
    inv_t :: "(t) \<Rightarrow> \<bool>"
    where
    "inv_t t \<equiv> isa_invTrue t"

end




 --- Got: ---
theory DEFAULT
  imports VDMToolkit
begin

type_synonym t = "VDMInt
 VDMSet
"



definition
    inv_t :: "(t) \<Rightarrow> \<bool>"
    where
    "inv_t t \<equiv> isa_invTrue t"

end




 --- Got: ---
theory DEFAULT
  imports VDMToolkit
begin

type_synonym t = "\<rat>"



definition
    inv_t :: "(t) \<Rightarrow> \<bool>"
    where
    "inv_t t \<equiv> isa_invTrue t"

end



 --- Got: ---
theory DEFAULT
  imports VDMToolkit
begin

type_synonym t = "char * VDMNat
 * VDMToken
"



definition
    inv_t :: "(t) \<Rightarrow> \<bool>"
    where
    "inv_t t \<equiv> isa_invTrue t"

end

 --- Got: ---
theory DEFAULT
  imports VDMToolkit
begin

type_synonym t = "VDMToken
"



definition
    inv_t :: "(t) \<Rightarrow> \<bool>"
    where
    "inv_t t \<equiv> isa_invTrue t"

end




 --- Got: ---
theory DEFAULT
  imports VDMToolkit
begin

type_synonym t = "VDMInt
 * char"



definition
    inv_t :: "(t) \<Rightarrow> \<bool>"
    where
    "inv_t t \<equiv> isa_invTrue t"

end





 --- Got: ---
theory DEFAULT
  imports VDMToolkit
begin

record RecType =
        recType_x :: VDMNat

    



definition
    inv_RecType :: "RecType \<Rightarrow> \<bool>"
    where
    "inv_RecType r \<equiv> isa_invVDMNat (recType_x r)"

end




 --- Got: ---
theory DEFAULT
  imports VDMToolkit
begin

type_synonym t = "char"



definition
    inv_t :: "(t) \<Rightarrow> \<bool>"
    where
    "inv_t t \<equiv> isa_invTrue t"

end




 --- Got: ---
theory DEFAULT
  imports VDMToolkit
begin

type_synonym t = "VDMInt
 \<rightharpoonup> char"



definition
    inv_t :: "(t) \<Rightarrow> \<bool>"
    where
    "inv_t t \<equiv> isa_invTrue t"

end




 --- Got: ---
theory DEFAULT
  imports VDMToolkit
begin

type_synonym t = "VDMInt
 * VDMInt
"



definition
    inv_t :: "(t) \<Rightarrow> \<bool>"
    where
    "inv_t t \<equiv> isa_invTrue t"

end




 --- Got: ---
theory DEFAULT
  imports VDMToolkit
begin

type_synonym t = "char VDMSeq
 * VDMInt
 VDMSet
"



definition
    inv_t :: "(t) \<Rightarrow> \<bool>"
    where
    "inv_t t \<equiv> isa_invTrue t"

end




 --- Got: ---
theory DEFAULT
  imports VDMToolkit
begin

type_synonym t = "\<real>"



definition
    inv_t :: "(t) \<Rightarrow> \<bool>"
    where
    "inv_t t \<equiv> isa_invTrue t"

end




 --- Got: ---
theory DEFAULT
  imports VDMToolkit
begin

type_synonym t = "VDMNat
 VDMSeq
"



definition
    inv_t :: "(t) \<Rightarrow> \<bool>"
    where
    "inv_t t \<equiv> isa_invTrue t"

end




 --- Got: ---
theory DEFAULT
  imports VDMToolkit
begin

type_synonym t = "VDMInt
 VDMSeq
"



definition
    inv_t :: "(t) \<Rightarrow> \<bool>"
    where
    "inv_t t \<equiv> isa_invTrue t"

end






 --- Got: ---
theory DEFAULT
  imports VDMToolkit
begin

record RecType =
        recType_x :: VDMNat

        recType_y :: VDMNat

    



definition
    inv_RecType :: "RecType \<Rightarrow> \<bool>"
    where
    "inv_RecType r \<equiv> (isa_invVDMNat (recType_x r) \<and> isa_invVDMNat (recType_y r))"

end




 --- Got: ---
theory DEFAULT
  imports VDMToolkit
begin

type_synonym t = "VDMInt
"



definition
    inv_t :: "(t) \<Rightarrow> \<bool>"
    where
    "inv_t t \<equiv> (isa_invTrue t \<and> (t > 0))"

end






 --- Got: ---
theory DEFAULT
  imports VDMToolkit
begin

record A =
        a_b :: char VDMSeq

        a_c :: VDMInt

    



definition
    inv_A :: "A \<Rightarrow> \<bool>"
    where
    "inv_A a \<equiv> ((isa_invSeqElems isa_invTrue (a_b a) \<and> isa_invTrue (a_c a)) \<and> ((a_c a) > 1))"

end




 --- Got: ---
theory DEFAULT
  imports VDMToolkit
begin

type_synonym t = "VDMInt
 VDMSet
"



definition
    inv_t :: "(t) \<Rightarrow> \<bool>"
    where
    "inv_t t \<equiv> (isa_invTrue t \<and> (t <> {}))"

end



 --- Got: ---
theory DEFAULT
  imports VDMToolkit
begin

abbreviation
 x :: \<bool> where
"x \<equiv> true"


definition
    inv_x :: "\<bool>"
    where
    "inv_x  \<equiv> isa_invTrue x"

end



 --- Got: ---
theory DEFAULT
  imports VDMToolkit
begin

abbreviation
 x :: VDMNat1
 where
"x \<equiv> 1"

abbreviation
 y :: VDMNat1
 where
"y \<equiv> x"


definition
    inv_x :: "\<bool>"
    where
    "inv_x  \<equiv> isa_invVDMNat1 x"


definition
    inv_y :: "\<bool>"
    where
    "inv_y  \<equiv> isa_invVDMNat1 y"

end


 --- Got: ---
theory DEFAULT
  imports VDMToolkit
begin

abbreviation
 x :: VDMNat1
 where
"x \<equiv> 1"


definition
    inv_x :: "\<bool>"
    where
    "inv_x  \<equiv> isa_invVDMNat1 x"

end


 --- Got: ---
theory DEFAULT
  imports VDMToolkit
begin

abbreviation
 a :: VDMNat1
 where
"a \<equiv> 10"

abbreviation
 b :: VDMNat1
 where
"b \<equiv> 20"

abbreviation
 c :: VDMNat1
 where
"c \<equiv> 30"


definition
    inv_a :: "\<bool>"
    where
    "inv_a  \<equiv> isa_invVDMNat1 a"


definition
    inv_b :: "\<bool>"
    where
    "inv_b  \<equiv> isa_invVDMNat1 b"


definition
    inv_c :: "\<bool>"
    where
    "inv_c  \<equiv> isa_invVDMNat1 c"

end




 --- Got: ---
theory DEFAULT
  imports VDMToolkit
begin

abbreviation
 x :: \<real> where
"x \<equiv> 1.2"


definition
    inv_x :: "\<bool>"
    where
    "inv_x  \<equiv> isa_invTrue x"

end



 --- Got: ---
theory DEFAULT
  imports VDMToolkit
begin

abbreviation
 x :: VDMNat1
 where
"x \<equiv> 1"


definition
    inv_x :: "\<bool>"
    where
    "inv_x  \<equiv> isa_invVDMNat1 x"

end

 --- Got: ---
theory DEFAULT
  imports VDMToolkit
begin

abbreviation
 x :: VDMNat1
 VDMSeq
 VDMSet
 where
"x \<equiv> {[1]}"

abbreviation
 y :: VDMNat1
 VDMSeq
 VDMSet
 where
"y \<equiv> x"


definition
    inv_x :: "\<bool>"
    where
    "inv_x  \<equiv> isa_invSetElems isa_invSeqElems isa_invVDMNat1 x"


definition
    inv_y :: "\<bool>"
    where
    "inv_y  \<equiv> isa_invSetElems isa_invSeqElems isa_invVDMNat1 y"

end




 --- Got: ---
theory DEFAULT
  imports VDMToolkit
begin

abbreviation
 x :: VDMNat
 where
"x \<equiv> 1"


definition
    inv_x :: "\<bool>"
    where
    "inv_x  \<equiv> isa_invVDMNat x"

end




 --- Got: ---
theory DEFAULT
  imports VDMToolkit
begin

abbreviation
 x :: VDMInt
 where
"x \<equiv> 1"


definition
    inv_x :: "\<bool>"
    where
    "inv_x  \<equiv> isa_invTrue x"

end


 --- Got: ---
theory DEFAULT
  imports VDMToolkit
begin

abbreviation
 x :: VDMNat1
 where
"x \<equiv> 1"

abbreviation
 y :: VDMNat1
 where
"y \<equiv> x"


definition
    inv_x :: "\<bool>"
    where
    "inv_x  \<equiv> isa_invVDMNat1 x"


definition
    inv_y :: "\<bool>"
    where
    "inv_y  \<equiv> isa_invVDMNat1 y"

end





 --- Got: ---
theory A
  imports VDMToolkit
begin


definition
    f :: "VDMNat
 \<Rightarrow> VDMNat
"
    where
    "f x  \<equiv> x"


definition
    pre_f :: "VDMNat
 \<Rightarrow> \<bool>"
    where
    "pre_f x  \<equiv> isa_invVDMNat x"


definition
    post_f :: "VDMNat
 \<Rightarrow> VDMNat
 \<Rightarrow> \<bool>"
    where
    "post_f xRESULT \<equiv> (isa_invVDMNat x \<and> isa_invVDMNat RESULT)"

end



f(1)


 --- Got: ---
theory A
  imports VDMToolkit
begin


definition
    f :: "VDMInt
 \<Rightarrow> VDMInt
"
    where
    "f x  \<equiv> 0"

abbreviation
 x :: VDMInt
 where
"x \<equiv> f 1"


definition
    inv_x :: "\<bool>"
    where
    "inv_x  \<equiv> isa_invTrue x"


definition
    pre_f :: "VDMInt
 \<Rightarrow> \<bool>"
    where
    "pre_f x  \<equiv> isa_invTrue x"


definition
    post_f :: "VDMInt
 \<Rightarrow> VDMInt
 \<Rightarrow> \<bool>"
    where
    "post_f xRESULT \<equiv> (isa_invTrue x \<and> isa_invTrue RESULT)"

end




 --- Got: ---
theory A
  imports VDMToolkit
begin


definition
    f :: "VDMInt
 \<Rightarrow> VDMInt
"
    where
    "f x  \<equiv> 0"


definition
    pre_f :: "VDMInt
 \<Rightarrow> \<bool>"
    where
    "pre_f x  \<equiv> isa_invTrue x"


definition
    post_f :: "VDMInt
 \<Rightarrow> VDMInt
 \<Rightarrow> \<bool>"
    where
    "post_f xRESULT \<equiv> ((isa_invTrue x \<and> isa_invTrue RESULT) \<and> true)"

end



f()


 --- Got: ---
theory A
  imports VDMToolkit
begin


definition
    f :: "VDMInt
"
    where
    "f  \<equiv> 0"

abbreviation
 x :: VDMInt
 where
"x \<equiv> f "


definition
    inv_x :: "\<bool>"
    where
    "inv_x  \<equiv> isa_invTrue x"


definition
    unimplemented_pre_f :: "\<bool>"
    where
    "unimplemented_pre_f  \<equiv> undef"


definition
    unimplemented_post_f :: "VDMInt
 \<Rightarrow> \<bool>"
    where
    "unimplemented_post_f  \<equiv> undef"

end





 --- Got: ---
theory A
  imports VDMToolkit
begin


definition
    g :: "VDMInt
 \<Rightarrow> VDMInt
"
    where
    "g x  \<equiv> 0"


definition
    f :: "VDMInt
 \<Rightarrow> VDMInt
"
    where
    "f x  \<equiv> g x"


definition
    pre_g :: "VDMInt
 \<Rightarrow> \<bool>"
    where
    "pre_g x  \<equiv> isa_invTrue x"


definition
    post_g :: "VDMInt
 \<Rightarrow> VDMInt
 \<Rightarrow> \<bool>"
    where
    "post_g xRESULT \<equiv> (isa_invTrue x \<and> isa_invTrue RESULT)"


definition
    pre_f :: "VDMInt
 \<Rightarrow> \<bool>"
    where
    "pre_f x  \<equiv> isa_invTrue x"


definition
    post_f :: "VDMInt
 \<Rightarrow> VDMInt
 \<Rightarrow> \<bool>"
    where
    "post_f xRESULT \<equiv> (isa_invTrue x \<and> isa_invTrue RESULT)"

end




 --- Got: ---
theory A
  imports VDMToolkit
begin


definition
    f :: "VDMNat
"
    where
    "f  \<equiv> 0"


definition
    unimplemented_pre_f :: "\<bool>"
    where
    "unimplemented_pre_f  \<equiv> undef"


definition
    unimplemented_post_f :: "VDMNat
 \<Rightarrow> \<bool>"
    where
    "unimplemented_post_f  \<equiv> undef"

end




 --- Got: ---
theory A
  imports VDMToolkit
begin


definition
    f :: "VDMNat
 \<Rightarrow> VDMNat
 \<Rightarrow> VDMNat
"
    where
    "f x y  \<equiv> x"


definition
    pre_f :: "VDMNat
 \<Rightarrow> VDMNat
 \<Rightarrow> \<bool>"
    where
    "pre_f x y  \<equiv> (isa_invVDMNat x \<and> isa_invVDMNat y)"


definition
    post_f :: "VDMNat
 \<Rightarrow> VDMNat
 \<Rightarrow> VDMNat
 \<Rightarrow> \<bool>"
    where
    "post_f xyRESULT \<equiv> (isa_invVDMNat x \<and> (isa_invVDMNat y \<and> isa_invVDMNat RESULT))"

end



f(1, 2, 3)


 --- Got: ---
theory A
  imports VDMToolkit
begin


definition
    f :: "VDMInt
 \<Rightarrow> VDMInt
 \<Rightarrow> VDMInt
 \<Rightarrow> VDMInt
"
    where
    "f x y z  \<equiv> 0"

abbreviation
 x :: VDMInt
 where
"x \<equiv> f 1 2 3"


definition
    inv_x :: "\<bool>"
    where
    "inv_x  \<equiv> isa_invTrue x"


definition
    pre_f :: "VDMInt
 \<Rightarrow> VDMInt
 \<Rightarrow> VDMInt
 \<Rightarrow> \<bool>"
    where
    "pre_f x y z  \<equiv> (isa_invTrue x \<and> (isa_invTrue y \<and> isa_invTrue z))"


definition
    post_f :: "VDMInt
 \<Rightarrow> VDMInt
 \<Rightarrow> VDMInt
 \<Rightarrow> VDMInt
 \<Rightarrow> \<bool>"
    where
    "post_f xyzRESULT \<equiv> (isa_invTrue x \<and> (isa_invTrue y \<and> (isa_invTrue z \<and> isa_invTrue RESULT)))"

end




 --- Got: ---
theory A
  imports VDMToolkit
begin


definition
    f :: "VDMInt
 \<Rightarrow> VDMInt
"
    where
    "f x  \<equiv> 0"


definition
    pre_f :: "VDMInt
 \<Rightarrow> \<bool>"
    where
    "pre_f x  \<equiv> (isa_invTrue x \<and> true)"


definition
    post_f :: "VDMInt
 \<Rightarrow> VDMInt
 \<Rightarrow> \<bool>"
    where
    "post_f xRESULT \<equiv> (isa_invTrue x \<and> isa_invTrue RESULT)"

end



 --- Got: ---
theory DEFAULT
  imports VDMToolkit
begin


definition
    f :: "VDMInt
 \<Rightarrow> VDMToken
"
    where
    "f x  \<equiv> undef"


definition
    pre_f :: "VDMInt
 \<Rightarrow> \<bool>"
    where
    "pre_f x  \<equiv> isa_invTrue x"


definition
    post_f :: "VDMInt
 \<Rightarrow> VDMToken
 \<Rightarrow> \<bool>"
    where
    "post_f xRESULT \<equiv> (isa_invTrue x \<and> isa_invTrue RESULT)"

end




 --- Got: ---
theory A
  imports VDMToolkit
begin


definition
    f :: "VDMNat
 \<Rightarrow> VDMInt
"
    where
    "f x  \<equiv> 0"


definition
    pre_f :: "VDMNat
 \<Rightarrow> \<bool>"
    where
    "pre_f x  \<equiv> (isa_invVDMNat x \<and> true)"


definition
    post_f :: "VDMNat
 \<Rightarrow> VDMInt
 \<Rightarrow> \<bool>"
    where
    "post_f xRESULT \<equiv> ((isa_invVDMNat x \<and> isa_invTrue RESULT) \<and> true)"

end




 --- Got: ---
theory DEFAULT
  imports VDMToolkit
begin


definition
    pre_f :: "VDMInt
 \<Rightarrow> \<bool>"
    where
    "pre_f x  \<equiv> (isa_invTrue x \<and> true)"


definition
    post_f :: "VDMInt
 \<Rightarrow> VDMInt
 \<Rightarrow> \<bool>"
    where
    "post_f xr \<equiv> ((isa_invTrue x \<and> isa_invTrue r) \<and> (r = x))"

end




 --- Got: ---
theory DEFAULT
  imports VDMToolkit
begin


definition
    pre_f :: "VDMInt
 \<Rightarrow> \<bool>"
    where
    "pre_f x  \<equiv> isa_invTrue x"


definition
    post_f :: "VDMInt
 \<Rightarrow> VDMInt
 \<Rightarrow> \<bool>"
    where
    "post_f xr \<equiv> ((isa_invTrue x \<and> isa_invTrue r) \<and> (r = x))"

end




 --- Got: ---
theory DEFAULT
  imports VDMToolkit
begin


definition
    unimplemented_pre_f :: "\<bool>"
    where
    "unimplemented_pre_f  \<equiv> undef"


definition
    post_f :: "VDMInt
 \<Rightarrow> \<bool>"
    where
    "post_f r \<equiv> true"

end




 --- Got: ---
theory DEFAULT
  imports VDMToolkit
begin


definition
    pre_f :: "VDMInt
 \<Rightarrow> VDMInt
 \<Rightarrow> \<bool>"
    where
    "pre_f x y  \<equiv> ((isa_invTrue x \<and> isa_invTrue y) \<and> (y <> 0))"


definition
    post_f :: "VDMInt
 \<Rightarrow> VDMInt
 \<Rightarrow> VDMInt
 \<Rightarrow> \<bool>"
    where
    "post_f xyr \<equiv> ((isa_invTrue x \<and> (isa_invTrue y \<and> isa_invTrue r)) \<and> ((x / y) = r))"

end



 --- Got: ---
theory DEFAULT
  imports VDMToolkit
begin


definition
    pre_f :: "\<bool>"
    where
    "pre_f  \<equiv> true"


definition
    post_f :: "VDMInt
 \<Rightarrow> \<bool>"
    where
    "post_f r \<equiv> true"

end
\end{lstlisting}

\end{appendices}


\end{document}
